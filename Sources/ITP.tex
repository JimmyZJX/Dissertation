%%%%%%%%%%%%%%%%%%%%%%%%%%%%%%%%%%%%%%%%%%%%%%%%%%%%%%%%%%%%%%%%%%%%%%%% 
\chapter{Higher-Ranked Polymorphism Subtyping Algorithm}
\label{chap:ITP}
%%%%%%%%%%%%%%%%%%%%%%%%%%%%%%%%%%%%%%%%%%%%%%%%%%%%%%%%%%%%%%%%%%%%%%%% 

In this chapter, we present a new algorithm for polymorphic subtyping
with mechanical formalizations in the Abella theorem prover.
There is little work on formalizing type inference algorithms before,
especially for higher-ranked systems,
due to the fact that environments and variable bindings are
tricky to mechanize in theorem provers.
In order to overcome the difficulty in formalization,
we propose the novel algorithm by means of \emph{worklist judgments}.
Worklist judgments turn complicated
global propagation of unification constraints into simple local substitutions.
Moreover, we exploit several ideas in the recent inductive
formulation of a type-inference algorithm by
\citet{dunfield2013complete}, which turn out to be useful
for mechanization in a theorem prover.

Building on these ideas we develop a complete formalization of
polymorphic subtyping in the Abella theorem prover. Moreover, we
 show that the algorithm is \emph{sound}, \emph{complete}, and \emph{decidable} with
respect to the well-known declarative formulation of polymorphic subtyping by
\citet{odersky1996putting}.
While these meta-theoretical results are not new, as far
as we know our work is the first to mechanically formalize them.


\section{Overview: Polymorphic Subtyping}

% We hope that
% this work encourages other researchers to use theorem provers for formalizing
% type-inference algorithms.


This section discusses Odersky and L\"aufer declarative subtyping rules further in depth,
and identifies the challenges in formalizing a corresponding
algorithmic version. Then the key ideas of our approach that address
those challenges are introduced.

%{Polymorpic Subtyping}
% introduction, a bit about history

\begin{comment}
The subtyping relation is typically
used by the subsumption rule for type inference, e.g.
$$\inferrule*[right=Sub]
  {\Gamma\vdash t : A \quad A\le B}
  {\Gamma\vdash t : B}
$$
where $t$ represents a term, and the relation $\Gamma\vdash t : A$
reads ``term $t$ has type $A$ under context $\Gamma$''.
\end{comment}

%-------------------------------------------------------------------------------
\subsection{Declarative Polymorphic Subtyping}\label{declarative_subtyping}

\begin{figure}[t]
\[
\begin{array}{l@{\qquad}lcl}
\text{Type variables}\qquad&a, b\\[3mm]
\text{Types}\qquad&A, B, C &::=&\quad 1 \mid a \mid \forall a. A \mid A\to B\\
\text{Monotypes}\qquad&\tau &::=&\quad 1 \mid a \mid \tau_1\to \tau_2\\
\text{Contexts}\qquad&\Psi &::=&\quad \cdot \mid \Psi, a\\
%\text{Judgments}\qquad&\exps &::=&\quad \cdot \mid A \le B : \exps
\end{array}
\]
\caption{Syntax of Declarative System}\label{fig:ITP:decl:syntax}
\end{figure}


In implicitly polymorphic type systems, the subtyping relation
compares the degree of polymorphism of types. In short, if a
polymorphic type $A$ can always be instantiated to any instantiation
of $B$, then $A$ is ``at least as polymorphic as'' $B$, or we just say
that $A$ is ``more polymorphic than'' $B$, or $A\le B$.  
%For example,
%by instantiating \verb|a| with \verb|Int|, \verb|forall a. a -> a|
%becomes \verb|Int -> Int|, thus \verb|forall a. a -> a| is a subtype
%of \verb|Int -> Int|.

There is a very simple declarative formulation of
polymorphic subtyping due to \citet{odersky1996putting}. The syntax
of this declarative system is shown in Figure~\ref{fig:ITP:decl:syntax}. Types,
represented by $A, B, C$, are the unit type $1$, type variables
$a, b$, universal quantification $\forall a. A$ and function type
$A\to B$. We allow nested universal quantifiers to appear in
types, but not in monotypes. Contexts $\Psi$ collect a list
of type variables.


\begin{figure}[t]
\centering \framebox{$\Psi \vdash A$}
\begin{gather*}
\inferrule*[right=$\mathtt{wf_d unit}$]
	{~}{\Psi\vdash 1}
\qquad
\inferrule*[right=$\mathtt{wf_d var}$]
	{a\in\Psi}{\Psi\vdash a}
\qquad
\inferrule*[right=$\mathtt{wf_d{\to}}$]
	{\Psi\vdash A\quad \Psi\vdash B}
	{\Psi\vdash A\to B}
\qquad
\inferrule*[right=$\mathtt{wf_d\forall}$]
	{\Psi, a\vdash A}
	{\Psi\vdash \forall a. A}
\end{gather*}

\centering \framebox{$\Psi \vdash A \le B$}
\begin{gather*}
\inferrule*[right=$\mathtt{{\le}Var}$]
	{a\in\Psi}{\Psi\vdash a\le a}
\qquad
\inferrule*[right=$\mathtt{{\le}Unit}$]
	{~}{\Psi \vdash 1 \le 1}
\qquad
\inferrule*[right=$\mathtt{{\le}{\to}}$]
	{\Psi \vdash B_1 \le A_1 \quad \Psi \vdash A_2 \le B_2}
	{\Psi\vdash A_1\to A_2 \le B_1\to B_2}
\\
\inferrule*[right=$\mathtt{{\le}\forall L}$]
	{\Psi\vdash \tau \quad \Psi\vdash [\tau/a] A \le B}
	{\Psi\vdash \forall a. A \le B}
\qquad
\inferrule*[right=$\mathtt{{\le}\forall R}$]
	{\Psi, a\vdash A\le B}
	{\Psi\vdash A \le \forall a. B}
\end{gather*}
\caption{Well-formedness of Declarative Types and Declarative Subtyping}\label{fig:decl:wf_sub}
\end{figure}

In Figure~\ref{fig:decl:wf_sub}, we give the well-formedness and
subtyping relation for the declarative system,
which is identical to the subtyping relation introduced in Subsection~\ref{subsec:dk:decl}.


% In Figure~\ref{fig:decl:wf_sub}, we give the well-formedness and
% subtyping relation for the declarative system. The cases without 
% universal quantifiers are handled by Rules~$\mathtt{{\le}Var}$,
% $\mathtt{{\le}Unit}$ and $\mathtt{{\le}{\to}}$. The subtyping rule 
% for function types ($\mathtt{{\le}{\to}}$) is standard, 
% being contravariant on the argument types.
% Rule~$\mathtt{{\le}\forall R}$ says that if $A$ is a subtype of $B$
% under the context extended with $a$, where $a$ is fresh in $A$, then
% $A\le \forall a. B$. Intuitively, if $A$ is more general than
% the universally quantified type $\forall a. B$, then $A$ must
% instantiate to $[\tau/a]B$ for every $\tau$. 
% %With $A\le B$, where $A$ is
% %instantiated to $B$, which is opened by a fresh variable, we can
% %conclude that for all $\tau$, $A$ have the potential to instantiate to
% %$[\tau/a]B$, therefore $A\le \forall a. B$.

% Finally, the most interesting rule is $\mathtt{{\le}\forall L}$, which instantiates $\forall a. A$ to
% $[\tau/a]A$, and concludes the subtyping $\forall a. A \le B$
% if the instantiation is a subtype of $B$. Notice that $\tau$ is
% \emph{guessed}, 
% and the algorithmic system should provide the means to compute this
% guess. Furthermore, the guess is a \emph{monotype}, which rules 
% out the possibility of polymorphic (or impredicative)
% instantiation. The restriction to monotypes and predicative
% instantiation is used by both Peyton Jones et al.~\citep{jones2007practical} and Dunfield and 
% Krishnaswami's~\citep{dunfield2013complete} algorithms, which are
% adopted by several practical implementations of programming languages.

\begin{comment}
The conclusions of the declarative subtyping rules do not overlap with
each other, except for the judgments with a shape of
$\forall a. A \le \forall a. B$. In this case, an eager application of
Rule~$\mathtt{{\le}\forall R}$ introduces type variable into the context
earlier, which results in an easier problem than the application of
Rule~$\mathtt{{\le}\forall L}$.
\end{comment}

%-------------------------------------------------------------------------------
\subsection{Finding Solutions for Variable Instantiation}

The declarative system specifies the behavior of subtyping relations,
but is not directly implementable: Rule $\mathtt{{\le}\forall L}$
requires guessing a monotype $\tau$.
The core problem that an algorithm for polymorphic 
subtyping needs to solve is to find an algorithmic way to compute the 
monotypes, instead of guessing them. An additional challenge is that
the declarative rule $\mathtt{{\le}{\to}}$ splits one judgment
into two, and the (partial) solutions found for existential variables when
processing the first judgment should be transfered to the second judgment.
%Traditionaly algorithms address this problem using techniques 
%similar to unification. Basically existential (unification) variables
%are introduced through the derivation and solved while traversing the
%types and discovering information. The solutions for those existential
%variables are then propagated across branches of derivation.

%- - - - - - - - - - - - - - - - - - - - - - - - - - - - - - - - - - - - - - - - 
\paragraph{Dunfield and Krishnaswami's Approach}
An elegant algorithmic solution to computing the monotypes is 
presented by \citet{dunfield2013complete}.
Their algorithmic subtyping judgment has the form:

\[\Psi \vdash A \le B \dashv \Phi\]

\noindent 
A notable difference to the declarative judgment is the presence of a
so-called \emph{output context} $\Phi$, which refines the \emph{input
context} $\Psi$ with solutions for existential variables found while
processing the two types being compared for subtyping. 
Both $\Psi$ and $\Phi$ are \emph{ordered contexts} with the
same structure. Ordered contexts are particularly useful to keep track
of the correct scoping for variables, and
are a notable difference to older type-inference
algorithms~\citep{damas1982principal} that use global
unification variables or constraints collected in a set. 

Output contexts
are useful to transfer information across judgments in Dunfield and
Krishnaswami's approach. For example, the algorithmic rule
corresponding to $\mathtt{{\le}{\to}}$ in their approach is:
$$\inferrule*[right=$\mathtt{{<:}{\to}}$]
	{\Psi\vdash B_1 <: A_1 \dashv \Phi\quad \Phi \vdash [\Phi]A_2 <: [\Phi]B_2\dashv \Phi'}
	{\Psi\vdash A_1\to A_2 <: B_1\to B_2 \dashv \Phi'}
$$
\noindent The information gathered by the output context when comparing the
input types of the functions for subtyping is transferred to the second
judgment by becoming the new input context, while any solution derived from the first judgment is applied to the types of the second judgment.

\paragraph{Example} If we want to show that $\forall a. a\to a$ is a subtype
of $1\to 1$, the declarative system will guess the proper $\tau=1$ for
Rule $\mathtt{{\le}\forall L}$: $$\frac{\cdot\vdash 1\quad \cdot\vdash 1
\to 1 \le 1\to 1} {\cdot\vdash \forall a. a\to a \le 1
\to 1}\ \mathtt{{\le}\forall L}$$

\noindent Dunfield and Krishnaswami introduce an \emph{existential variable}---denoted
with $\al,\bt$---whenever a monotype $\tau$ needs to be guessed. Below is a sample
derivation of their algorithm:
% \jimmy{a figure with the algorithmic rules that we use in this and other examples}
$$\inferrule*[right=$\mathtt{{<:}\forall L}$]
{\inferrule*[right=$\mathtt{{<:}{\to}}$]
	{
		\inferrule*[right=$\mathtt{InstRSolve}$]{~}
		{\al\vdash 1\le \al \dashv \al=1}
		\quad
		\inferrule*[right=$\mathtt{{<:}Unit}$]{~}
		{\al=1\vdash 1\le 1\dashv \al=1}
	}
	{\al\vdash \al\to\al \le 1\to 1\dashv \al = 1}
}
{\cdot\vdash \forall a. a\to a \le 1 \to 1 \dashv \cdot}$$

\noindent 
The first step applies Rule $\mathtt{{<:}\forall L}$, which introduces a
fresh existential variable, $\al$, and opens the left-hand-side
$\forall$-quantifier with it. Next, Rule $\mathtt{{<:}{\to}}$
splits the judgment in two. For the first branch, Rule
$\mathtt{InstRSolve}$ satisfies $1\le \al$ by solving $\al$ to
$1$, and stores the solution in its output context. The output context
of the first branch is used as the input context of the second branch,
and the judgment is updated according to current solutions. Finally,
the second branch becomes a base case, and Rule $\mathtt{{<:}Unit}$
finishes the derivation, makes no change to the input context and
propagates the output context back.

Dunfield and Krishnaswami's algorithmic specification is elegant
and contains several useful ideas for a mechanical
formalization of polymorphic subtyping. For example,
\emph{ordered contexts} and \emph{existential variables} enable a purely inductive formulation 
of polymorphic subtyping. However, the binding/scoping structure of their 
algorithmic judgment is still fairly complicated and poses 
challenges when porting their approach to a theorem prover. 

\subsection{The Worklist Approach}
%However, our worklist approach aims at eager
%propagation, which applies any solution or partial
%solution immediately to all the judgments.
We inherit Dunfield and Krishnaswami's ideas of ordered contexts,
existential variables and the idea of solving those variables, but
drop output contexts. Instead, our algorithmic rule has the form:

\[\Gamma \vdash \exps\]

\noindent where $\exps$ is a list of judgments $A \le B$ instead of a
single one. This judgment form, which we call \emph{worklist judgment},
simplifies two aspects of Dunfield and
Krishnaswami's approach.

%\jimmy{Mention the relation to constraint solving, e.g. in unification. Possibly talk about nondeterminism.}

Firstly, as already stated, there are no output
contexts. Secondly, the form of the ordered contexts becomes simpler.
The transfer of information across judgments is simplified because 
all judgments share the input context. Moreover, the order of the
judgments in the list allows information discovered when processing
the earlier judgments to be easily transferred to the later judgments.
In the worklist approach the rule for function types is:
\[\inferrule*[right=$\mathtt{{\le_a}{\to}}$]
  {\Gamma \vdash \jcons{B_1\le A_1}{\jcons{A_2\le B_2}{\exps}}}
  {\Gamma \vdash \jcons{A_1\to A_2\le B_1\to B_2}{\exps}}\]


The derivation of the previous example with the worklist approach is:
\begin{equation*}
\inferrule*[Right=$\mathtt{{\le_a}\forall L}$]
{\inferrule*[Right=$\mathtt{{\le_a}{\to}}$]
	{\inferrule*[Right=$\mathtt{{\le_a}solve\_ex}$]
		{\inferrule*[Right=$\mathtt{{\le_a}unit}$]
			{\inferrule*[Right=$\mathtt{a\_nil}$]
				{~}
				{\cdot \vdash \cdot}
			}
			{\cdot \vdash \jcons{1 \le 1}{\cdot}}
		}
		{\al \vdash \jcons{1\le \al}{\jcons{\al \le 1}{\cdot}}}
	}
	{\al \vdash \jcons{\al \to \al \le 1\to 1}{\cdot}}
}
{\cdot \vdash \jcons{\forall a. a\to a\le 1\to 1}{\cdot}}
\end{equation*}

\begin{comment}
When Rule $\mathtt{{<:}{\to}}$ is applied in their algorithm, we
simply add a judgment to the judgment list and let the algorithm
continue focusing on the first branch. When an existential variable
can be solved to some monotype, we substitute the variable by its
solution to each of the judgments.
\end{comment}

To derive $\cdot\vdash \forall a. a\to a \le 1\to 1$
with the worklist approach, we first introduce an existential variable
and change the judgment to
$\jcons{\al\vdash \al\to\al \le 1\to 1}{\cdot}$. Then, we
split the judgment in two for the function types and the derivation
comes to $\jcons{\al\vdash 1\le \al}{\jcons{\al \le 1}{\cdot}}$. When the first
judgment is solved with $\al = 1$, we immediately remove $\al$
from the context, while propagating the solution as a substitution to
the rest of the judgment list, resulting in $\jcons{\cdot\vdash 1 \le 1}{\cdot}$,
which finishes the derivation in two trivial steps.

With this form of eager propagation, solutions no longer
need to be recorded in contexts, simplifying the encoding and
reasoning in a proof assistant.

\paragraph{Key Results}
Both the declarative and algorithmic systems are formalized in Abella.
We have proven 3 important properties for this algorithm: 
\emph{decidability}, ensuring that the algorithm always terminates; and \emph{soundness} and
\emph{completeness}, showing the equivalence of the declarative and algorithmic systems. 

%----------------------------------------------------------------------------------------
%\subsection{Worklist Algorithm}
% key ideas: replace tree of judgments by a list of judgments...

%\subsection{The Choice of Abella}



\section{A Worklist Algorithm for Polymorphic Subtyping}\label{algorithmic_subtyping}

This section presents our algorithm for polymorphic
subtyping. A novel aspect of our algorithm is the use of worklist
judgments: a form of judgement that facilitates the propagation 
of information. 


%-------------------------------------------------------------------------------
\subsection{Syntax and Well-Formedness of the Algorithmic System}
Figure~\ref{fig:alg:syntax} shows the
syntax and the well-formedness judgement.  

% - - - - - - - - - - - - - - - - - - - - - - - - - - - - - - - - - - - - - - - - 
\paragraph{Existential Variables}
In order to solve the unknown types $\tau$, the algorithmic system extends the
declarative syntax of types with \emph{existential variables} $\al$.  They
behave like unification variables, but are not globally defined. Instead, the
ordered \emph{algorithmic context}, inspired by Dunfield and
Krishnaswami~\cite{dunfield2013complete}, defines their scope. Thus 
the type $\tau$ represented by the corresponding existential variable is
always bound in the corresponding declarative context $\Psi$.

%- - - - - - - - - - - - - - - - - - - - - - - - - - - - - - - - - - - - - - - - 
\paragraph{Worklist Judgements} The form of our algorithmic judgements is
non-standard. 
%tracks the (partial) solutions of existential variables
%in the algorithmic context; they denote a delayed substitution that is
%incrementally applied to outstanding work as it is encoutered.  
%Instead of reifying the substitution, 
Our algorithm keeps track of an explicit list of
outstanding work: the list $\Omega$ of (reified) \emph{algorithmic judgements} 
of the form $A \leq B$,
to which a substitution can be applied once and for all to propagate the solution
of an existential variable. 

\begin{figure}[t]
\[
\begin{array}{l@{\qquad}lcl}
\text{Type variables} & a, b\\
\text{Existential variables} & \al, \bt\\[3mm]
\text{Algorithmic types} &A, B, C &::=&\quad 1 \mid a \mid \al \mid \forall a. A \mid A\to B\\
\text{Algorithmic context}&\Gamma &::=&\quad \cdot \mid \Gamma, a \mid \Gamma, \al\\
\text{Algorithmic judgments}&\exps &::=&\quad \cdot \mid \jcons{A \le B}{\exps}
\end{array}
\]
\centering \framebox{$\Gamma \vdash A$}
\begin{gather*}
\inferrule*[right=$\mathtt{{wf_a}unit}$]
  {~}
  {\Gamma \vdash 1}
\qquad
\inferrule*[right=$\mathtt{{wf_a}var}$]
  {a\in\Gamma}
  {\Gamma\vdash a}
\qquad
\inferrule*[right=$\mathtt{{wf_a}exvar}$]
  {\al\in\Gamma}
  {\Gamma\vdash \al} \\
\inferrule*[right=$\mathtt{{wf_a}{\to}}$]
  {\Gamma\vdash A \\ \Gamma\vdash B}
  {\Gamma\vdash A\to B}
\qquad
\inferrule*[right=$\mathtt{{wf_a}\forall}$]
  {\Gamma, a\vdash A}
  {\Gamma\vdash \forall a. A}
\end{gather*}
\caption{Syntax and Well-Formedness Judgement for the Algorithmic System.}\label{fig:alg:syntax}
\end{figure}

\begin{comment}
\begin{figure}[t]
\centering \framebox{$\Gamma \vdash A$}
\begin{gather*}
\inferrule*[right=$\mathtt{{wf_a}unit}$]
  {~}
  {\Gamma \vdash 1}
\qquad
\inferrule*[right=$\mathtt{{wf_a}var}$]
  {a\in\Gamma}
  {\Gamma\vdash a}
\qquad
\inferrule*[right=$\mathtt{{wf_a}exvar}$]
  {\al\in\Gamma}
  {\Gamma\vdash \al} \\
\inferrule*[right=$\mathtt{{wf_a}{\to}}$]
  {\Gamma\vdash A \\ \Gamma\vdash B}
  {\Gamma\vdash A\to B}
\qquad
\inferrule*[right=$\mathtt{{wf_a}\forall}$]
  {\Gamma, a\vdash A}
  {\Gamma\vdash \forall a. A}
\end{gather*}
\caption{Well-Formedness Judgement of the Algorithmic System}\label{fig:alg:wf}
\end{figure}
\end{comment}

\paragraph{Hole Notation}
To facilitate context manipulation, we use the syntax $\Gamma[\Gamma_M]$ to
denote a context of the form $\Gamma_L, \Gamma_M, \Gamma_R$ where $\Gamma$ is
the context $\Gamma_L, \bullet, \Gamma_R$ with a hole ($\bullet$).
Hole notations with the same name implicitly share the same $\Gamma_L$ and $\Gamma_R$. A multi-hole notation like $\Gamma[\al][\bt]$ means $\Gamma_1,\al,\Gamma_2,\bt,\Gamma_3$.

%-------------------------------------------------------------------------------
\subsection{Algorithmic Subtyping}

The algorithmic subtyping judgement, defined in Figure~\ref{fig:alg}, has the form $\Gamma\vdash\exps$, where
$\exps$ collects multiple subtyping judgments $A\le B$. 
% \bruno{Text comparing to Dunfield. Maybe mention in RW instead?:
% In contrast to the
% original formulation---which features 3 interdepenent judgements---our
% algorithmic rules are all part of the same judgement. This is better for
% formalization in proof assistants and avoids mutual dependencies. 
% }
The algorithm treats $\exps$ as a worklist. In every step
it takes one task from the worklist for processing, possibly
pushes some new tasks on the worklist, and repeats this
process until the list is empty. This last and single base case
is handled by Rule~$\mathtt{a\_nil}$.
The remaining rules all deal with the first task in the worklist.
Logically we can discern 3 groups of rules.

\begin{figure}[t]
\centering \framebox{$\Gamma \vdash \exps$}
\begin{gather*}
\inferrule*[right=$\mathtt{a\_nil}$]
  {~}
  {\Gamma \vdash \cdot} 
\\ \\
\inferrule*[right=$\mathtt{{\le_a}unit}$]
  {\Gamma \vdash \exps}
  {\Gamma \vdash \jcons{1\le 1}{\exps}}
\qquad
\inferrule*[right=$\mathtt{{\le_a}var}$]
  {a\in\Gamma \\ \Gamma \vdash \exps}
  {\Gamma \vdash \jcons{a\le a}{\exps}}
\qquad
\inferrule*[right=$\mathtt{{\le_a}exvar}$]
  {\al\in\Gamma \\ \Gamma \vdash \exps}
  {\Gamma \vdash \jcons{\al\le \al}{\exps}}
\\
\inferrule*[right=$\mathtt{{\le_a}{\to}}$]
  {\Gamma \vdash \jcons{B_1\le A_1}{\jcons{A_2\le B_2}{\exps}}}
  {\Gamma \vdash \jcons{A_1\to A_2\le B_1\to B_2}{\exps}}
\\
\inferrule*[right=$\mathtt{{\le_a}\forall L}$]
  {\al \text{ fresh} \\ \Gamma,\al \vdash \jcons{[\al/a]A\le B}{\exps}}
  {\Gamma\vdash \jcons{\forall a. A\le B}{\exps}}
\qquad
\inferrule*[right=$\mathtt{{\le_a}\forall R}$]
  {b \text{ fresh} \\ \Gamma,b \vdash \jcons{A\le B}{\exps}}
  {\Gamma\vdash \jcons{A\le \forall b. B}{\exps}}
\\
\\
\inferrule*[right=$\mathtt{{\le_a}instL}$]
  {\al\notin \mathit{FV}(A)\cup FV(B)\quad
  	\Gamma[\al[1], \al[2]]\vdash \jcons{\al[1]\to \al[2]\le A\to B}{ [\al[1]\to \al[2]/\al]\exps}}
  {\Gamma[\al] \vdash \jcons{\al\le A\to B}{\exps}}
\\
\inferrule*[right=$\mathtt{{\le_a}instR}$]
  {\al\notin FV(A)\cup FV(B)\quad
	\Gamma[\al[1], \al[2]]\vdash \jcons{A\to B\le \al[1]\to \al[2]}{ [\al[1]\to \al[2]/\al]\exps}}
  {\Gamma[\al] \vdash \jcons{A\to B\le \al}{\exps}}
\\
\\
\inferrule*[right=$\mathtt{{\le_a}solve\_ex}$]
  {\Gamma[\al][]\vdash [\al/\bt]\exps}
  {\Gamma[\al][\bt]\vdash \jcons{\al\le \bt}{\exps}}
\qquad
\inferrule*[right=$\mathtt{{\le_a}solve\_ex'}$]
  {\Gamma[\al][]\vdash [\al/\bt]\exps}
  {\Gamma[\al][\bt]\vdash \jcons{\bt\le \al}{\exps}}
\\
\inferrule*[right=$\mathtt{{\le_a}solve\_var}$]
  {\Gamma[a][]\vdash [a/\bt]\exps}
  {\Gamma[a][\bt]\vdash \jcons{a\le \bt}{\exps}}
\qquad
\inferrule*[right=$\mathtt{{\le_a}solve\_var'}$]
  {\Gamma[a][]\vdash [a/\bt]\exps}
  {\Gamma[a][\bt]\vdash \jcons{\bt\le a}{\exps}}
\\
\inferrule*[right=$\mathtt{{\le_a}solve\_unit}$]
  {\Gamma[]\vdash [1/\al]\exps}
  {\Gamma[\al]\vdash \jcons{\al\le 1}{\exps}}
\qquad
\inferrule*[right=$\mathtt{{\le_a}solve\_unit'}$]
  {\Gamma[]\vdash [1/\al]\exps}
  {\Gamma[\al]\vdash \jcons{1\le \al}{\exps}}
\end{gather*}
\caption{Algorithmic Subtyping}
\label{fig:alg}
\end{figure}

Firstly, we have five rules that are similar to those in the declarative
system, mostly just adapted to the worklist style. For instance, Rule
$\mathtt{{\le_a}{\to}}$ consumes one judgment and pushes two to the
worklist.  A notable difference with the declarative Rule $\mathtt{{\le}\forall
L}$ is that Rule $\mathtt{{\le_a}\forall L}$ requires no guessing of a type $\tau$ to instantiate
the polymorphic type $\forall a. A$, but instead
introduces an existential variable $\al$ to the context and to $A$. In
accordance with the declarative system, where 
the monotype $\tau$ should be bound in the context $\Psi$, here $\al$ should only
be solved to a monotype bound in $\Gamma$. More generally, for any algorithmic context $\Gamma[\al]$, the algorithmic variable $\al$ 
can only be solved to a monotype that is well-formed with respect to $\Gamma_L$.

Secondly, Rules $\mathtt{{\le_a}instL}$ and $\mathtt{{\le_a}instR}$ partially
instantiate existential types $\al$, to function types. The domain and range
of the new function type are undetermined: they are set to two
fresh existential variables $\al[1]$ and $\al[2]$. To make sure that
$\al[1] \to \al[2]$ has the same scope as $\al$, the new variables
$\al[1]$ and $\al[2]$ are inserted in the same position in the context
where the old variable $\al$ was. To propagate the instantiation to the remainder
of the worklist, $\al$ is substituted for $\al[1] \to \al[2]$ in $\Omega$.
The \emph{occurs-check} side-condition is necessary to prevent a diverging
infinite instantiation. For example
$1 \to \al \le \al$ would diverge with no such check.

Thirdly, in the remaining six rules an existential variable can be immediately
solved. Each of the six similar rules removes an existential variable from the
context,  performs a substitution on the remainder of the worklist and
continues.

The algorithm on judgment list is designed to share the context across all judgments.
However, the declarative system does not share a single context in its derivation.
This gap is filled by strengthening and weakening lemmas of both systems,
where most of them are straightforward to prove,
except for the strengthening lemma of the declarative system, which is a little trickier.

\begin{figure}[t]
\begin{tabular}{cc}
$\begin{aligned}
	\inferrule*[Right=$\mathtt{{\le_a}\forall L}$]
	{\inferrule*[Right=$\mathtt{{\le_a}{\to}}$]
		{\inferrule*[Right=$\mathtt{{\le_a}\forall L}$]
			{\inferrule*[Right=$\mathtt{{\le_a}instR}$]
				{\inferrule*[Right=$\mathtt{{\le_a}{\to}}$]
					{\inferrule*[Right=$\mathtt{{\le_a}solve\_ex}$]
						{\inferrule*[Right=$\mathtt{{\le_a}solve\_ex}$]
							{\inferrule*[Right=$\mathtt{{\le_a}unit}$]
								{\inferrule*[Right=$\mathtt{a\_nil}$]
									{~}
									{\al[1] \vdash \cdot}
								}
								{\al[1] \vdash \jcons{1 \le 1}{\cdot}}
							}
							{\al[1], \al[2] \vdash \jcons{\al[1] \le \al[2]}{\jcons{1 \le 1}{\cdot}}}
						}
						{\al[1], \al[2], \bt \vdash \jcons{\al[1] \le \bt}{\jcons{\bt \le \al[2]}{\jcons{1 \le 1}{\cdot}}}}
					}
					{\al[1], \al[2], \bt \vdash \jcons{\bt \to \bt \le \al[1] \to \al[2]}{\jcons{1 \le 1}{\cdot}}}
				}
				{\al, \bt \vdash \jcons{\bt \to \bt \le \al}{\jcons{1 \le 1}{\cdot}}}
			}
			{\al \vdash \jcons{\forall a.\ a \to a \le \al}{\jcons{1 \le 1}{\cdot}}}
		}
		{\al \vdash \jcons{\al \to 1 \le (\forall a.\ a \to a) \to 1}{\cdot}}
	}
	{\cdot \vdash \jcons{\forall a.\ a\to 1\le (\forall a.\ a\to a)\to 1}{\cdot }}
	\end{aligned}\qquad\quad$
	%\label{fig:alg_sample}
&
	$\begin{aligned}
	\inferrule*[Right=$\mathtt{{\le_a}\forall L}$]
	{\inferrule*[Right=$\mathtt{{\le_a}{\to}}$]
		{\inferrule*[Right=$\mathtt{{\le_a}unit}$]
			{\inferrule*[Right=$\mathtt{{\le_a}\forall R}$]
				{\inferrule*[Right=$\mathtt{?}$]
					{stuck
					}
					{\al, b \vdash \jcons{\al \le b}{\cdot}}
				}
				{\al \vdash {\jcons{\al \le \forall b.\ b}{\cdot}}}
			}
			{\al \vdash \jcons{1\le 1}{\jcons{\al \le \forall b.\ b}{\cdot}}}
		}
		{\al \vdash \jcons{1 \to \al \le 1\to \forall b.\ b}{\cdot}}
	}
	{\cdot\vdash \jcons{\forall a.\ 1\to a \le 1\to \forall b.\ b}{\cdot }}
	\end{aligned}$
	%\label{fig:alg_sample2}
\end{tabular}
\caption{Successful and Failing Derivations for the Algorithmic Subtyping Relation}
\label{fig:alg_samples}
\end{figure}

\paragraph{Example}
We illustrate the subtyping rules through a sample derivation in the left of
Figure~\ref{fig:alg_samples}, which shows that that $\forall a.\ a\to 1\le (\forall a.\ a\to
a)\to 1$. Thus the derivation starts with an empty context and a
judgment list with only one element.


% \begin{center}
% 	\begin{tabular}{|c|c|c|c|}\hline
% 		\# & Context & Worklist & Rule\\\hline
% 		1&$\cdot$ & $\forall x.\ x\to 1\le (\forall x.\ x\to x)\to 1$ & $\mathtt{{\le_a}\forall L}$\\\hline
% 		2&$\al$ & $\al\to 1\le (\forall x.\ x\to x)\to 1$ & $\mathtt{{\le_a}{\to}}$\\\hline
% 		3&$\al$ & $\forall x.\ x\to x\le \al : 1\le 1$ & $\mathtt{{\le_a}\forall L}$\\\hline
% 		4&$\al,\bt$ & $\bt\to \bt\le \al : 1\le 1$ & $\mathtt{instR}$\\\hline
% 		5&$\al[1],\al[2], \bt$ & $\bt\to \bt\le \al[1]\to \al[2] : 1\le 1$ & $\mathtt{{\le_a}{\to}}$\\\hline
% 		6&$\al[1],\al[2], \bt$ & $\al[1]\le \bt : \bt\le \al[2] : 1\le 1$ & $\mathtt{{\le_a}solve\_ex (\bt\leftarrow \al[1])}$\\\hline
% 		7&$\al[1],\al[2]$ & $\al[1]\le \al[2] : 1\le 1$ & $\mathtt{{\le_a}solve\_ex (\al[2]\leftarrow \al[1])}$\\\hline
% 		8&$\al[1]$ & $1\le 1$ & $\mathtt{{\le_a}unit}$\\\hline
% 	\end{tabular}
% \end{center}

In step 1, we have only one judgment, and that one has a top-level $\forall$ on
the left hand side. So the only choice is rule $\mathtt{{\le_a}\forall L}$, which
opens the universally quantified type with an unknown existential variable
$\al$. Variable $\al$ will be solved later to some monotype that is well-formed
within the context before $\al$. That is, the empty context $\cdot$ in this
case.
In step 2, rule $\mathtt{{\le_a}{\to}}$ is applied to the worklist,
splitting the first judgment into two.
Step 3 is similar to step 1, where the left-hand-side $\forall$ of the first
judgment is opened according to rule $\mathtt{{\le_a}\forall L}$ with a fresh
existential variable.
In step 4, the first judgment has an arrow on the left hand side, but the
right-hand-side type is an existential variable. It is obvious
that $\al$ should be solved to a monotype of the form
$\sigma \to \tau$. Rule $\mathtt{instR}$ implements this, but avoids
guessing $\sigma$ and $\tau$ by ``splitting'' $\al$ into two existential
variables, $\al[1]$ and $\al[2]$, which will be solved to some $\sigma$ and
$\tau$ later.
Step 5 applies Rule $\mathtt{{\le_a}{\to}}$ again. Notice that after the
split, $\bt$ appears in two judgments. When the first $\bt$ is solved
during any step of derivation, the next $\bt$ will be substituted by that
solution.  This propagation mechanism ensures the consistent solution of the
variables, while keeping the context as simple as possible.
Steps 6 and 7 solve existential variables. The existential
variable that is right-most in the context is always solved in terms of the other. Therefore in step 6,
$\bt$ is solved in terms of $\al[1]$, and in step 7, $\al[2]$ is solved in terms of $\al[1]$.
Additionally, in step 6, when $\bt$ is solved, the substitution $[\al[1] /
\bt]$ is propagated to the rest of the judgment list, and thus the second
judgment becomes $\al[1]\le\al[2]$.
Steps 8 and 9 trivially finish the derivation. Notice that $\al[1]$ is not
instantiated at the end. This means that any well-scoped instantiation is fine.

\paragraph{A Failing Derivation} We illustrate the role of ordered contexts through another example: $\forall a.\ 1\to a \le 1\to \forall b.\ b$. From the declarative perspective, $a$ should be instantiated to some $\tau$ first, then $b$ is introduced to the context, so that $b\notin FV(\tau)$. As a result, we cannot find $\tau$ such that $\tau \le b$. The right of Figure~\ref{fig:alg_samples} shows the algorithmic derivation, which also fails due to the scoping---$\al$ is introduced earlier than $b$, thus it cannot be solved to $b$.






\section{Metatheory}\label{metatheory}

This section presents the 3 main meta-theoretical results that we have proved
in Abella. The first two are soundness and 
completeness of our algorithm with respect to Odersky and L\"aufer's declarative
subtyping. The third result is our algorithm's decidability. 

% The formalization of our higher-order polymorphism is developed in 
% Abella. Abella offers great flexibility on variable bindings and
% variable substitutions, which significantly decreases the difficulty
% of handling freshness and substitutions on the judgment list.

%-------------------------------------------------------------------------------
\subsection{Transfer to the Declarative System}

To state the correctness of the algorithmic subtyping rules,
Figure~\ref{fig:transfer} 
introduces two \textit{transfer} judgements to
relate the declarative and the algorithmic system. 
The first judgement, transfer of contexts $\Gamma \to \Psi$, removes
existential variables from the algorithmic context $\Gamma$ to obtain a
declarative context $\Psi$.
The second judgement, transfer of the judgement list $\Gamma \mid \exps \rightsquigarrow \exps'$,
replaces all occurrences of existential variables in $\exps$  by well-scoped mono-types.
Notice that this judgment is not decidable, i.e. a pair of
$\Gamma$ and $\exps$ may be related with multiple $\exps'$. However, if there
exists some substitution that transforms $\exps$ to $\exps'$, and each
subtyping judgment in $\exps'$ holds, we know that $\exps$ is potentially
satisfiable.

\begin{figure}[t]
\centering \framebox{$\Gamma \to \Psi$}
$$
\inferrule*[right=$\mathtt{{\to}\cdot}$]
	{~}{\cdot\to \cdot}
\qquad
\inferrule*[right=$\mathtt{{\to}var}$]
	{\Gamma\to \Psi}
	{\Gamma, a\to \Psi, a}
\qquad
\inferrule*[right=$\mathtt{{\to}exvar}$]
	{\Gamma\to \Psi}
	{\Gamma, \alpha\to \Psi}
\qquad
$$
\framebox{$\Gamma \mid \exps \rightsquigarrow \exps'$}
$$
\inferrule*[right=$\mathtt{{\rightsquigarrow}\cdot}$]
	{~}{\cdot \mid \exps\rightsquigarrow \exps}
\qquad
\inferrule*[right=$\mathtt{{\rightsquigarrow}var}$]
	{\Gamma \mid \exps \rightsquigarrow \exps'}
	{\Gamma, a \mid \exps \rightsquigarrow \exps'}
\qquad
\inferrule*[right=$\mathtt{{\rightsquigarrow}exvar}$]
	{\Gamma\to \Psi \quad \Psi\vdash \tau\quad
		\Gamma \mid [\tau/\alpha]\exps \rightsquigarrow \exps'}
	{\Gamma, \alpha \mid \exps \rightsquigarrow \exps'}
$$
\caption{Transfer Rules}
\label{fig:transfer}
\end{figure}

The following two lemmas generalize Rule~$\mathtt{{\rightsquigarrow}exvar}$ from substituting the first
existential variable to substituting any existential variable.

\begin{lemma}[Insert]
If $\Gamma\to\Psi$ and $\Psi\vdash \tau$ and $\Gamma, \Gamma_1 \mid [\tau/\alpha]\exps \rightsquigarrow \exps'$
, then $\Gamma, \alpha, \Gamma_1\mid \exps \rightsquigarrow \exps'$.
% \[
% \inferrule
%   {\Gamma\to\Psi \\ \Psi\vdash \tau \\ \Gamma, \Gamma_1 \mid [\tau/\alpha]\exps \rightsquigarrow \exps'}
%   {\Gamma, \alpha, \Gamma_1\mid \exps \rightsquigarrow \exps'}
% \]
\end{lemma}
\begin{lemma}[Extract]
	If $\Gamma, \alpha, \Gamma_1\mid \exps \rightsquigarrow \exps'$, then $\exists\tau$ s.t. $\Gamma\to\Psi, \Psi\vdash \tau$ and $\Gamma, \Gamma_1 \mid [\tau/\alpha]\exps \rightsquigarrow \exps'$.
\end{lemma}

In order to match the shape of algorithmic subtyping relation for the following proofs, we define a relation $\Psi\vdash \exps$ for the declarative system, meaning that all the declarative judgments hold under context $\Psi$.
\begin{definition}[Declarative Subtyping Worklist]
	$$\Psi\vdash\exps := \forall (A\le B)\in \exps, \Psi\vdash A\le B$$
\end{definition}

%The transfer rules play a role of an axiom which specifies the relation between our algorithmic system and declarative system.

%-------------------------------------------------------------------------------
\subsection{Soundness}
Our algorithm is sound with respect to the
declarative specification. For any derivation of a list of algorithmic
judgments $\Gamma\vdash\exps$, we can find a valid transfer
$\Gamma\mid\exps\rightsquigarrow\exps'$ such that all judgments in $\exps'$ hold
in $\Psi$, with $\Gamma \to \Psi$.

\begin{theorem}[Soundness]
	If $\Gamma\vdash \exps$ and $\Gamma \to \Psi$, then there exists $\exps'$, s.t. $\Gamma \mid \exps\rightsquigarrow \exps'$ and $\Psi\vdash \exps'$.
\end{theorem}
The proof proceeds by induction on the derivation
of $\Gamma\vdash\exps$, finished off by appropriate applications of the insertion and
extraction lemmas.

%\tom{Say something about the corollary, e.g., why it's important.}
%\begin{corollary}[Soundness\_decl]
%	If $\Gamma = \Psi, A = A', B = B'$, $\Gamma\vdash A\le B : \cdot$, then $\Psi\vdash A'\le B'$.
%\end{corollary}

%-------------------------------------------------------------------------------
\subsection{Completeness}

Completeness of the algorithm means that any declarative derivation has an
algorithmic counterpart. 
% More specifically, for any derivation of declarative
% judgments $\Psi\vdash\exps'$ and any extended environment $\Gamma$, any
% algorithmic judgments that satisfies the transfer relation
% $\Gamma\mid\exps\rightsquigarrow\exps'$, we can successfully derive
% $\Gamma\vdash\exps$.
\begin{theorem}[Completeness]
	If $\Psi\vdash \exps'$ and $\Gamma \to \Psi$ and $\Gamma \mid \exps\rightsquigarrow \exps'$, then $\Gamma\vdash \exps$.
\end{theorem}

The proof proceeds by induction on the derivation of $\Psi\vdash\exps'$. As the
declarative system does not involve information propagation across judgments,
the induction can focus on the subtyping derivation of the first judgment
without affecting other judgments.
The difficult cases correspond to the $\mathtt{{\le_a}instL}$ and
$\mathtt{{\le_a}instR}$ rules.  When the proof by induction on $\Psi\vdash
\exps'$ reaches the $\mathtt{{\le}{\to}}$ case, the first declarative
judgment has a shape like $A_1\to A_2 \le B_1\to B_2$. 
One of the possibile cases for the first corresponding algorithmic judgement
is $\alpha\le A\to B$. However, the case analysis does not indicate
that $\alpha$ is fresh in $A$ and $B$. Thus we cannot apply
Rule~$\mathtt{{\le_a}instL}$ and make use of the induction hypothesis.
The following lemma helps us out in those cases:
it rules out subtypings with infinite types as solutions (e.g. $\alpha \le 1
\to \alpha$) and guarantees that $\alpha$ is free in $A$ and $B$.
\begin{lemma}[Prune Transfer for Instantiation]
	If $\Psi \vdash \jcons{A_1\to A_2 \le B_1\to B_2}{\exps'}$ and $\Gamma\to \Psi$ and $\Gamma\mid (\jcons{\alpha \le A\to B}{\exps}) \rightsquigarrow
	(\jcons{A_1\to A_2 \le B_1\to B_2}{\exps'})$
	% or $\Gamma\mid A\to B \le \alpha : \exps \rightsquigarrow A_1\to A_2 \le B_1\to B_2 : \exps'$
	, then $\alpha \notin FV(A)\cup FV(B)$. 
\end{lemma}
A similar lemma holds for the symmetric case $(\jcons{A \to B \le \alpha}{\Omega})$.
%\tom{Say something about the corollary, e.g., why it's important.}
%\begin{corollary}[Completeness\_decl]
%	If $\Gamma = \Psi, A = A', B = B'$, $\Psi\vdash A'\le B'$, then $\Gamma\vdash A\le B$.
%\end{corollary}

%-------------------------------------------------------------------------------
\subsection{Decidability}

The third key result for our algorithm is decidability.
\begin{theorem}[Decidability]
	Given any well-formed judgment list $\exps$ under $\Gamma$, it is decidable whether $\Gamma\vdash \exps$ or not.
\end{theorem}

\noindent We have proven this theorem by means of a lexicographic group of induction
measurements $\left\langle|\Omega|_\forall,|\Gamma|_\alpha, |\Omega|_\to\right\rangle$ on the worklist $\Omega$
and algorithmic context $\Gamma$. 
The worklist measures $|\cdot|_\forall$ and $|\cdot|_\to$ count the
number of universal quantifiers and function types respectively.
\begin{definition}[Worklist Measures]
\[
\begin{array}{rcl@{\hspace{-10mm}}rclrcl} 
|1|_\forall = |a|_\forall = |\alpha|_\forall &=& 0      &|1|_\to = |a|_\to = |\alpha|_\to &=& 0        \\
|A\to B|_\forall &=& |A|_\forall + |B|_\forall  &|A\to B|_\to &=& |A|_\to + |B|_\to + 1\\
|\forall x. A|_\forall &=& |A|_\forall + 1              &|\forall x. A|_\to &=& |A|_\to                        \\
|\exps|_\forall &=& \sum_{A \le B\in \exps} |A|_\forall + |B|_\forall       &|\exps|_\to &=& \sum_{A \le B \in \exps} |A|_\to + |B|_\to
\end{array}
\]
\end{definition}

\noindent The context measure $|\cdot|_\alpha$ counts the number of unsolved existential variables.
\begin{definition}[Context Measure]
$$|\cdot|_\alpha = 0\qquad
|\Gamma,a|_\alpha = |\Gamma|_\alpha\qquad
|\Gamma,\alpha|_\alpha = |\Gamma|_\alpha + 1$$
\end{definition}

It is not difficult to see that all but two of the algorithm's rules decrease
one of the three measures. The two exceptions are the Rules $\mathtt{{\le_a}instL}$ and $\mathtt{{\le_a}instR}$; both increment the number of existential
variables and the number of function types without affecting the number of
universal quantifiers.
To handle these rules, we handle a special class of judgements, which
we call \emph{instantiation judgements} $\exps_i$, separately. They 
take the form:
\begin{definition}[$\exps_i$]
$$\exps_i := \cdot \mid \jcons{\alpha\le A}{\exps_i'} \mid \jcons{A\le \alpha}{\exps_i'}
\quad\text{where } \alpha\notin FV(A)\cup FV(\exps_i')$$
\end{definition}
These instantiation judgements are these ones consumed and
produced by the Rules $\mathtt{{\le_a}instL}$ and $\mathtt{{\le_a}instR}$.
The following lemma handles their decidability.
\begin{lemma}[Instantiation Decidability]
	For any context $\Gamma$ and judgment list $\exps_i, \exps$, it is decidable whether $\Gamma\vdash \exps_i,\exps$ if both of the conditions hold
\begin{enumerate}[1)]
	\item $\forall \Gamma', \exps'$ s.t. $|\exps'|_\forall < |\exps_i,\exps|_\forall$, it is decidable whether $\Gamma'\vdash \exps'$.
	\item $\forall \Gamma', \exps'$ s.t. $|\exps'|_\forall = |\exps_i,\exps|_\forall$ and $|\Gamma'|_\alpha = |\Gamma|_\alpha - |\exps_i|$, it is decidable whether $\Gamma'\vdash \exps'$.
\end{enumerate}
\label{lemma:inst_decidable}
\end{lemma}
In other words, for any instantiation judgment prefix $\exps_i$, the algorithm
either reduces the number of $\forall$'s or solves one existential variable per
instantiation judgment. The proof of this lemma is by induction on the measure
$2*|\exps_i|_{\to} + |\exps_i|$ of the instantiation judgment list.

In summary, the decidability theorem can be shown through a lexicographic group
of induction measurements $\left\langle|\Omega|_\forall, |\Omega|_\alpha,
|\Omega|_\to\right\rangle$. The critical case is that, whenever we
encounter an instantiation judgment at the front of the worklist, we refer to
Lemma~\ref{lemma:inst_decidable}, which reduces the number of unsolved
variables by consuming that instantiation judgment, or reduces the number of
$\forall$-quantifiers. Other cases are relatively straightforward.


\section{The Choice of Abella}

We have chosen the Abella (v2.0.5) proof assistant~\cite{AbellaDesc} to
develop our formalization.
Our development is only based on the reasoning logic of Abella, and does not make use of its specification logic.
%Although Abella takes a two-level logic approach,
%where the specification logic can be expressed separately from the
%reasoning logic, we only make use of its reasoning logic, due to the
%difficulty of expressing our algorithmic rules with only the
%specification.
Abella is particularly helpful due to its built-in support for variable bindings, and
its $\lambda$-tree syntax~\cite{miller2000abstract} is a form of HOAS,
which helps with the encoding and reasoning about substitutions.  For
instance, the type $\forall x. x \to a$ is encoded as \abellae{all (x\  arrow x a)}, where \abellae{x\ arrow x a} is a lambda abstraction in
Abella. An opening $[b/x](x\to a)$ is encoded as an application
\abellae{(x\ arrow x a) b}, which can be simplified(evaluated) to
\abellae{arrow b a}.
Name supply and freshness conditions are controlled by the
$\nabla$-quantifier.  The expression \abellae{nabla x, F} means that
\abellae{x} is a unique variable in \abellae{F}, i.e. it is different
from any other names occurring elsewhere.  Such variables are called
nominal constants.  They can be of any type, in other words, every
type may contain unlimited number of such atomic nominal constants.

%\subsection{Encoding of Declarative System}

\paragraph{Encoding of the Declarative System}
As a concrete example, our declarative context and well-formedness rules are encoded as follows.
\begin{abella}
	Kind ty     type.
	Type i      ty.                % the unit type
	Type all    (ty -> ty) -> ty.    % forall-quantifier
	Type arrow  ty -> ty -> ty.      % function type
	Type bound  ty -> o.            % variable collection in contexts
	
	Define env : olist -> prop by
		env nil;
		nabla x, env (bound x :: E) := env E.
	
	Define wft : olist -> ty -> prop by
		wft E i;
		nabla x, wft (E x) x := nabla x, member (bound x) (E x);
		wft E (arrow A B) := wft E A /\ wft E B;
		wft E (all A) := nabla x, wft (bound x :: E) (A x).
\end{abella}

We use the type \abellae{olist} just as normal list of \abellae{o} with two constructors, namely \abellae{nil : olist} and \abellae{(::) : o -> olist -> olist}, where \abellae{o} purely means ``the element type of \abellae{olist}''. The \abellae{member : o -> olist -> prop} relation is also pre-defined.
The second case of the relation \abellae{wft} states rule $\mathtt{wf_d var}$.
The encoding \abellae{(E x)} basically means that the context \emph{may} contain \abellae{x}.
If we write \abellae{(E x)} as \abellae{E}, then the context should not contain \abellae{x}, and both \abellae{wft E x} and \abellae{member (bound x) E} make no sense.
Instead, we treat \abellae{E : ty -> olist} as an \emph{abstract structure} of a context, such as \abellae{x\ bound x :: bound a :: nil}.
For the fourth case of the relation \abellae{wft}, the type $\forall x. A$ in our target language is expressed as \abellae{(all A)}, and its opening $A$, \abellae{(A x)}.

\paragraph{Encoding of the Algorithmic System}
In terms of the algorithmic system, notably, Abella handles the
$\mathtt{{\le_a}instL}$ and $\mathtt{{\le_a}instR}$ rules in a nice way:
\begin{abella}
	% sub_alg_list : enva -> [subty_judgment] -> prop
	Define subal : olist -> olist -> prop by
		subal E nil;
		subal E (subt i i :: Exp) := subal E Exp;
		% some cases omitted ...
		% <: instL
		nabla x, subal (E x) (subt x (arrow A B) :: Exp x) :=
			exists E1 E2 F, nabla x y z, append E1 (exvar x :: E2) (E x) /\
				append E1 (exvar y :: exvar z :: E2) (F y z) /\
				subal (F y z) (subt (arrow y z) (arrow A B) :: Exp (arrow y z));
		% <: instR is symmetric to <: instL, omitted here
		% other cases omitted ...
\end{abella}
Thanks to the way Abella deals with nominal constants, the pattern 
\abellae{subt x (arrow A B)} implicitly states that $x\notin FV(A) \land x\notin FV(B)$.
If the condition were not required, we would have encoded the pattern as 
\abellae{subt x (arrow (A x) (B x))} instead.

\subsection{Statistics and Discussion}\label{subsection:discussion}
\begin{figure}[t]
	\centering\begin{tabular}{|c|c|c|l|}\hline
		File(s) & SLOC & \# of Theorems & Description\\\hline
		olist.thm, nat.thm      &  303 & 55  & Basic data structures\\\hline
		higher.thm, order.thm   &  164 & 15  & Declarative system\\\hline
		higher\_alg.thm         &  618 & 44  & Algorithmic system\\\hline
		trans.thm               &  411 & 46  & Transfer\\\hline
		sound.thm               &  166 & 2   & Soundness theorem\\\hline
		depth.thm               &  143 & 12  & Definition of depth\\\hline
		complete.thm            &  626 & 28  & Lemmas and Completeness theorem\\\hline
		decidable.thm           & 1077 & 53  & Lemmas and Decidability theorem\\\hline
		Total                   & 3627 & 267 & (33 definitions in total)\\\hline
	\end{tabular}
	\caption{Statistics for the proof scripts}\label{fig:SLOC}
\end{figure}

Some basic statistics on our proof script are shown in Figure~\ref{fig:SLOC}.
The proof consists of 3627 lines of code with a total of 33 definitions and 267 theorems.
We have to mention that Abella provides few built-in tactics and does not support user-defined ones, and we would reduce significant lines of code if Abella provided more handy tactics.
Moreover, the definition of natural numbers, the plus operation and less-than relation are defined within our proof due to Abella's lack of packages.
However, the way Abella deals with name bindings
is very helpful for type system formalizations and substitution-intensive formalizations, such as this one.


%Dunfield and Krishnaswami calculus tracks the (partial) solutions of existential variables
%in the algorithmic context; they denote a delayed substitution that is
%incrementally applied to outstanding work as it is encoutered.  
%Instead of reifying the substitution, our algorithm keeps track of an explicit list of
%outstanding work.


