%%%%%%%%%%%%%%%%%%%%%%%%%%%%%%%%%%%%%%%%%%%%%%%%%%%%%%%%%%%%%%%%%%%%%%%% 
\chapter{Background}
\label{chap:Background}
%%%%%%%%%%%%%%%%%%%%%%%%%%%%%%%%%%%%%%%%%%%%%%%%%%%%%%%%%%%%%%%%%%%%%%%% 


\section{Hindley-Milner Type System}

\subsection{Declarative System}

\begin{figure}[t]
    \begin{gather*}
    \begin{aligned}
        \text{Type variables}\qquad&a, b\\
        \text{Types}\qquad&\sigma &::=&\quad \tau \mid \all \sigma\\
        \text{Monotypes}\qquad&\tau &::=&\quad 1 \mid a \mid \tau_1\to \tau_2\\
        \text{Expressions}\qquad&e &::=&\quad x \mid () \mid \lam e \mid e_1~e_2 \mid \letin{e_1}{e_2}\\
        \text{Contexts}\qquad&\Psi &::=&\quad \cdot \mid \Psi, x:\sigma
    \end{aligned}
    \end{gather*}
\Description{HM Syntax}
\caption{HM Syntax}\label{fig:hm_decl_syntax}
\end{figure}

\paragraph{Syntax}
The declarative syntax is shown in Figure~\ref{fig:hm_decl_syntax}.
The HM types are consist of polymorphic types (or type schemes) and monomorphic types.
A polymorphic type contains zero or more universal quantifiers only at the top level.
When no universal quantifier occurs, the type belongs to a mono-type.
Mono-types are constructed by a unit type $1$, a type variable $a$,
or a function type $\tau_1 \to \tau_2$.

Expressions $e$ includes variables $x$, literals $()$, lambda abstractions $\lam e$,
applications $e_1~e_2$ and the let expression $\letin{e_1}{e_2}$.
A context $\Psi$ is a collection of type bindings for variables.

\begin{figure}[t]
    \framebox{$\sigma_1 \sqsubseteq \sigma_2$} HM Type Instantiation
    $$
    \inferrule*[right=$\mathtt{HM \dash TInst}$]
        {\tau' = [\overline{\tau}/\overline{a}] \tau \\ \overline{b} \notin \text{FV}(\all[\overline{a}] \tau)}
        {\all[\overline{a}] \tau \sqsubseteq \all[\overline{b}] \tau'}$$
    \framebox{$\Psi \vdash_{HM} e : \sigma$} HM Typing
    \begin{gather*}
        \inferrule*[right=$\mathtt{HM \dash Var}$]
            {(x : \sigma) \in \Psi}
            {\Psi \vdash_{HM} x : \sigma}
        \qquad
        \inferrule*[right=$\mathtt{HM \dash Unit}$]
            {~}{\Psi \vdash_{HM} () : 1}
        \qquad
        \inferrule*[right=$\mathtt{HM \dash Abs}$]
            {\Psi, x:\tau_1 \vdash_{HM} e:\tau_2}
            {\Psi \vdash_{HM} \lam e : \tau_1 \to \tau_2}
        \\
        \inferrule*[right=$\mathtt{HM \dash App}$]
            {\Psi \vdash_{HM} e_1 : \tau_1 \to \tau_2 \\ \Psi \vdash_{HM} e_2 : \tau_1}
            {\Psi \vdash_{HM} e_1~e_2 : \tau_2}
        \\
        \inferrule*[right=$\mathtt{HM \dash Let}$]
            {\Psi \vdash_{HM} e_1:\sigma \\ \Psi, x:\sigma \vdash_{HM} e_2:\tau}
            {\Psi \vdash_{HM} \letin{e_1}{e_2} : \tau}
        \\
        \inferrule*[right=$\mathtt{HM \dash Gen}$]
            {\Psi \vdash_{HM} e:\sigma \\ \overline{a} \notin \text{FV}(\Psi)}
            {\Psi \vdash_{HM} e : \all[\overline{a}] \sigma}
        \qquad
        \inferrule*[right=$\mathtt{HM \dash Inst}$]
            {\Psi \vdash_{HM} e:\sigma_1 \\ \sigma_1 \sqsubseteq \sigma_2}
            {\Psi \vdash_{HM} e : \sigma_2}
    \end{gather*}
\Description{HM Type System}
\caption{HM Type System}\label{fig:hm_decl_type}
\end{figure}

\paragraph{Type Instantiation}
The relations between types are described via type instantiations.
The rule shown to the top of Figure~\ref{fig:hm_decl_type} checks if
$\all[\overline{a}] \tau$ is a \emph{generic instance} of $\all[\overline{b}] \tau'$.
This relation is valid when $\tau' = [\overline{\tau}/\overline{a}] \tau$ for a
series of mono-types $\overline{\tau}$ and each variable in $\overline{b}$ is not
free in $\all[\overline{a}] \tau$.

For example, $$\all a \to a \sqsubseteq 1 \to 1$$
is obtained by the substitution $[1/a]$, and
$$\all a \to a \sqsubseteq \all[b] (b \to b) \to (b \to b)$$
substitutes $a$ by $b \to b$, and generalizes $b$ after the substitution.


\paragraph{Typing}
The typing relation $\Psi \vdash_{HM} e:\sigma$ synthesizes a type $\sigma$ for
an expression $e$ under the context $\Psi$.
Rule $\mathtt{HM \dash Var}$ looks up the binding of a variable $x$ in the context.
Rule $\mathtt{HM \dash Unit}$ always give the unit type $1$ to the unit expression $()$.
For a lambda abstraction $\lam e$, rule $\mathtt{HM \dash Abs}$ guesses its input type ($\tau_1$)
and compute the type of its body ($\tau_2$) as the return type.
Rule $\mathtt{HM \dash App}$ eliminates a function type by an application $e_1~e_2$,
where the argument type must be the same as the input type of the function,
and the type of the whole application is $\tau_2$.

Rule $\mathtt{HM \dash Let}$ is also referred as let-polymorphism.
In (untyped) lambda calculus, $\letin{e_1}{e_2}$ behaves the same as $(\lam e_2)~e_1$.
However, the HM let rule derives the type of $e_1$ first,
and binds the polymorphic type into the context before $e_2$.
This enables polymorphic expressions to be reused multiple times in different instantiated types.

Rules $\mathtt{HM \dash Gen}$ and $\mathtt{HM \dash Inst}$ changes the type of an expression
at any time during the derivation.
Rule $\mathtt{HM \dash Gen}$ generalizes over fresh type variables $\overline{a}$.
Rule $\mathtt{HM \dash Inst}$, as opposed to generalization, specializes a type
according to the type instantiation relation.

The type system of HM supports \emph{implicit instantiation} through
rule $\mathtt{HM \dash Inst}$.
This means that any expression (function) that has a polymorphic type
can be automatically instantiated with a proper monotype for any reasonable application.
The fact that only monotypes are guessed indicates that the system is \emph{predicative}.
In contrast, an \emph{impredicative} system might guess polymorphic types.
Unfortunately, type inference on impredicative systems is undecidable~\cite{wells1999typability}.
In this thesis, we focus on predicative systems only.

\subsection{Algorithmic System and Principality}

\paragraph{Syntax-Directed System}
The declarative system is not syntax-directed due to
rules $\mathtt{HM \dash Gen}$ and $\mathtt{HM \dash Inst}$,
which can be applied on any expression.
A syntax-directed system can be obtained by
replacing rules $\mathtt{HM \dash Var}$ and $\mathtt{HM \dash Let}$
by the following rules:
\begin{gather*}
    \inferrule*[right=$\mathtt{HM \dash Var \dash Inst}$]
        {(x : \sigma) \in \Psi \\ \sigma \sqsubseteq \tau}
        {\Psi \vdash_{HM}^S x : \tau}
    \qquad
    \inferrule*[right=$\mathtt{HM \dash Let \dash Gen}$]
        {
            \Psi \vdash_{HM} e_1:\sigma \\\\
            \overline{a} = \text{FV}(\sigma) - \text{FV}(\Psi)\\\\
            \Psi, x:\all[\overline{a}] \sigma \vdash_{HM} e_2:\tau
        }
        {\Psi \vdash_{HM}^S \letin{e_1}{e_2} : \tau}
\end{gather*}

A generalization on $\sigma$, the synthesized type of $e_1$,
is added to rule $\mathtt{HM \dash Let}$,
since it is the source place where a polymorphic type is generated.
However, a too generalized type might reject applications due to its shape,
therefore, an instantiation procedure is added to eliminate all the universal quantifiers
on rule $\mathtt{HM \dash Var}$.
We omit rules $\mathtt{HM \dash Unit}$, $\mathtt{HM \dash Abs}$, and $\mathtt{HM \dash App}$
for the syntax-directed system $\Psi\vdash_{HM}^S$.
The following property shows that the new system is (almost) equivalent to the original declarative system.

\begin{theorem}[Equivalence of Syntax-Directed System]~
    \begin{enumerate}
        \item If $\Psi \vdash_{HM}^S e : \sigma$ then $\Psi \vdash_{HM} e : \sigma$
        \item If $\Psi \vdash_{HM} e : \sigma$ then $\Psi \vdash_{HM} e : \tau$,
                and $\all[\overline{a}]\tau \sqsubseteq \sigma$,
                where $\overline{a} = \text{FV}(\tau) - \text{FV}(\Psi)$.
    \end{enumerate}
\end{theorem}


\paragraph{Type Inference Algorithm}
Although being syntax-directed solves some problems, the rules still requires some guessings,
including rule $\mathtt{HM \dash Abs}$ and $\mathtt{HM \dash Var \dash Inst}$.
Algorithm W~\cite{milner1978theory}, based on unification,
is proven to be sound and complete w.r.t the declarative specifications.

\begin{theorem}[Algorithmic Completeness (Principality)]
    If $\Psi \vdash_{HM} e : \sigma$, then W computes a principal type sheme $\sigma_p$, i.e.
    \begin{enumerate}
        \item $\Psi \vdash_{HM} e : \sigma_p$
        \item $\sigma_p \sqsubseteq \sigma$.
    \end{enumerate}
\end{theorem}

\section{Odersky-L\"aufer Type System}\label{sec:bg:ol}


\subsection{Higher-Ranked Types}

The \emph{rank} of a type represents how deep a universal quantifier
appear at the contravariant position~\cite{rank1992kfoury}. Formally speaking,

\begin{gather*}
    \begin{aligned}
        \text{Rank 0 / Monotypes}\qquad& \tau, \sigma^0 &::=&\quad 1 \mid a \mid \tau_1\to \tau_2\\
        \text{Rank $k (k \ge 1)$, Polytypes}\qquad& \sigma^k &::=&\quad
            \sigma^{k-1} \mid \sigma^{k-1} \to \sigma^k \mid \all \sigma^k\\
    \end{aligned}
\end{gather*}

The following example illustrates what rank a type belongs to:

\begin{gather*}
    \begin{aligned}
        1 \to 1 && \text{Rank 0}\\
        \all a \to a && \text{Rank 1}\\
        1 \to \all a \to a && \text{Rank 1}\\
        (\all a \to a) \to (\all a \to a) && \text{Rank 2}\\
    \end{aligned}
\end{gather*}

According to the definition, monotypes are types that does not contain any universal quantifier.
In the HM type system, all polymorphic types have rank 1.

\jimmy{TODO Background about System F incompleteness, examples about higher-rank types}

\subsection{Declarative System}

\begin{figure}[t]
    \begin{gather*}
    \begin{aligned}
        \text{Type variables}\qquad&a, b\\
        \text{Types}\qquad&\sigma &::=&\quad 1 \mid a \mid \sigma_1\to \sigma_2 \mid \all \sigma\\
        \text{Monotypes}\qquad&\tau &::=&\quad 1 \mid a \mid \tau_1\to \tau_2\\
        \text{Expressions}\qquad&e &::=&\quad x \mid () \mid \lam[x:\sigma] e
            \mid e : \sigma \mid \lam e \mid e_1~e_2 \mid \letin{e_1}{e_2}\\
        \text{Contexts}\qquad&\Psi &::=&\quad \cdot \mid \Psi, x:\sigma \mid \Psi, a
    \end{aligned}
    \end{gather*}
\Description{Syntax of Odersky-L\"aufer System}
\caption{Syntax of Odersky-L\"aufer System}\label{fig:ol_decl_syntax}
\end{figure}

\newcommand{\vdashOL}{\mathrel{\vdash_{OL}}}

\begin{figure}[t]
    \begin{gather*}
        \inferrule*[right=$\mathtt{OL \dash WF \dash Unit}$]
            {~}{\Psi \vdashOL 1}
        \qquad
        \inferrule*[right=$\mathtt{OL \dash WF \dash TVar}$]
            {a \in \Psi}
            {\Psi \vdashOL a}
        \\
        \inferrule*[right=$\mathtt{OL \dash WF \dash Arr}$]
            {\Psi \vdashOL \sigma_1 \\ \Psi \vdashOL \sigma_2}
            {\Psi \vdashOL \sigma_1 \to \sigma_2}
        \qquad
        \inferrule*[right=$\mathtt{OL \dash WF \dash Forall}$]
            {\Psi, a \vdashOL \sigma}
            {\Psi \vdashOL \all \sigma}
    \end{gather*}
\Description{Well-formedness of types in the Odersky-L\"aufer System}
\caption{Well-formedness of types in the Odersky-L\"aufer System}\label{fig:ol_decl_wft}
\end{figure}


The syntax of Odersky-L\"aufer system is shown in Figure~\ref{fig:ol_decl_syntax}.
There are several differences compared to the HM system.

First, polymorphic types can be of arbitrary rank,
i.e. the forall quantifier may occur at any part of a type.
Yet, mono-type remains the same definition as HM's.

Second, expressions now allows annotations $e:\sigma$ and
(argument) annotated lambda functions $\lam[x:\sigma] e$.
Annotations on expressions help guide the type system properly,
acting as a machine-checked document by the programmers.
By annotating the argument of a lambda function with a polymorphic type $\sigma$,
one may encode a function of higher rank in this system compared to HM's.

Finally, contexts consist of not only variable bindings,
but also type variable declarations.
Here we adopt a slightly different approach than the original work~\cite{odersky1996putting},
which does not track type variables explicitly in a context.
Such explicit declarations reduce formalization difficulties especially
when dealing with freshness conditions or variable name encodings.
This also enables us to formally define well-formedness of types,
shown in Figure~\ref{fig:ol_decl_wft}.


\paragraph{Subtyping}

\begin{figure}[t]
    \begin{gather*}
        \inferrule*[right=$\mathtt{OL \dash SUB \dash Unit}$]
            {~}{\Psi \vdashOL 1 \le 1}
        \qquad
        \inferrule*[right=$\mathtt{OL \dash SUB \dash Var}$]
            {a \in \Psi}
            {\Psi \vdashOL a \le a}
        \\
        \inferrule*[right=$\mathtt{OL \dash SUB \dash Arr}$]
            {\Psi \vdashOL \sigma_1' \le \sigma_1 \\
                \Psi \vdashOL \sigma_2 \le \sigma_2'}
            {\Psi \vdashOL \sigma_1 \to \sigma_2 \le \sigma_1' \to \sigma_2'}
        \\
        \inferrule*[right=$\mathtt{OL \dash SUB \dash \forall L}$]
            {\Psi \vdashOL \tau \\ \Psi \vdashOL [\tau/a] \sigma \le \sigma'}
            {\Psi \vdashOL \all \sigma \le \sigma'}
        \qquad
        \inferrule*[right=$\mathtt{OL \dash SUB \dash \forall R}$]
            {\Psi, a \vdashOL \sigma \le \sigma'}
            {\Psi \vdashOL \sigma \le \all \sigma'}
    \end{gather*}
\Description{Subtyping of the Odersky-L\"aufer System}
\caption{Subtyping of the Odersky-L\"aufer System}\label{fig:ol_decl_sub}
\end{figure}

The subtyping relation, defined in Figure~\ref{fig:ol_decl_sub},
is more powerful than that (type instantiation) of HM.
In contrast to HM's subtyping, higher-ranked types can be compared thanks to
rule $\mathtt{OL \dash SUB \dash Arr}$.
Functions are contravariant on argument types and covariant on return types.
Rule $\mathtt{OL \dash SUB \dash \forall L}$ instantiates a polymorphic type
by a monotype $\tau$.
Rule $\mathtt{OL \dash SUB \dash \forall R}$ picks a fresh type variable
for the right-hand-side polymorphic type,
which is made possible by renaming according to alpha-equivalence.
Given that such implicit freshness condition is widely adopt,
we omit that throughout the thesis.

\paragraph{Typing}

\begin{figure}[t]
    \begin{gather*}
        \inferrule*[right=$\mathtt{OL \dash Var}$]
            {(x:\sigma) \in \Psi}
            {\Psi \vdashOL x : \sigma}
        \qquad
        \inferrule*[right=$\mathtt{OL \dash Unit}$]
            {~}
            {\Psi \vdashOL () : 1}
        \qquad
        \inferrule*[right=$\mathtt{OL \dash Anno}$]
            {\Psi \vdashOL e : \sigma}
            {\Psi \vdashOL (e:\sigma) : \sigma}
        \\
        \inferrule*[right=$\mathtt{OL \dash Lam}$]
            {\Psi \vdashOL \tau \\ \Psi, x:\tau \vdashOL e : \sigma}
            {\Psi \vdashOL \lam e : \tau \to \sigma}
        \qquad
        \inferrule*[right=$\mathtt{OL \dash LamAnno}$]
            {\Psi, x:\sigma_1 \vdashOL e : \sigma_2}
            {\Psi \vdashOL \lam[x:\sigma_1] e : \sigma_1 \to \sigma_2}
        \\
        \inferrule*[right=$\mathtt{OL \dash App}$]
            {\Psi \vdashOL e_1 : \sigma_1 \to \sigma_2 \\ \Psi \vdashOL e_2 : \sigma_1}
            {\Psi \vdashOL e_1~e_2 : \sigma_2}
        \qquad
        \inferrule*[right=$\mathtt{OL \dash Gen}$]
            {\Psi, a \vdashOL e : \sigma}
            {\Psi \vdashOL e : \all \sigma}
        \\
        \inferrule*[right=$\mathtt{OL \dash Let}$]
            {\Psi \vdashOL e_1 : \sigma_1 \\ \Psi, x:\sigma_1 \vdashOL e_2 : \sigma_2}
            {\Psi \vdashOL \letin{e_1}{e_2} : \sigma_2}
        \qquad
        \inferrule*[right=$\mathtt{OL \dash Sub}$]
            {\Psi \vdashOL e : \sigma_1 \\ \Psi \vdashOL \sigma_1 \le \sigma_2}
            {\Psi \vdashOL e : \sigma_2}
    \end{gather*}
\Description{Typing of the Odersky-L\"aufer System}
\caption{Typing of the Odersky-L\"aufer System}\label{fig:ol_decl_typing}
\end{figure}

The type system of Odersky-L\"aufer, shown in Figure~\ref{fig:ol_decl_typing},
extends HM's type system in the following aspects.

Rule $\mathtt{OL \dash Lam}$ now accepts polymorphic return type,
because such type is well-formed.
The guess on parameter types is still limited to monotypes like HM's.
However, if a parameter type is specified in advance,
the type system accepts polymorphic argument type with rule $\mathtt{OL \dash LamAnno}$.
Functions of arbitrary rank can be encoded through proper annotations.
The application and let-generalization rules also accept polymorphic return types.

Rule $\mathtt{OL \dash Gen}$ encodes the generalization rule of HM in a different way
under explicit type variable declarations.
A fresh type variable is introduced into the context before
the type of expression $e$ is calculated.
Then we conclude that $e$ has a polymorphic type by generalizing the type variable.
For example, the type of the identity function is derived as follows
$$
\inferrule*[Right=$\mathtt{OL \dash Gen}$]
{
    \inferrule*[Right=$\mathtt{OL \dash Lam}$]
    {
        \cdot, a \vdashOL a
        \\
        \inferrule*[Right=$\mathtt{OL \dash Var}$]
        {(x:a) \in (\cdot, a, x : a)}
        {\cdot, a, x : a \vdashOL x : a}
    }
    {\cdot, a \vdashOL \lam x : a \to a}
}
{\cdot \vdashOL \lam x : \all a \to a}
$$

The subsumption rule $\mathtt{OL \dash Sub}$ converts the type of an expression
with the help of the subtyping relation.

\subsection{Relating to HM}




\section{Dunfield's Bidirectional Type System}

\subsection{Declarative System}\label{subsec:dk:decl}

\begin{figure}[t]
\[
\begin{array}{l@{\qquad}lcl}
\text{Type variables}\qquad&a, b
\\
\text{Types}\qquad&A, B, C &::=&\quad 1 \mid a \mid \all A \mid A\to B\\
\text{Monotypes}\qquad&\tau,\sigma &::=&\quad 1 \mid a \mid \tau\to \sigma
\\
\text{Expressions}\qquad&e &::=&\quad x \mid () \mid \lam e \mid e_1~e_2 \mid (e:A)
\\
\text{Contexts}\qquad&\Psi &::=&\quad \nil \mid \Psi, a \mid \Psi, x:A
\end{array}
\]
\Description{Syntax of Declarative System}
\caption{Syntax of Declarative System}\label{fig:decl:syntax}
\end{figure}

\paragraph{Syntax.}
The syntax of DK's declarative system~\cite{dunfield2013complete} is shown in Figure~\ref{fig:decl:syntax}.
A declarative type $A$ is either the unit type $1$, a type variable $a$,
a universal quantification $\all A$ or a function type $A \to B$.
Nested universal quantifiers are allowed for types,
but monotypes $\tau$ do not have any universal quantifier.
Terms include a unit term $()$, variables $x$, lambda-functions $\lam e$,
applications $e_1~e_2$ and annotations $(e:A)$.
Contexts $\Psi$ are sequences of type variable declarations and
term variables with their types declared $x:A$.

\paragraph{Well-formedness} Well-formedness of types and terms is 
shown at the top of Figure~\ref{fig:decl:sub}. The rules are standard
and simply ensure that variables in types and terms are well-scoped.  

\begin{figure}[t]
\centering \framebox{$\Psi \vdash A$} Well-formed declarative type
\begin{gather*}
\inferrule*[right=$\mathtt{wf_d unit}$]
    {~}{\Psi\vdash 1}
\qquad
\inferrule*[right=$\mathtt{wf_d var}$]
    {a\in\Psi}{\Psi\vdash a}
\qquad
\inferrule*[right=$\mathtt{wf_d{\to}}$]
    {\Psi\vdash A\quad \Psi\vdash B}
    {\Psi\vdash A\to B}
\qquad
\inferrule*[right=$\mathtt{wf_d\forall}$]
    {\Psi, a\vdash A}
    {\Psi\vdash \forall a. A}
\end{gather*}

\centering \framebox{$\Psi \vdash e$} Well-formed declarative expression
\begin{gather*}
\inferrule*[right=$\mathtt{wf_d tmvar}$]
    {x:A\in\Psi}{\Psi\vdash x}
\qquad
\inferrule*[right=$\mathtt{wf_d tmunit}$]
    {~}{\Psi\vdash ()}
\qquad
\inferrule*[right=$\mathtt{wf_d abs}$]
    {\Psi,x:A\vdash e}
    {\Psi\vdash \lam e}
\\
\inferrule*[right=$\mathtt{wf_d app}$]
    {\Psi\vdash e_1 \quad \Psi\vdash e_2}
    {\Psi\vdash e_1~e_2}
\qquad
\inferrule*[right=$\mathtt{wf_d anno}$]
    {\Psi\vdash A \quad \Psi\vdash e}
    {\Psi\vdash (e:A)}
\end{gather*}

\centering \framebox{$\Psi \vdash A \le B$} Declarative subtyping
\begin{gather*}
\inferrule*[right=$\mathtt{{\le}Var}$]
    {a\in\Psi}{\Psi\vdash a\le a}
\qquad
\inferrule*[right=$\mathtt{{\le}Unit}$]
    {~}{\Psi \vdash 1 \le 1}
\qquad
\inferrule*[right=$\mathtt{{\le}{\to}}$]
    {\Psi \vdash B_1 \le A_1 \quad \Psi \vdash A_2 \le B_2}
    {\Psi\vdash A_1\to A_2 \le B_1\to B_2}
\\
\inferrule*[right=$\mathtt{{\le}\forall L}$]
    {\Psi\vdash \tau \quad \Psi\vdash [\tau/a] A \le B}
    {\Psi\vdash \all A \le B}
\qquad
\inferrule*[right=$\mathtt{{\le}\forall R}$]
    {\Psi, b\vdash A\le B}
    {\Psi\vdash A \le \all[b]B}
\end{gather*}
\Description{Declarative Well-formedness and Subtyping}
\caption{%Well-formedness of Declarative Types and 
Declarative Well-formedness and Subtyping}\label{fig:decl:sub}
\end{figure}

%The DK subtyping relation was adopted from \citet{odersky1996putting}.

\paragraph{Declarative Subtyping}
The bottom of Figure~\ref{fig:decl:sub} shows DK's declarative subtyping judgment $\Psi \vdash A \le B$,
which was adopted from \citet{odersky1996putting}. It compares the
degree of polymorphism between $A$ and $B$ in DK's implicit polymorphic type system. 
Essentially, if $A$ can always be instantiated to match any instantiation of $B$,
then A is ``at least as polymorphic as'' $B$. We also 
say that $A$ is ``more polymorphic than'' $B$ and write $A \le B$.

Subtyping rules $\mathtt{{\le}Var}$, $\mathtt{{\le}Unit}$ and $\mathtt{{\le}{\to}}$
handle simple cases that do not involve universal quantifiers.
The subtyping rule for function types $\mathtt{{\le}{\to}}$ is standard,
being covariant on the return type and contravariant on the argument type.
Rule $\mathtt{{\le}\forall R}$ states that if $A$ is a subtype of $B$
in the context $\Psi, a$, where $a$ is fresh in $A$, then $A\le\all B$.
Intuitively, if $A$ is more general than $\all B$ (where the universal quantifier
already indicates that $\all B$ is a general type),
then $A$ must instantiate to $[\tau/a]B$ for every $\tau$.

The most interesting rule is $\mathtt{{\le}\forall L}$.
If some instantiation of $\all A$, $[\tau/a]A$, is a subtype of $B$,
then $\all A \le B$.
The monotype $\tau$ we used to instantiate $a$ is \emph{guessed} in this
declarative rule, but the algorithmic system does not guess and defers the
instantiation until it can determine the monotype deterministically.
The fact that $\tau$ is a monotype rules out the possibility of
polymorphic (or impredicative) instantiation.
However this restriction ensures that the subtyping relation remains
decidable. Allowing an arbitrary type (rather than a monotype) in rule $\mathtt{{\le}\forall L}$
is known to give rise to an undecidable subtyping relation~\cite{tiuryn1996subtyping}.
\citet{jones2007practical} also impose the restriction of
predicative instantiation in their type system.
Both systems are adopted by several practical programming languages.

% \jimmy{Mention the implicit freshness conditions in the premises}
Note that when we introduce a new binder in the premise, we implicitly pick a fresh one.
This applies to rules such as $\mathtt{wf_d\forall}$, $\mathtt{wf_dabs}$, $\mathtt{{\le}\forall R}$,
throughout the whole text.

\begin{figure}[t]
\begin{tabular}{rl}
    \framebox{$\Psi \vdash e \Lto A$} & $e$ checks against input type $A$.\\[0.5mm]
    \framebox{$\Psi \vdash e \To A$} & $e$ synthesizes output type $A$.\\[0.5mm]
    \framebox{$\Psi \vdash \appInf{A}{e}{C}$} & Applying a function of type $A$ to $e$ synthesizes type $C$.
\end{tabular}
\begin{gather*}
\inferrule*[right=$\mathtt{DeclVar}$]
    {(x:A)\in\Psi}{\Psi\vdash x\To A}
\qquad
\inferrule*[right=$\mathtt{DeclSub}$]
%e \neq \lam e' \quad B \neq \all B' \quad 
    {\Psi\vdash e\To A \quad \Psi\vdash A\le B}
    {\Psi \vdash e\Lto B}
\\
\inferrule*[right=$\mathtt{DeclAnno}$]
    {\Psi \vdash A \quad \Psi\vdash e\Lto A}
    {\Psi\vdash (e:A)\To A}
\qquad
\inferrule*[right=$\mathtt{Decl1I}$]
    {~}{\Psi\vdash () \Lto 1}
\qquad
\inferrule*[right=$\mathtt{Decl1I{\To}}$]
    {~}{\Psi\vdash () \To 1}
\\
\inferrule*[right=$\mathtt{Decl\forall I}$]
    {\Psi,a \vdash e \Lto A}
    {\Psi\vdash e\Lto \all A}
\qquad
\inferrule*[right=$\mathtt{Decl\forall App}$]
    {\Psi \vdash \tau \quad \Psi\vdash \appInf{[\tau/a]A}{e}{C} }
    {\Psi\vdash \appInf{\all A}{e}{C}}
\\
\inferrule*[right=$\mathtt{Decl{\to}I}$]
    {\Psi,x:A \vdash e\Lto B}
    {\Psi\vdash \lam e \Lto A \to B}
\qquad
\inferrule*[right=$\mathtt{Decl{\to}I{\To}}$]
    {\Psi\vdash \sigma\to\tau \quad \Psi,x:\sigma \vdash e\Lto \tau}
    {\Psi\vdash \lam e \To \sigma\to\tau}
\\
\inferrule*[right=$\mathtt{Decl{\to} E}$]
    {\Psi\vdash e_1\To A \quad \Psi\vdash \appInf{A}{e_2}{C}}
    {\Psi\vdash e_1~e_2 \To C}
\qquad
\inferrule*[right=$\mathtt{Decl{\to}App}$]
    {\Psi\vdash e \Lto A}
    {\Psi\vdash \appInf{A \to C}{e}{C}}
\end{gather*}
\Description{Declarative Typing}
\caption{Declarative Typing}\label{fig:decl:typing}
\end{figure}

\paragraph{Declarative Typing}
The bidirectional type system, shown in Figure~\ref{fig:decl:typing}, has three judgments.
The checking judgment $\Psi\vdash e\Lto A$ checks expression $e$ against the type $A$ in the context $\Psi$.
The synthesis judgment $\Psi\vdash e\To A$ synthesizes the type $A$ of expression $e$ in the context $\Psi$.
The application judgment $\Psi\vdash \appInf{A}{e}{C}$ synthesizes the type $C$ of the application of a function of type $A$
(which could be polymorphic) to the argument $e$.

Many rules are standard.
Rule $\mathtt{DeclVar}$ looks up term variables in the context.
Rules $\mathtt{Decl1I}$ and $\mathtt{Decl1I{\To}}$ respectively check and synthesize the unit type.
Rule $\mathtt{DeclAnno}$ synthesizes the annotated type $A$ of the annotated expression $(e:A)$
and checks that $e$ has type $A$.
Checking an expression $e$ against a polymorphic type $\all A$ in the context $\Psi$ succeeds
if $e$ checks against $A$ in the extended context $(\Psi, a)$.
The subsumption rule $\mathtt{DeclSub}$ depends on the subtyping relation,
and changes mode from checking to synthesis: if $e$ synthesizes type $A$ and $A\le B$
($A$ is more polymorphic than $B$), then $e$ checks against $B$.
If a checking problem does not match any other rules,
this rule can be applied to synthesize a type instead and then
check whether the synthesized type entails the checked type.
Lambda abstractions are the hardest construct of the bidirectional
type system to deal with. 
Checking $\lam e$ against function type $A\to B$ is easy:
we check the body $e$ against $B$ in the context extended with $x:A$.
However, synthesizing a lambda-function is a lot harder, and 
this type system only synthesizes monotypes $\sigma\to\tau$.

Application $e_1~e_2$ is handled by Rule $\mathtt{Decl{\to}E}$,
which first synthesizes the type $A$ of the function $e_1$.
If $A$ is a function type $B\to C$, Rule $\mathtt{Decl{\to}App}$ is applied;
it checks the argument $e_2$ against $B$ and returns type $C$.
The synthesized type of function $e_1$ can also be polymorphic, of the form $\all A$.
In that case, we instantiate $A$ to $[\tau/a]A$ with a monotype $\tau$ % (which is also guessed)
using according to Rule $\mathtt{Decl{\to}I{\To}}$.
If $[\tau/a]A$ is a function type, Rule $\mathtt{Decl{\to}App}$ proceeds;
if $[\tau/a]A$ is another universal quantified type,
Rule $\mathtt{Decl{\to}I{\To}}$ is recursively applied.




\section{MLsub}\label{sec:mlsub}

MLsub~\cite{mlsub} extends the HM type system with subtyping.
In presence of subtyping, type inference does not simply handle equality during unification.
Therefore, types are extended with lattice to express bounds properly.
Furthermore, polar types are introduced to help separate input and output types,
which simplifies the type inference algorithm.
Like HM's type inference, MLsub always infers a principal type.

\subsection{Types and Polar Types}

\begin{figure}[t]
    \[
    \begin{array}{l@{\qquad}lcl}
    \text{Types}\qquad&\tau &::=&\quad 1 \mid a \mid \top \mid \bot \mid \tau_1 \to \tau_2 \mid
        \tau_1 \sqcup \tau_2 \mid \tau_1 \sqcap \tau_2 \\
    \text{Positive Types}\qquad&\tau^+ &::=&\quad 1 \mid a \mid \bot \mid \tau_1^-\to \tau_2^+ \mid
        \tau_1^+ \sqcup \tau_2^+\\
    \text{Negative Types}\qquad&\tau^- &::=&\quad 1 \mid a \mid \top \mid \tau_1^+\to \tau_2^- \mid
        \tau_1^- \sqcap \tau_2^-\\
    \end{array}
    \]
    \Description{Types of MLsub}
    \caption{Types of MLsub}\label{fig:mlsub:types}
\end{figure}

In comparison to the type system of HM, types (Figure~\ref{fig:mlsub:types})
now include $\top$ and $\bot$,
as minimal components to support subtyping.
Besides, the least-upper-bound ($\sqcup$) and greatest-lower-bound ($\sqcap$) lattice operations
are used to represent a bound expressed by two types.
For finite types, a distributive lattice can be defined via a set of equivalence classes
of $\equiv$~\citep{mlsub}.
The most interesting equations are the distributivity rule and rules for function types:

\[\begin{aligned}
    \tau_1 \sqcup (\tau_2 \sqcap \tau_3) &\equiv
        (\tau_1 \sqcup \tau_2) \sqcap (\tau_1 \sqcup \tau_3)\\
    \tau_1 \sqcap (\tau_2 \sqcup \tau_3) &\equiv
        (\tau_1 \sqcap \tau_2) \sqcup (\tau_1 \sqcap \tau_3)\\
    (\tau_1 \to \tau_2) \sqcup (\tau_1' \to \tau_2') &\equiv 
        (\tau_1 \sqcap \tau_1') \to (\tau_2 \sqcup \tau_2')\\
    (\tau_1 \to \tau_2) \sqcap (\tau_1' \to \tau_2') &\equiv 
        (\tau_1 \sqcup \tau_1') \to (\tau_2 \sqcap \tau_2')\\
\end{aligned}\]

The partial order $\tau_1 \le \tau_2$ is defined as $\tau_1 \sqcup \tau_2 \equiv \tau_2$ or
$\tau_1 \sqcap \tau_2 \equiv \tau_1$.
$\top$ and $\bot$ are the least and greatest types.
The above rules on function types imply the usual subtyping rule for function types,
considering the definition of partial order:
\[
    \inferrule*
    {\tau_1' \le \tau_1 \\ \tau_2 \le \tau_2'}
    {\tau_1 \to \tau_2 \le \tau_1' \to \tau_2'}
\]

Type schemes are not defined as the usual $\sigma$.
Instead, a monotype $\tau$ already represent a type scheme by
omitting the $\forall$ quantifiers---all the
free type variables are implicitly generalized.

Recursive types play an important role regarding the principality of type inference,
but we omit them for simplicity.

\paragraph{Polar Types}
Polar types are restrictions on the lattice operations;
they should not occur arbitrarily in any position.
Specifically, function outputs consist of types ($\tau_1, \tau_2$) from different branches,
resulting in $\tau_1 \sqcup \tau_2$;
a function input might be used in various ways (under different constraints),
thus $\tau_1 \sqcap \tau_2$ is more suitable.
In summary, $\sqcup$ only arises in return types,
while $\sqcap$ only arises in argument types.
Figure~\ref{fig:mlsub:types} formally defines the restriction,
where positive types $\tau^+$ describe return types,
and negative types $\tau^-$ describe argument types.

An important consequence is that all the constraints are of the form
$\tau^+ \le \tau^-$, which represents the subtyping relation when
using an output expression in a function application as argument.
The following subtyping rules involving the lattice operations
reflects their basic properties:
\[
    \inferrule*
        {\tau_1^+ \le \tau^- \\ \tau_2^+ \le \tau^-}
        {\tau_1^+ \sqcup \tau_2^+ \le \tau^-}
    \qquad
    \inferrule*
        {\tau^+ \le \tau_1^- \\ \tau^+ \le \tau_2^-}
        {\tau^- \le \tau_1^- \sqcap \tau_2^-}
    .
\]
Interestingly, the polar subtyping judgments avoids difficult judgments like
$\tau_1 \sqcap \tau_2 \le \tau$ or $\tau \le \tau_1 \sqcup \tau_2$
through its syntactic restriction.



\subsection{Biunification}

Type inference for MLsub, similar to that for ML, is mainly a unification algorithm.
However, in presence of subtyping,
equality-based unification loses information about subtyping constraints.

For the atomic constraint $\al = \tau$ where $\al \notin \text{FV}(\tau)$,
the ML unification algorithm produces the substitution $[\tau/\al]$.
In constrast, an MLsub atomic constraint might be $\al \le \tau$,
and the substitution $[\tau/\al]$ treat the subtyping constraint as a equality constraint,
which eliminates a whole set of possibilities.

Luckily, lattices in MLsub helps express subtyping constraints on types directly.
The constraint $\al \le \tau$ ($\al \notin \text{FV}(\tau)$) may produce the substitution
$[\al \sqcap \tau/\al]$, since $\al \sqcap \tau \le \tau$.
In the meantime, $\al \sqcap \tau$ does not lose any expressiveness:
for any $\tau_0$ s.t. $\tau_0 \le \tau$,
picking $\al = \tau_0$ gives $\al \sqcap \tau = \tau_0$,
and the substitution $[\al \sqcap \tau/\al]$ is equivalent to $[\tau_0/\al]$.

In presence of polar types, the biunification algorithm of MLsub
produces a \emph{bisubstitution} $[\al \sqcap \tau^-/\al^-]$
against the constraint $\al \le \tau^-$,
where only negative occurances are substituted,
keeping polar types properly ``polarized''.
For example, a positive type $\al \to \al$ becomes $(\al \sqcap \tau^-) \to \al$
under such substitution and remains a positive type.
A more important fact is that this type is equivalent to
the original type with the constraint $\al \le \tau^-$.
Similarly, a constraint like $\tau^+ \le \al$ is reduced to
a substitution $[\al \sqcup \tau^+/\al^+]$.

For example, the \emph{choose} function is typed $\all a \to a \to a$ in ML.
However, MLsub might also infer an equivalent type
$\all[a~b] a \to b \to a \sqcup b$.
One can easily read the MLsub type in a form
where constraints are explicitly stated
\[\all[a~b~c] a \to b \to c \text{ where } a \le c, b \le c.\]
Therefore, MLsub encodes the constraints directly onto types with the help of
the lattice operations.
Furthermore, a simplification step is taken after the type inference algorithm,
reducing the size and improving readablity of the type inferred.

As a result, biunification for MLsub extends unification for ML,
accepting subtyping in addition to type schemes,
while maintaining principality.

