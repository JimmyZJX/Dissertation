\section{Related Work}

Throughout the paper we have already discussed much of the closest related work.
In this section we summarize the key differences and novelties, and we discuss some other
related work.

\paragraph{Predicative Higher-Ranked Polymorphism Type Inference Algorithms}
Higher-ranked polymorphism is a convenient and practical feature of
programming languages.  Since full type-inference for System F is
undecidable~\cite{wells1999typability}, various decidable partial
type-inference algorithms were developed.
% No need to cite here: the following text is describing the point ~\cite{}.
The declarative system of this paper,
proposed by \citet{dunfield2013complete}, is \emph{predicative}: 
$\forall$'s only instantiate to monotypes.  The monotype restriction
on instantiation is considered reasonable and practical for most
programs, except for those that require sophisticated forms of
higher-order polymorphism.  In those cases, the bidirectional system
accepts guidance through type annotations, which allow polymorphic types.
Such annotations also improve
readability of the program, and are not much of a burden in practice.

DK's algorithm is shown to be sound, complete and decidable in 70 pages of manual proofs.
Though carefully written, some of the proofs are incorrect
(see discussion in Appendix
%~\ref{appendix:false_lemmas} and \ref{appendix:subsumption}
),
which creates difficulties when formalizing them in a proof assistant.
In their follow-up work \citet{DunfieldIndexed} enrich the bidirectional higher-rank system with
existentials and indexed types.
With a more complex declarative system, they developed a proof of over 150 pages.
It is even more difficult to argue its correctness for every single detail within such a big development.
Unfortunately, we find that their Lemma 26 (Parallel Admissibility) appears to have the same issue 
as lemma 29 in \citet{dunfield2013complete}: the conclusion is false. We also discuss
the issue in more detail in Appendix. %~\ref{appendix:false_lemmas}.

\citet{jones2007practical} developed another higher-rank predicative bidirectional type system.
Their subtyping relation is enriched with \emph{deep skolemisation},
which is more general than ours and allows more valid relations.
In comparison to DK's system, they do not use the application inference judgment,
resulting in a complicated mechanism for implicit instantiation taken care by the unification process for the algorithm.
A manual proof is given, showing that the algorithm is sound and
complete with respect to their declarative specification.

In a more recent work, \citet{xie2018letarguments} proposed a variant of a
bidirectional type inference system for a predicative system with higher-ranked types.
Type information flows from arguments to
functions with an additional \emph{application} mode. This variant 
allows more higher-order typed programs to be inferred without additional annotations.
Following the new mode, the let-generalization of the Hindley-Milner system
is well supported as a syntactic sugar. The formalization includes some
mechanized proofs for the declarative type system, but all proofs regarding
the algorithmic type system are manual.

\paragraph{Impredicative Higher-Ranked Polymorphism Type Inference Algorithms}
Impredicative System F allows instantiation with polymorphic types,
but unfortunately its subtyping system is already undecidable~\cite{tiuryn1996subtyping}.
Works on partial impredicative type-inference algorithms~\cite{le2003ml,leijen2008hmf,vytiniotis2008fph}
navigate a variety of design tradeoffs for a decidable algorithm.
As a result, such algorithms tend to be more complicated, and thus less adopted in practice.
Recent work proposed \emph{Guarded Impredicative Polymorphism}~\cite{Serrano2018},
as an improvement on GHC's type inference algorithm with impredicative instantiation.
They make use of local information in $n$-ary applications to
infer polymorphic instantiations with a relatively simple specification and unification algorithm.
Although not all impredicative instantiations can be handled well,
their algorithm is already quite useful in practice.

\paragraph{Mechanical Formalization of Polymorphic Subtyping}
In all previous work on type inference for higher-ranked polymorphism
(predicative and impredicative) discussed above, proofs and
metatheory for the algorithmic aspects are manual. The only partial effort on mechanizing algorithmic
aspects of type inference
for higher-ranked types is
the Abella formalization of \emph{polymorphic subtyping} by \citet{itp2018}.
The judgment form of worklist $\Gm \vdash \Omega$ used in the formalization simplifies
the propagation of existential variable instantiations.
However, the approach has two main drawbacks:
it does not collect unused variable declarations effectively;
and the simple form of judgment cannot handle inference modes, which output types.
The new worklist introduced in this paper inherits the simplicity of propagating instantiations,
but overcomes both of the issues by mixing judgments with declarations
and using the continuation-passing-style judgment chains. Furthermore,
we formalize the complete bidirectional type system by
\citet{dunfield2013complete}, whereas Zhao et al. only formalize
the subtyping relation. 

\paragraph{Mechanical Formalizations of Other Type-Inference Algorithms}
Since the publication of the \textsc{POPLMark} challenge~\cite{aydemir2005mechanized},
many theorem provers and packages provide new methods for dealing
with variable binding~\cite{aydemir2008engineering,urban2008nominalTech,chlipala2008parametric}.
More and more type systems are formalized with these tools.
However, mechanizing certain algorithmic aspects, like unification and
constraint solving, has received very little attention and is still challenging.
Moreover, while most tools support local (input) contexts in a neat way,
many practical type-inference algorithms require
more complex binding structures with output contexts or various forms of constraint solving procedures.

Algorithm $\mathcal{W}$,
as one of the classic type inference algorithms for polymorphic type systems,
has been manually proven to be sound and complete
with respect to the Hindley-Milner type system~\cite{hindley1969principal,milner1978theory,damas1982principal}.
After around 15 years, the algorithm was formally verified by
\citet{naraschewski1999type} in Isabelle/HOL~\cite{nipkow2002isabelle}.
The treatment of new variables was tricky at that time, while the overall structure follows the
structure of Damas's manual proof closely.
%Another complication is the encoding of substitutions,
%which they chose to formalize as a function instead of an association list.
Later on, other researchers~\cite{dubois2000proving,dubois1999certification}
formalized algorithm $\mathcal{W}$ in Coq~\cite{Coq}.
Nominal techniques~\cite{urban2008nominalTech} in Isabelle/HOL have been
developed to help programming language formalizations, and are used for a similar
verification~\cite{urban2008nominal}. Moreover, Garrigue~\cite{garrigue2015certified}
mechanized a type inference algorithm,
with the help of locally nameless~\cite{LocallyNameless},
for Core ML extended with structural polymorphism and recursion.

\paragraph{Ordered Contexts in Type Inference}
Gundry et al.~\cite{gundry2010type} revisit algorithm $\mathcal{W}$ and
propose a new unification algorithm with the help of ordered contexts.
Similar to DK's algorithm, information of meta variables flow from input contexts to output contexts.
Not surprisingly, its information increase relation has a similar role to DK's context extension.
Our algorithm, in contrast,
eliminates output contexts and solution records ($\al = \tau$),
simplifying the information propagation process through immediate substitution
by collecting all the judgments in a single worklist.

\paragraph{The Essence of ML Type Inference}
Constraint-based type inference is adopted by \citet{remy-attapl} for
ML type systems, which do not employ higher-ranked polymorphism. An
interesting feature of their algorithm is that it keeps precise
scoping of variables, similarly to our approach.  Their algorithm is
divided into constraint generation and solving phases (which are
typical of constraint-based algorithms). Furthermore an intermediate
language is used to describe constraints and their constraint solver
utilizes a stack to track the state of the solving process.  In
contrast, our algorithm has a single phase, where the judgment chains
themselves act as constraints, thus no separate constraint language is
needed.

%A possible advantage of a separate constraint language is to provide
%a general and reuseable interface
%that interprets the desired properties for type inference.

\paragraph{Lists of Judgments in Unification}
Some work~\cite{Reed2009,Abel2011higher} adopts a similar idea to this paper
in work on unification for dependently typed languages. Similarly to our work
the algorithms need to be very careful about scoping, since the order of variable
declarations is fundamental in a dependently typed setting. 
Their algorithms simplify a collection of unification constraints progressively in a single-step style.
In comparison, our algorithm mixes variable declarations with judgments,
resulting in a simpler judgment form,
while processing them in a similar way.
One important difference is that contexts are
duplicated in their unification judgments, which complicates the unification process,
since the information of each local context needs to be synchronized.
Instead we make use of the nature of ordered context to control the scopes of unification variables.
While their algorithms focus only on unification,
our algorithm also deals with other types of judgments like synthesis.
A detailed discussion is in Section~\ref{sec:overview:list}.
