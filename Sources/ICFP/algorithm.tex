\section{Algorithmic System}

This section introduces a novel algorithmic system that implements 
DK's declarative specification. The new algorithm extends the idea
of worklists in Chapter~\ref{chap:ITP} in two ways. Firstly,
unlike its worklists, the scope of variables is precisely tracked
and variables do not escape their scope. This is achieved by unifying algorithmic contexts and the worklists themselves.
Secondly, our algorithm also
accounts for the type system (and not just subtyping). To deal with
inference judgments that arise in the type system we employ a \emph{continuation
passing style} to enable the transfer of inferred information across
judgments in a worklist.

\subsection{Syntax and Well-Formedness}

Figure~\ref{fig:alg:syntax} shows the syntax and well-formedness
judgments used by the algorithm. Similarly to the declarative system 
the well-formedness rules are unsurprising and merely ensure 
well-scopedness.

\begin{figure}
\begin{gather*}
\begin{aligned}
% \text{Type variables}\qquad&a, b\\
\text{Existential variables}\qquad&\al, \bt
\\
%\text{Types}\qquad&A', B', C' &::=&\quad 1 \mid a \mid \forall x. A' \mid A'\to B'\\
% \text{Mono-types}\qquad&\tau &::=&\quad 1 \mid a \mid \tau_1\to \tau_2\\
%\text{Context}\qquad&\Psi &::=&\quad \nil \mid \Psi, a \mid \Psi, x:A' \color{red} \\[2mm]
% \text{Declarative Judgments}\qquad&\jo &::=&\quad \nil \mid \P\vdash A' \le B' \jc \jo\\
% &&&\quad \mid \P\vdash e\Leftarrow A' \jc \jo \mid \P\vdash e\To A' \jc \jo \mid \P\vdash A' \bullet e\TTo C' \jc \jo\\
% \noalign{\jimmy{\text{What do the markers do: $\G \vdash,a \vdash,b\vdash \j_1;\j_2;\j_3$ means $\G,a,b\vdash \j_1 \land \G,a\vdash \j_1,\j_2 \land \G\vdash \j_3$}}} \\[2mm]
\text{Algorithmic types}\qquad&A, B, C &::=& \quad 1 \mid a \mid \all A \mid A\to B \mid \al % \quad 1 \mid a \mid \al \mid \forall a. A \mid A\to B
\\
% \text{Terms}\qquad&e&::=&\quad x \mid () \mid \lam{e} \mid e_1~e_2 \mid (e:A)
% \\[2mm]
\text{Algorithmic judgment chain}\qquad&\jg &::=&\quad A \le B \mid e\Lto A \mid e\To_{a} \jg \mid \appInfAlg{A}{e}
\\
\text{Algorithmic worklist}\qquad&\Gm &::=&\quad \nil \mid \Gm, a \mid \Gm, \al \mid \Gm, x: A \mid \Gm \Vdash \jg%\\
% \text{Declarative worklist}\qquad&\Om &::=&\quad \nil \mid \Om, a \mid \Om, x: A \mid \Om \Vdash \jg
\end{aligned}
\end{gather*}

\centering \framebox{$\Gm \vdash A$} Well-formed algorithmic type
\begin{gather*}
\inferrule*[right=$\mathtt{wf\_ unit}$]
    {~}{\Gm\vdash 1}
\quad
\inferrule*[right=$\mathtt{wf\_ var}$]
    {a\in\Gm}{\Gm\vdash a}
\quad
\inferrule*[right=$\mathtt{wf\_ exvar}$]
    {\al\in\Gm}{\Gm\vdash \al}
\quad
\inferrule*[right=$\mathtt{wf\_{\to}}$]
    {\Gm\vdash A\quad \Gm\vdash B}
    {\Gm\vdash A\to B}
\quad
\inferrule*[right=$\mathtt{wf\_\forall}$]
    {\Gm, a\vdash A}
    {\Gm\vdash \forall a. A}
\end{gather*}

\centering \framebox{$\Gm \vdash e$} Well-formed algorithmic expression
\begin{gather*}
\inferrule*[right=$\mathtt{wf\_ tmvar}$]
    {x:A\in\Gm}{\Gm\vdash x}
\qquad
\inferrule*[right=$\mathtt{wf\_ tmunit}$]
    {~}{\Gm\vdash ()}
\qquad
\inferrule*[right=$\mathtt{wf\_ abs}$]
    {\Gm,x:A\vdash e}
    {\Gm\vdash \lam e}
\\
\inferrule*[right=$\mathtt{wf\_ app}$]
    {\Gm\vdash e_1 \quad \Gm\vdash e_2}
    {\Gm\vdash e_1~e_2}
\qquad
\inferrule*[right=$\mathtt{wf\_ anno}$]
    {\Gm\vdash A \quad \Gm\vdash e}
    {\Gm\vdash (e:A)}
\end{gather*}

\framebox{$\Gm\vdash\jg$} Well-formed algorithmic judgment
\begin{gather*}
\inferrule*[right=$\mathtt{wf{\le}}$]
{\Gm\vdash A \\ \Gm\vdash B}
{\Gm\vdash A\le B}
\qquad
\inferrule*[right=$\mathtt{wf{\Lto}}$]
{\Gm\vdash e \\ \Gm\vdash A}
{\Gm\vdash e \Lto A}
\\
\inferrule*[right=$\mathtt{wf{\To}}$]
{\Gm\vdash e \\ \Gm, a\vdash \jg}
{\Gm\vdash e \To_a \jg}
\qquad
\inferrule*[right=$\mathtt{wf{\TTo}}$]
{\Gm\vdash A \\ \Gm, a\vdash \jg \\ \Gm\vdash e}
{\Gm\vdash \appInfAlg{A}{e}}
\end{gather*}

\framebox{$\text{wf }\Gm$} Well-formed algorithmic worklist
\begin{gather*}
\inferrule*[right=$\mathtt{wf\nil}$]
{~}{\text{wf }\nil}
\qquad
\inferrule*[right=$\mathtt{wf_a}$]
{\text{wf }\Gm}
{\text{wf }\Gm, a}
\qquad
\inferrule*[right=$\mathtt{wf_{\al}}$]
{\text{wf }\Gm}
{\text{wf }\Gm, \al}
\qquad
\inferrule*[right=$\mathtt{wf_{of}}$]
{\text{wf }\Gm \\ \Gm\vdash A}
{\text{wf }\Gm, x:A}
\qquad
\inferrule*[right=$\mathtt{wf_{\jg}}$]
{\text{wf }\Gm \\ \Gm\vdash\jg}
{\text{wf }\Gm \Vdash\jg}
\end{gather*}
\Description{Extended Syntax and Well-Formedness for the Algorithmic System}
\caption{Extended Syntax and Well-Formedness for the Algorithmic System}\label{fig:alg:syntax}
\end{figure}

\paragraph{Existential Variables} 
The algorithmic system inherits the syntax of terms and types from 
the declarative system. It only introduces one additional feature.
In order to find unknown types $\tau$ in the declarative system, the
algorithmic system extends the declarative types $A$ with \emph{existential variables} $\al, \bt$.
They behave like unification variables,
but their scope is restricted by their position in the
algorithmic worklist rather than being global.
Any existential variable $\al$ should only be solved to
a type that is well-formed with respect to the worklist to which $\al$ has been added.
The point is that the monotype $\tau$, represented by the corresponding existential variable,
is always well-formed under the corresponding declarative context.
In other words, the position of $\al$'s reflects the well-formedness restriction of $\tau$'s.

%% A \le B \mid e\Lto A \mid e\To_{a} \jg \mid \appInfAlg{A}{e}

\paragraph{Judgment Chains} 
Judgment chains $\jg$, or judgments for short, are the core components of our algorithmic
type-checking. There are four kinds
of judgments in our system: subtyping ($A \le B$), checking ($e\Lto
A$), inference ($e\To_{a} \jg$) and
application inference ($\appInfAlg{A}{e}$).  Subtyping and checking are relatively simple,
since their result is only success or failure. However both inference and
application inference return a type that is used in subsequent judgments. We use a
continuation-passing-style encoding to accomplish this. For example, the judgment
chain $e \To_a (a \le B)$ contains two judgments: first we want to
infer the type of the expression $e$, and then check if that type is a
subtype of $B$. The \emph{unknown} type of $e$ is represented by a
type variable $a$, which is used as a placeholder in the second judgment to denote the 
type of $e$.

\paragraph{Worklist Judgments} Our algorithm has a non-standard form.
We combine traditional contexts and judgment(s) into a single sort, the \emph{worklist} $\Gm$.
The worklist is an \emph{ordered} collection of both variable bindings and judgments. The order captures the scope:
only the objects that come after a variable's binding in the worklist can refer to it.
For example, $[\nil, a, x:a \Vdash x \Lto a]$ is a valid worklist,
but $[\nil \Vdash \underline{x} \Lto \underline{a}, x:\underline{a}, a]$ is not
(the underlined symbols refer to out-of-scope variables).

\paragraph{Hole Notation}
We use the syntax $\Gm[\Gm_M]$ to denote the worklist $\Gm_L,\Gm_M,\Gm_R$,
where $\Gm$ is the worklist $\Gm_L,\bullet,\Gm_R$ with a hole ($\bullet$).
Hole notations with the same name implicitly share the same structure $\Gm_L$ and $\Gm_R$.
A multi-hole notation splits the worklist into more parts.
For example, $\Gm[\al][\bt]$ means $\Gm_1,\al,\Gm_2,\bt,\Gm_3$.

\subsection{Algorithmic System}

\newcounter{algRuleCounter}
\newcommand \algrule {\stepcounter{algRuleCounter}\rrule{\arabic{algRuleCounter}}}

% TODO rules too long
\begin{figure}[htp]
\hfill \framebox{$\Gm\rto \Gm'$} \hfill $\Gm$ reduces to $\Gm'$.
%\jimmy{Whenever there is a change on judgment count, marker should be properly added/removed. Decidability: all valid judgment chain should be reduced to nil.}
\begin{gather*}
\begin{aligned}
\Gm, a &\algrule \Gm \qquad
\Gm, \al \algrule \Gm \qquad
\Gm, x:A \algrule \Gm
\\[3mm]
\Gm \Vdash 1\le 1 &\algrule \Gm\\
\Gm \Vdash a\le a &\algrule \Gm\\
\Gm \Vdash \al\le \al &\algrule \Gm\\
\Gm \Vdash A_1\to A_2 \le B_1\to B_2 &\algrule \Gm \Vdash A_2 \le B_2 \Vdash B_1\le A_1\\
\Gm \Vdash \all A\le B &\algrule \Gm,\al \Vdash [\al/a]A\le B \quad\text{when } B \neq \all B'\\
\Gm \Vdash A\le \all[b]B &\algrule \Gm,b \Vdash A\le B
\\[3mm]
%\\
%\text{Let } \color{red}\Gm[\al \toto B // G_M] &:= \Gm_L, G_M, [B/\al]\Gm_R, \text{ when } \Gm[\al] = \Gm_L,\al,\Gm_R \ (G_M\text{ defaults to }\nil)\\
\Gm[\al] \Vdash \al \le A\to B &\algrule [\al[1]\to\al[2]/\al] (\Gm[\al[1], \al[2]] \Vdash \al[1]\to \al[2] \le A \to B)\\
 &\qquad\qquad \text{when }\al\notin FV(A)\cup FV(B)\\
\Gm[\al] \Vdash A\to B \le \al &\algrule [\al[1]\to \al[2]/\al] (\Gm[\al[1], \al[2]] \Vdash A \to B \le \al[1]\to \al[2])\\
 &\qquad\qquad \text{when }\al\notin FV(A)\cup FV(B)
 \\[3mm]
\Gm[\al][\bt] \Vdash \al \le \bt &\algrule [\al/\bt](\Gm[\al][])\\
\Gm[\al][\bt] \Vdash \bt \le \al &\algrule [\al/\bt](\Gm[\al][])\\
\Gm[a][\bt] \Vdash a \le \bt &\algrule [a/\bt](\Gm[a][])\\
\Gm[a][\bt] \Vdash \bt \le a &\algrule [a/\bt](\Gm[a][])\\
\Gm[\bt] \Vdash 1 \le \bt &\algrule [1/\bt](\Gm[])\\
\Gm[\bt] \Vdash \bt \le 1 &\algrule [1/\bt](\Gm[])
\\[3mm]
\Gm \Vdash e \Lto B &\algrule \Gm \Vdash e\To_a a\le B \quad
    \text{when } e \neq \lam e' \text{ and } B \neq \all B'\\
% \Gm \Vdash () \Lto 1 &\algrule \Gm\\
\Gm \Vdash e\Lto \all A &\algrule \Gm,a \Vdash e\Lto A\\
\Gm \Vdash \lam e \Lto A\to B &\algrule \Gm, x:A  \Vdash e \Lto B\\
\Gm[\al] \Vdash \lam e \Lto \al &\algrule [\al[1]\to \al[2] / \al](\Gm[\al[1],\al[2]], x:\al[1] \Vdash e \Lto \al[2])
% \quad\text{\jimmy{Additional}}
\\[3mm]
\Gm \Vdash x\To_a \jg &\algrule \Gm \Vdash [A/a] \jg \quad \text{when } (x:A)\in \Gm\\
\Gm \Vdash (e:A)\To_a \jg &\algrule \Gm \Vdash [A/a]\jg \Vdash e \Lto A\\
\Gm \Vdash ()\To_a \jg &\algrule \Gm \Vdash [1/a]\jg\\
\Gm \Vdash \lam e \To_a \jg &\algrule
    \Gm,\al,\bt \Vdash [\al\to\bt/a]\jg, x:\al \Vdash e\Lto \bt\\
\Gm \Vdash e_1\ e_2 \To_a \jg &\algrule \Gm \Vdash e_1\To_b (\appInfAlg{b}{e_2})
\\[3mm]
\Gm \Vdash \appInfAlg{\all A}{e} &\algrule \Gm,\al \Vdash \appInfAlg{[\al/a]A}{e}\\
\Gm \Vdash \appInfAlg{A\to C}{e} &\algrule \Gm \Vdash [C/a]\jg \Vdash e \Lto A\\
\Gm[\al] \Vdash \appInfAlg{\al}{e} &\algrule
    [\al[1]\to\al[2]/\al](\Gm[\al[1], \al[2]] \Vdash \appInfAlg{\al[1]\to\al[2]}{e})
%	[\al[1]\to\al[2]/\al](\Gm[\al[1], \al[2]] \Vdash [\al[2]/a]\jg \Vdash e\Lto \al[1])\\
% &\color{magenta} \makebox[0pt]{\qquad or} \phantom{{}\rrule{}{}}
\end{aligned}
\end{gather*}
\Description{Algorithmic Typing}
\caption{Algorithmic Typing}\label{fig:alg}
\end{figure}

The algorithmic typing reduction rules, defined in Figure~\ref{fig:alg}, have
the form $\Gm \rto \Gm'$.
% The worklists $\Gm, \Gm'$ contain both variable declarations and judgments.
The reduction process treats the worklist as a stack.
In every step, it pops the first judgment
from the worklist for processing and possibly pushes new
judgments onto the worklist.  The syntax $\Gm\redto \Gm'$ denotes multiple
reduction steps. 

$$\inferrule*[right=$\mathtt{{\redto} id}$]
{~}{\Gm\redto\Gm}
\qquad
\inferrule*[right=$\mathtt{{\redto} step}$]
{\Gm \rto \Gm_1 \quad \Gm_1 \redto \Gm'}{\Gm \redto \Gm'}$$

\noindent In the case that $\Gm\redto\nil$ this corresponds to successful type checking.

% \jimmy{The implicit freshness conditions for algorithmic system}
Please note that when a new variable is introduced in the right-hand side worklist $\Gm'$,
we implicitly pick a fresh one,
since the right-hand side can be seen as the premise of the reduction.

Rules 1-3 pop variable declarations that are essentially garbage.
Those variables are out of scope
for the remaining judgments in the worklist.
All other rules concern a judgment at the front of the worklist.
Logically we can discern 6 groups of rules.

\paragraph{{\bf 1. Algorithmic subtyping}}
We have six subtyping rules (Rules 4-9) that are similar to their
declarative counterparts. For instance, Rule 7 consumes a subtyping
judgment and pushes two back to the worklist.  Rule 8 differs from
declarative Rule $\mathtt{{\le}{\forall}L}$ by introducing an existential
variable $\al$ instead of guessing the monotype $\tau$
instantiation. Each existential variable is later solved to a
monotype $\tau$ with the same context, so the final solution $\tau$ of
$\al$ should be well-formed under $\Gm$.

\paragraph{Worklist Variable Scoping}
Rules 8 and 9 involve variable declarations and demonstrate how our
worklist treats variable scopes. Rule 8 introduces an existential
variable $\al$ that is only visible to the judgment $[\al/a]A \le B$.
Reduction continues until all the subtyping judgments in front of $\al$
are satisfied.  Finally we can safely remove $\al$ since no occurrence
of $\al$ could have leaked into the left part of the worklist.  Moreover,
the algorithm garbage-collects the $\al$
variable at the right time: it leaves the environment immediately
after being unreferenced completely for sure.

\paragraph{Example} Consider the derivation of the subtyping judgment
$(1\to 1)\to 1 \le (\all 1\to 1) \to 1$:
\begin{gather*}
\begin{aligned}
  & \nil\vdash (1\to 1)\to 1 \le (\all 1 \to 1) \to 1\\
\rrule{7}  & \nil \Vdash 1 \le 1 \Vdash \all 1 \to 1 \le 1 \to 1\\
\rrule{8}  & \nil \Vdash 1 \le 1 ,\al \Vdash 1 \to 1 \le 1 \to 1\\
\rrule{7}  & \nil \Vdash 1 \le 1 ,\al \Vdash 1 \le 1 \Vdash 1 \le 1\\
\rrule{4} & \nil \Vdash 1 \le 1 ,\al \Vdash 1 \le 1\\
\rrule{4} & \nil \Vdash 1 \le 1 ,\al\\
\rrule{2} & \nil \Vdash 1 \le 1\\
\rrule{4} & \nil
\end{aligned}
\end{gather*}
First, the subtyping of two function types is split into two judgments by Rule 7:
a covariant subtyping on the return type and a contravariant subtyping on the argument type.
Then we apply Rule 8 to reduce the $\forall$ quantifier on the left side.
The rule introduces an existential variable $\al$ to the context (even though 
the type $\all 1 \to 1$ does not actually refer to the quantified type
variable $a$).
In the following 3 steps we satisfy the judgment $1 \to 1 \le 1 \to 1$
by Rules 7, 4, and 4 (again).

Now the existential variable $\al$, introduced before but still unsolved,
is at the top of the worklist and Rule 2 garbage-collects it.
The process is carefully designed within the algorithmic rules:
when $\al$ is introduced earlier by Rule 8,
we foresee the recycling of $\al$ after all the judgments (potentially)
requiring $\al$ have been processed.
Finally Rule 4 reduces one of the base cases and finishes the subtyping derivation.

\paragraph{\bf 2. Existential decomposition.}
Rules 10 and 11 are algorithmic versions of Rule $\mathtt{{\le}{\to}}$; they
both partially instantiate $\al$ to function types.
The domain $\al[1]$ and range $\al[2]$ of
the new function type are not determined:
they are fresh existential variables with the same scope as $\al$.
We replace $\al$ in the worklist with  $\al[1], \al[2]$.
To propagate the instantiation to the rest of the worklist and maintain well-formedness,
every reference to $\al$ is replaced by $\al[1] \to \al[2]$.
The \emph{occurs-check} condition prevents divergence as usual.
For example, without it $\al \le 1 \to \al$ would diverge.

\paragraph{\bf 3. Solving existentials} Rules 12-17 are base cases where an existential variable is solved.
They all remove an existential variable and substitute the
variable with its solution in the remaining worklist. Importantly the rules
respect the scope of existential variables. For example, Rule 12 
states that an existential variable $\al$ can be solved with another
existential variable $\bt$ only if $\bt$ occurs after $\al$.

One may notice that the subtyping relation for simple types is just equivalence,
which is true according to the declarative system.
The DK's system works in a similar way.

\paragraph{\bf 4. Checking judgments.}
Rules 18-21 deal with checking judgments.
Rule 18 is similar to $\mathtt{DeclSub}$, but rewritten in the
continuation-passing-style.
The side conditions $e \neq \lam e'$ and $B \neq \all B'$ 
prevent overlap with Rules 19, 20 and 21;
this is further discussed at the end of this section.
% \bruno{The side condition needs to be explained, and some mention of
%   the overlapping is probably needed here.}
Rules 19 and 20 adapt their declarative counterparts to the worklist style.
Rule 21 is a special case of $\mathtt{Decl\to I}$,
dealing with the case when the input type is an existential variable,
representing a monotype \emph{function} as in the declarative system
(it must be a function type, since the expression $\lam e$ is a function).
% \bruno{Rule 21
%   deserves some more explanation. What is the motivation for this
%   rule? You may want to give some concrete example to motivate it.}
The same instantiation technique as in Rules 10 and 11 applies.
The declarative checking rule $\mathtt{Decl1I}$ does not have a direct counterpart in the algorithm, 
because Rules 18 and 24 can be combined to give the same result.

\paragraph{Rule 21 Design Choice}
The addition of Rule 21 is a design choice we have made to simplify the side
condition of Rule 18 (which avoids overlap). It also streamlines the algorithm
and the metatheory as we now treat all cases
where we can see that an existential variable should be instantiated to a
function type  (i.e., Rules 10, 11, 21 and 29) uniformly.

The alternative would have been to omit Rule 21 and drop the condition on $e$
in Rule 18. The modified Rule 18 would then handle $\Gm \Vdash \lam e \Lto \al$
and yield $\Gm \Vdash \lam e \To_a a \le \al$, which would be further processed
by Rule 25 to $\Gm, \bt[1], \bt[2] \Vdash \bt[1] \to \bt[2] \le \al, x:\bt[1] \Vdash e \Lto \bt[2]$.
As a subtyping constraint between monotypes is simply equality, $\bt[1] \to \bt[2] \le \al$
must end up equating $\bt[1] \to \bt[2]$ with $\al$ and thus have the same effect as Rule 21, but in a more roundabout
fashion.

In comparison,
DK's algorithmic subsumption rule has no restriction on the expression $e$,
and they do not have a rule that explicitly handles the case $\lam e \Lto \al$.
Therefore the only way to check a lambda function against an existential variable
is by applying the subsumption rule, which further breaks into
type inference of a lambda function and a subtyping judgment.
% A combination of Rules $18'$ (Rule 18 without the restriction on $e$) and 25 and further derivations
% may result in a similar effect to Rule 21.
% Rule 21 guesses a monotype for the lambda function $\lam e$, so as Rule 25.
% The only difference is that a subtyping relation is introduced by Rule $18'$,
% and is to be processed after the type inference:
% $$\Gm \Vdash \lam e \Lto \al \rrule{18'} \Gm \Vdash \lam e \To_a a \le \al
% \rrule{25} \Gm, \bt[1], \bt[2] \Vdash \bt[1] \to \bt[2] \le \al, x:\bt[1] \Vdash e \Lto \bt[2]$$
% Since two monotypes with a valid subtyping relation indicates equality,
% eventually $\bt[1] \to \bt[2]$ and $\al$ should be unified with each other.
% DK's algorithm takes a similar approach to such variant,
% and we believe that they are equivalent.
% 
% Choosing Rule 21 simplifies the side condition on Rule 18,
% the metatheory and the design: whenever an input type could be a function type,
% the algorithmic rule should always take care of
% the possibility when the type being a single existential variable.
% This pattern applies to all the possible cases, including Rules 10, 11, 21 and 29.

\paragraph{\bf 5. Inference judgments.}
Inference judgments behave differently compared with subtyping and checking judgments:
they \emph{return} a type instead of only accepting or rejecting.
For the algorithmic system, where guesses are involved,
it may happen that the output type of an inference judgment refers to new existential variables,
such as Rule 25.
In comparison to Rule 8 and 9, where new variables are
only referred to by the sub-derivation,
Rule 25 introduces variables $\al, \bt$ that affect the remaining judgment chain.
This rule is carefully designed so that the output variables are bound by earlier declarations,
thus the well-formedness of the worklist is preserved,
and the garbage will be collected at the correct time.
By making use of the continuation-passing-style judgment chain,
inner judgments always share the context with their parent judgment.

\begin{comment}
Old text:
The design of our judgment chain is closely related to the shape of
the judgments
\jimmy{requires further clarification}
\bruno{What property? Not very clear}.
Subtyping and checking do not return anything, so variables cannot
leak anyway, as applied to Rules 8 and 9.
\bruno{Is the discussion that follows in the right place? We just jump
  to
  rule 26. Perhaps we can wait until we talk about inference to
  discuss those issues?}
However, inference and application inference may return a type that contains new variables.
Take Rule 26 as an example, if it simply returns $\al \to \bt$,
passes that to the next judgment and continues the type-checking process,
variables $\al$ and $\bt$ in the next judgment are out of scope
and will break the well-formedness of the worklist.
\end{comment}

Rules 22-26 deal with type inference judgments, written in continuation-passing-style.
When an inference judgment succeeds with type $A$,
the algorithm continues to work on the inner-chain $\jg$ by
assigning $A$ to its placeholder variable $a$.
Rule 23 infers an annotated expression by changing into checking mode,
therefore another judgment chain is created.
Rule 24 is a base case,
where the unit type $1$ is inferred and thus passed to its child judgment chain.
Rule 26 infers the type of an application by
firstly inferring the type of the function $e_1$,
and then leaving the rest work to an application inference judgment,
which passes $a$, representing the return type of the application,
to the remainder of the judgment chain $\jg$.

Rule 25 infers the type of a lambda expression by introducing $\al, \bt$
as the input and output types of the function, respectively.
After checking the body $e$ under the assumption $x:\al$,
the return type might reflect more information than simply $\al \to \bt$
through propagation when existential variables are solved or partially solved.
The variable scopes are maintained during the process:
the assumption of argument type ($x:\al$) is recycled after checking against the function body;
the existential variables used by the function type ($\al,\bt$) are only visible in the remaining chain $[\al\to\bt/a]\jg$.
The recycling process of Rule 25 differs from DK's corresponding rule significantly,
and we further discuss the differences in Section~\ref{sec:discussion:scoping}.

% \bruno{You have some note ``differs'' on the rules. Should that
%   deserve an explanation here?}

\paragraph{\bf 6. Application inference judgments}
Finally, Rules 27-29 deal with application inference judgments.
Rules 27 and 28 behave similarly to declarative rules $\mathtt{Decl\forall App}$ and $\mathtt{Decl\to App}$.
Rule 29 instantiates $\al$ to the function type $\al[1] \to \al[2]$, just like Rules 10, 11 and 21.

\paragraph{Example}
Figure~\ref{fig:alg:sample} shows a sample derivation of the algorithm.
It checks the application $(\lam x)~()$ against the unit type.
According to the algorithm, we apply Rule 18 (subsumption), changing to inference mode.
Type inference of the application breaks into two steps by Rule 26:
first we infer the type of the function,
and then the application inference judgment helps to determine the return type.
In the following 5 steps the type of the identity function, $\lam x$, is inferred to be $\al \to \al$:
checking the body of the lambda function (Rule 25),
switching from check mode to inference mode (Rule 18),
inferring the type of a term variable (Rule 22),
solving a subtyping judgment between existential variables (Rule 12) and
garbage collecting the term variable $x$ (Rule 3).

After that, Rule 28 changes the application inference judgment to
a check of the argument against the input type $\al$ and returns the output type $\al$.
Checking $()$ against the existential variable $\al$ solves $\al$ to the unit type $1$
through Rules 18, 24 and 16.
Immediately after $\al$ is solved, the algorithm replaces every occurrence of $\al$ with $1$.
Therefore the worklist remains $1 \le 1$, which is finished off by Rule 4.
Finally, the empty worklist indicates the success of the whole derivation.

\begin{figure}
\begin{gather*}
\begin{aligned}
           & \nil \Vdash (\lam x)~() \Lto 1\\
\rrule{18} & \nil \Vdash (\lam x)~() \To_a a\le 1\\
\rrule{26} & \nil \Vdash (\lam x) \To_b (\appInfAlg{b}{()}[a][a\le 1])\\
\rrule{25} & \nil,\al,\bt \Vdash \appInfAlg{\al\to\bt}{()}[a][a\le 1], x:\al \Vdash x\Lto \bt\\
\rrule{18} & \nil,\al,\bt \Vdash \appInfAlg{\al\to\bt}{()}[a][a\le 1], x:\al \Vdash x\To_b b\le \bt\\
\rrule{22} & \nil,\al,\bt \Vdash \appInfAlg{\al\to\bt}{()}[a][a\le 1], x:\al \Vdash \al\le \bt\\
\rrule{12} & \nil,\al \Vdash \appInfAlg{\al\to\al}{()}[a][a\le 1], x:\al\\
\rrule{3}  & \nil,\al \Vdash \appInfAlg{\al\to\al}{()}[a][a\le 1]\\
\rrule{28} & \nil,\al \Vdash \al\le 1 \Vdash () \Lto \al\\
\rrule{18} & \nil,\al \Vdash \al\le 1 \Vdash () \To_a a\le \al\\
\rrule{24} & \nil,\al \Vdash \al\le 1 \Vdash 1\le \al\\
\rrule{16} & \nil \Vdash 1\le 1\\
\rrule{4}  & \nil
\end{aligned}
\end{gather*}
\Description{A Sample Derivation for Algorithmic Typing}
\caption{A Sample Derivation for Algorithmic Typing}
\label{fig:alg:sample}
\end{figure}

In summary, our type checking algorithm accepts $(\lam x)~() \Lto 1$.

\paragraph{Non-overlapping and Deterministic Reduction}
An important feature of our algorithmic rules is that they are directly
implementable. Indeed, although written in the form of reduction rules, they do
not overlap and are thus deterministic.
%This has the added benefit that our % INCORRECT!
%algorithm runs in polynomial time, which makes it suitable for practical use.

Consider in particular Rules 8 and 9, which correspond to the declarative rules
$\mathtt{{\le}\forall L}$ and $\mathtt{{\le}\forall R}$. While those
declarative rules both match the goal $\all A\le \all[b]B$,
we have eliminated this overlap in the algorithm by restricting Rule 8
($B\neq\all B'$) and thus always applying Rule 9 to $\all A\le \all[b]B$.

Similarly, the declarative rule $\mathtt{DeclSub}$ overlaps highly with the
other checking rules. Its algorithmic counterpart is Rule 18. Yet, we have
avoided the overlap with other algorithmic checking rules by adding
side-conditions to Rule 18, namely $e\neq\lam e'$ and $B\neq\all B'$.

These restrictions have not been imposed arbitrarily:
we formally prove that the restricted algorithm is still complete.
In Section~\ref{sec:metatheory:non-overlapping} we discuss the relevant metatheory,
with the help of a non-overlapping version of the declarative system.

