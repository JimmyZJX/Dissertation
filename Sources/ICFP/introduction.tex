\section{Introduction}



This paper presents the first fully mechanized formalization of the
metatheory for higher-ranked polymorphic type inference.
The system
that we formalize is the bidirectional type system by \citet{dunfield2013complete}.
We chose DK's type system because it is
quite elegant, well-documented and it comes with detailed manually
written proofs. Furthermore the system is adopted in practice by a few
real implementations of functional languages, including PureScript and
Unison. The DK type system has two variants: a declarative
and an algorithmic one. The two variants have been
\emph{manually} proved to be \emph{sound}, \emph{complete} and
\emph{decidable}.
We present a mechanical formalization in the Abella theorem prover~\cite{AbellaDesc} for
DK's declarative type system using a different algorithm. While our
initial goal was to formalize both DK's declarative and algorithmic
versions, we faced technical challenges with the latter, prompting us to find
an alternative formulation.

The first challenge that we faced were missing details as well as
a few incorrect proofs and lemmas in DK's formalization. While DK's
original formalization comes with very well written manual proofs,
there are still several details missing. These complicate the task of
writing a mechanically verified proof. Moreover some proofs and
lemmas are wrong and, in some cases, it is not clear to us how to fix them.


Despite the problems in DK's manual formalization,
we believe that these problems do not
invalidate their work and that their results are still true. In fact we have nothing but praise for their detailed
and clearly written metatheory and proofs, which provided invaluable
help to our own work.
We expect that for most non-trivial manual
proofs similar problems exist, so this should not be understood as a sign of sloppiness
on their part. Instead it should be an indicator that reinforces the arguments
for mechanical formalizations: manual formalizations are error-prone due to the multiple
tedious details involved in them.
There are several other examples of manual formalizations that were found to have
similar problems. For example, Klein et al.~\cite{KleinRunYourResearch}
mechanized formalizations
in Redex for nine ICFP 2009 papers and all were found to have mistakes.

Another challenge was variable binding. Type inference algorithms
typically do not rely simply on local environments but instead
propagate information across judgments. While local environments are
well-studied in mechanical formalizations, there is little work on how
to deal with the complex forms of binding employed by type inference algorithms
in theorem provers. To
keep track of variable scoping, DK's algorithmic version employs input
and output contexts to track information that is discovered through
type inference. However modeling output contexts in a theorem prover
is non-trivial.

Due to those two challenges, our work takes a different approach by refining and
extending the idea of \emph{worklist judgments}~\cite{itp2018},
proposed recently to mechanically formalize an algorithm for
\emph{polymorphic subtyping}~\cite{odersky1996putting}. A key innovation in our work is how
to adapt the idea of worklist judgments to
\emph{inference judgments}, which are not needed for polymorphic
subtyping, but are necessary for type-inference.  The idea is to use a \emph{continuation
passing style} to enable the transfer of inferred information across
judgments. A further refinement to the idea of worklist judgments is
the \emph{unification between ordered
  contexts~\cite{gundry2010type,dunfield2013complete} and worklists}.  This
enables precise scope tracking of free variables in
judgments. Furthermore it avoids the duplication of context
information across judgments in worklists that occurs in other
techniques~\cite{Reed2009,Abel2011higher}.
Despite the use of a different algorithm we prove the
same results as DK, although with significantly different proofs and
proof techniques. The calculus and its metatheory
have been fully formalized in the Abella theorem prover~\cite{AbellaDesc}.

In summary, the contributions of this paper are:

\begin{itemize}

\item {\bf A fully mechanized formalization of type inference with
  higher-ranked types:} Our work presents the first fully mechanized formalization
  for type inference of higher ranked types. The formalization is done in the
  Abella theorem prover~\cite{AbellaDesc} and it is available
  online at \url{https://github.com/JimmyZJX/TypingFormalization}.

\item {\bf A new algorithm for DK's type system:} Our work proposes a novel algorithm that implements
  DK's declarative bidirectional type system. We prove
  \emph{soundness}, \emph{completeness} and
  \emph{decidability}. 

\item {\bf Worklists with inference judgments:} One technical contribution is the
  support for inference judgments using worklists. The idea is to
  use a continuation passing style to enable the transfer of inferred information across
  judgments. 

\item {\bf Unification of worklists and contexts:} Another technical contribution is the unification
  between ordered contexts and worklists. This enables precise scope tracking
  of variables in judgments, and avoids the duplication of context information across
  judgments in worklists.

\begin{comment}
\jimmy{Notes @20190211 4 points of novalty:\\
1) Dealing with inference judgments and CPS-style chains\\
2) The form of the judgment itself with a single shared context\\
3) The way we deal with scope (which may follow from 2)\\
4) Immediate substitution (judgment list)
}
\end{comment}

\end{itemize}
