%% For double-blind review submission, w/o CCS and ACM Reference (max
%% submission space)
%\documentclass[acmsmall,review]{acmart}\settopmatter{printfolios=true}
%%\documentclass[acmsmall,review,anonymous]{acmart}\settopmatter{printfolios=true,printccs=false,printacmref=false}
%% For double-blind review submission, w/ CCS and ACM Reference
%\documentclass[acmsmall,review,anonymous]{acmart}\settopmatter{printfolios=true}
%% For single-blind review submission, w/o CCS and ACM Reference (max submission space)
%\documentclass[acmsmall,review]{acmart}\settopmatter{printfolios=true,printccs=false,printacmref=false}
%% For single-blind review submission, w/ CCS and ACM Reference
%\documentclass[acmsmall,review]{acmart}\settopmatter{printfolios=true}
%% For final camera-ready submission, w/ required CCS and ACM Reference
\documentclass[acmsmall,screen]{acmart}

%% Remove almost every footnotes and page numbers, for grammar check use.
%\documentclass[acmsmall,anonymous]{acmart}\settopmatter{printfolios=true,printccs=false,printacmref=false}
%\settopmatter{printacmref=false} % Removes citation information below abstract
%\renewcommand\footnotetextcopyrightpermission[1]{} % removes footnote with conference information in first column
%\pagestyle{plain} % removes running headers
%\pagenumbering{gobble}


%% Journal information
%% Supplied to authors by publisher for camera-ready submission;
%% use defaults for review submission.
%\acmJournal{PACMPL}
%\acmVolume{1}
%\acmNumber{ICFP} % CONF = POPL or ICFP or OOPSLA
%\acmArticle{1}
%\acmYear{2018}
%\acmMonth{1}
%\acmDOI{} % \acmDOI{10.1145/nnnnnnn.nnnnnnn}
%\startPage{1}

%% Copyright information
%% Supplied to authors (based on authors' rights management selection;
%% see authors.acm.org) by publisher for camera-ready submission;
%% use 'none' for review submission.
%\setcopyright{none}
%\setcopyright{acmcopyright}
%\setcopyright{acmlicensed}
%\setcopyright{rightsretained}
%\copyrightyear{2018}           %% If different from \acmYear

%% Bibliography style
\bibliographystyle{ACM-Reference-Format}
%% Citation style
%% Note: author/year citations are required for papers published as an
%% issue of PACMPL.
\citestyle{acmauthoryear}   %% For author/year citations


%%%%%%%%%%%%%%%%%%%%%%%%%%%%%%%%%%%%%%%%%%%%%%%%%%%%%%%%%%%%%%%%%%%%%%
%% Note: Authors migrating a paper from PACMPL format to traditional
%% SIGPLAN proceedings format must update the '\documentclass' and
%% topmatter commands above; see 'acmart-sigplanproc-template.tex'.
%%%%%%%%%%%%%%%%%%%%%%%%%%%%%%%%%%%%%%%%%%%%%%%%%%%%%%%%%%%%%%%%%%%%%%


%% Some recommended packages.
\usepackage{booktabs}   %% For formal tables:
                        %% http://ctan.org/pkg/booktabs
\usepackage{subcaption} %% For complex figures with subfigures/subcaptions
                        %% http://ctan.org/pkg/subcaption

\usepackage{amsmath}
\usepackage{amssymb}
\usepackage{amstext}
\usepackage{enumerate}
%\usepackage{enumitem}

\usepackage{multirow}
\usepackage{makecell}
\usepackage{color}

\usepackage{xparse}
\usepackage{listings}
\usepackage{lstabella}

\usepackage{mathpartir}
\usepackage{comment}

\usepackage{mathtools}
\usepackage{graphicx}

\usepackage{lscape}
\usepackage{afterpage}
% \usepackage{rotating}

\usepackage{booktabs}

%\usepackage{tikz}


\newcommand\mynote[3]{\textcolor{#2}{#1: #3}}
\newcommand\bruno[1]{\mynote{Bruno}{red}{#1}}
\newcommand\jimmy[1]{\mynote{Jimmy}{blue}{#1}}
\newcommand\tom[1]{\mynote{Tom}{magenta}{#1}}

\newcommand\wh{\widehat}

\newcommand\jg{\omega}                                 % judgment
\newcommand\toto{\rightrightarrows}
\newcommand\To{\Rightarrow}                            % =>
\newcommand\Lto{\Leftarrow}                            % <=
\newcommand\TTo{\mathrel{\mathrlap{\To}\phantom{~}\To}}  % =>>
\newcommand\sto{\rightsquigarrow}
\newcommand\tto{\rightarrowtail}

\newcommand\Gm{\Gamma}
\newcommand\Om{\Omega}

\newcommand\nil\cdot

% just to fool TexStudio
\providecommand\inferrule{}

\newcommand\rto{\longrightarrow}                % "reduce to" arrow ---`
\newcommand\redto{\longrightarrow^*}            % ---`*
\newcommand\rrule[1]{~\longrightarrow_{\makebox[0pt][l]{$\scriptstyle#1$}\hphantom{00}}}

\newcommand\jExt{\rightharpoonup}

\newcommand{\tRed}[1]{\textcolor{red}{#1}}

\newcommand{\lam}{} %[2]{\lambda {#1}.~{#2}}
\RenewDocumentCommand \lam {O{x} m} {\lambda {#1}.~{#2}}  % \lambda x.~e --- x is optional. \lam[x]{e}

\newcommand{\all}{} %[2]{\forall {#1}.~{#2}}
\RenewDocumentCommand \all {O{a} m} {\forall {#1}.~{#2}}  % \forall a.~A --- a is optional. \all[a]{A}

\newcommand{\appInf}[3]{{#1}\bullet{#2}\TTo{#3}}
\newcommand{\appInfAlg}{} %[4]{{#1}\bullet{#2}\TTo_{#3} {#4}}
\RenewDocumentCommand \appInfAlg {m m O{a} O{\jg}} {{#1}\bullet{#2}\TTo_{#3} {#4}}

\newcommand\al{}
\newcommand\bt{}

\RenewDocumentCommand \al {O{}} {\wh\alpha_{#1}}   % \widehat{\alpha} --- subscript is optional \al[1]
\RenewDocumentCommand \bt {O{}} {\wh\beta_{#1}}    % \widehat{\beta}

\newcommand{\blue}[1]{\textcolor{blue}{#1}}



% https://tex.stackexchange.com/questions/346870/bf-is-an-undefined-command
\DeclareOldFontCommand{\bf}{\normalfont\bfseries}{\mathbf}

\newcommand{\Description}[1]{}

% ITP used
\newcommand\exps{\Omega}
\newcommand\jcons[2]{#1 ; #2}




%%% The following is specific to ICFP '19 and the paper
%%% 'A Mechanical Formalization of Higher-Ranked Polymorphic Type Inference'
%%% by Jinxu Zhao, Bruno C. d. S. Oliveira, and Tom Schrijvers.
%%%
\setcopyright{acmcopyright}
\acmPrice{}
\acmDOI{10.1145/3341716}
\acmYear{2019}
\copyrightyear{2019}
\acmJournal{PACMPL}
\acmVolume{3}
\acmNumber{ICFP}
\acmArticle{112}
\acmMonth{8}


\begin{document}

%% Title information
\title{A Mechanical Formalization of Higher-Ranked Polymorphic
  Type Inference}         %% [Short Title] is optional;
                                        %% when present, will be used in
                                        %% header instead of Full Title.
%\titlenote{with title note}             %% \titlenote is optional;
                                        %% can be repeated if necessary;
                                        %% contents suppressed with 'anonymous'
%\subtitle{Subtitle}                     %% \subtitle is optional
%\subtitlenote{with subtitle note}       %% \subtitlenote is optional;
                                        %% can be repeated if necessary;
                                        %% contents suppressed with 'anonymous'


%% Author information
%% Contents and number of authors suppressed with 'anonymous'.
%% Each author should be introduced by \author, followed by
%% \authornote (optional), \orcid (optional), \affiliation, and
%% \email.
%% An author may have multiple affiliations and/or emails; repeat the
%% appropriate command.
%% Many elements are not rendered, but should be provided for metadata
%% extraction tools.

%% Author with single affiliation.
\author{Jinxu Zhao}
%\authornote{with author1 note}          %% \authornote is optional;
                                        %% can be repeated if necessary
%\orcid{nnnn-nnnn-nnnn-nnnn}             %% \orcid is optional
\affiliation{
  %\position{Position1}
  \department{Department of Computer Science}              %% \department is recommended
  \institution{The University of Hong Kong}            %% \institution is required
  %\streetaddress{Pokfulam}
  \city{Hong Kong}
  %\state{State1}
  %\postcode{Post-Code1}
  \country{China}                    %% \country is recommended
}
\email{jxzhao@cs.hku.hk}          %% \email is recommended

%% Author with two affiliations and emails.
\author{Bruno C. d. S. Oliveira}
%\authornote{with author1 note}          %% \authornote is optional;
%% can be repeated if necessary
%\orcid{nnnn-nnnn-nnnn-nnnn}             %% \orcid is optional
\affiliation{
	%\position{Position1}
	\department{Department of Computer Science}              %% \department is recommended
	\institution{The University of Hong Kong}            %% \institution is required
	%\streetaddress{Pokfulam}
	\city{Hong Kong}
	%\state{State1}
	%\postcode{Post-Code1}
	\country{China}                    %% \country is recommended
}
\email{bruno@cs.hku.hk}          %% \email is recommended

%% Author with two affiliations and emails.
\author{Tom Schrijvers}
%\authornote{with author2 note}          %% \authornote is optional;
%% can be repeated if necessary
%\orcid{nnnn-nnnn-nnnn-nnnn}             %% \orcid is optional
\affiliation{
	%\position{Position2a}
	\department{Department of Computer Science}             %% \department is recommended
	\institution{KU Leuven}           %% \institution is required
	%\streetaddress{Street2a Address2a}
	\city{Leuven}
	%\state{State2a}
	%\postcode{Post-Code2a}
	\country{Belgium}                   %% \country is recommended
}
\email{tom.schrijvers@cs.kuleuven.be}


%% Abstract
%% Note: \begin{abstract}...\end{abstract} environment must come
%% before \maketitle command
\begin{abstract}
Modern functional programming languages, such as Haskell or OCaml,
use sophisticated forms of type inference. While an
important topic in the Programming Languages research, there is little
work on the mechanization of the metatheory of type inference in
theorem provers. In particular we are unaware of any complete
formalization of the type inference algorithms that are the backbone of modern 
functional languages. 

This paper presents the first full mechanical formalization of the metatheory for
higher-ranked polymorphic type inference. The system that we formalize
is the bidirectional type system by Dunfield and
Krishnaswami (DK). The DK type system has two variants (a declarative
and an algorithmic one) that have been \emph{manually} proven
\emph{sound}, \emph{complete} and \emph{decidable}. We present 
a mechanical formalization in the Abella theorem prover 
of DK's declarative type system with a novel algorithmic system. We have
a few reasons to use a new algorithm. Firstly, our new algorithm
employs \emph{worklist judgments}, which precisely capture the
scope of variables and simplify the formalization of scoping in a
theorem prover. Secondly, while DK's original formalization comes with very
well-written manual proofs, there are several details missing and some
incorrect proofs,
which complicate the task of writing a mechanized proof. 
Despite the use of a different algorithm we prove the
same results as DK, although with significantly different proofs and
proof techniques. Since such type inference algorithms are quite
subtle and have a complex metatheory, mechanical
formalizations are an important advance in type-inference research.
\end{abstract}


%% 2012 ACM Computing Classification System (CSS) concepts
%% Generate at 'http://dl.acm.org/ccs/ccs.cfm'.
\begin{CCSXML}
<ccs2012>
<concept>
<concept_id>10011007.10011006.10011008.10011009.10011012</concept_id>
<concept_desc>Software and its engineering~Functional languages</concept_desc>
<concept_significance>500</concept_significance>
</concept>
<concept>
<concept_id>10011007.10011006.10011008.10011024.10011025</concept_id>
<concept_desc>Software and its engineering~Polymorphism</concept_desc>
<concept_significance>500</concept_significance>
</concept>
<concept>
<concept_id>10011007.10011006.10011039</concept_id>
<concept_desc>Software and its engineering~Formal language definitions</concept_desc>
<concept_significance>500</concept_significance>
</concept>
</ccs2012>
\end{CCSXML}

\ccsdesc[500]{Software and its engineering~Functional languages}
\ccsdesc[500]{Software and its engineering~Polymorphism}
\ccsdesc[500]{Software and its engineering~Formal language definitions}
%% End of generated code


%% Keywords
%% comma separated list
\keywords{type inference, higher-ranked polymorphism, mechanization}  %% \keywords are mandatory in final camera-ready submission


%% \maketitle
%% Note: \maketitle command must come after title commands, author
%% commands, abstract environment, Computing Classification System
%% environment and commands, and keywords command.
\maketitle

\section{Introduction}



This paper presents the first fully mechanized formalization of the
metatheory for higher-ranked polymorphic type inference.
The system
that we formalize is the bidirectional type system by \citet{dunfield2013complete}.
We chose DK's type system because it is
quite elegant, well-documented and it comes with detailed manually
written proofs. Furthermore the system is adopted in practice by a few
real implementations of functional languages, including PureScript and
Unison. The DK type system has two variants: a declarative
and an algorithmic one. The two variants have been
\emph{manually} proved to be \emph{sound}, \emph{complete} and
\emph{decidable}.
We present a mechanical formalization in the Abella theorem prover~\cite{AbellaDesc} for
DK's declarative type system using a different algorithm. While our
initial goal was to formalize both DK's declarative and algorithmic
versions, we faced technical challenges with the latter, prompting us to find
an alternative formulation.

The first challenge that we faced were missing details as well as
a few incorrect proofs and lemmas in DK's formalization. While DK's
original formalization comes with very well written manual proofs,
there are still several details missing. These complicate the task of
writing a mechanically verified proof. Moreover some proofs and
lemmas are wrong and, in some cases, it is not clear to us how to fix them.


Despite the problems in DK's manual formalization,
we believe that these problems do not
invalidate their work and that their results are still true. In fact we have nothing but praise for their detailed
and clearly written metatheory and proofs, which provided invaluable
help to our own work.
We expect that for most non-trivial manual
proofs similar problems exist, so this should not be understood as a sign of sloppiness
on their part. Instead it should be an indicator that reinforces the arguments
for mechanical formalizations: manual formalizations are error-prone due to the multiple
tedious details involved in them.
There are several other examples of manual formalizations that were found to have
similar problems. For example, Klein et al.~\cite{KleinRunYourResearch}
mechanized formalizations
in Redex for nine ICFP 2009 papers and all were found to have mistakes.

Another challenge was variable binding. Type inference algorithms
typically do not rely simply on local environments but instead
propagate information across judgments. While local environments are
well-studied in mechanical formalizations, there is little work on how
to deal with the complex forms of binding employed by type inference algorithms
in theorem provers. To
keep track of variable scoping, DK's algorithmic version employs input
and output contexts to track information that is discovered through
type inference. However modeling output contexts in a theorem prover
is non-trivial.

Due to those two challenges, our work takes a different approach by refining and
extending the idea of \emph{worklist judgments}~\cite{itp2018},
proposed recently to mechanically formalize an algorithm for
\emph{polymorphic subtyping}~\cite{odersky1996putting}. A key innovation in our work is how
to adapt the idea of worklist judgments to
\emph{inference judgments}, which are not needed for polymorphic
subtyping, but are necessary for type-inference.  The idea is to use a \emph{continuation
passing style} to enable the transfer of inferred information across
judgments. A further refinement to the idea of worklist judgments is
the \emph{unification between ordered
  contexts~\cite{gundry2010type,dunfield2013complete} and worklists}.  This
enables precise scope tracking of free variables in
judgments. Furthermore it avoids the duplication of context
information across judgments in worklists that occurs in other
techniques~\cite{Reed2009,Abel2011higher}.
Despite the use of a different algorithm we prove the
same results as DK, although with significantly different proofs and
proof techniques. The calculus and its metatheory
have been fully formalized in the Abella theorem prover~\cite{AbellaDesc}.

In summary, the contributions of this paper are:

\begin{itemize}

\item {\bf A fully mechanized formalization of type inference with
  higher-ranked types:} Our work presents the first fully mechanized formalization
  for type inference of higher ranked types. The formalization is done in the
  Abella theorem prover~\cite{AbellaDesc} and it is available
  online at \url{https://github.com/JimmyZJX/TypingFormalization}.

\item {\bf A new algorithm for DK's type system:} Our work proposes a novel algorithm that implements
  DK's declarative bidirectional type system. We prove
  \emph{soundness}, \emph{completeness} and
  \emph{decidability}. 

\item {\bf Worklists with inference judgments:} One technical contribution is the
  support for inference judgments using worklists. The idea is to
  use a continuation passing style to enable the transfer of inferred information across
  judgments. 

\item {\bf Unification of worklists and contexts:} Another technical contribution is the unification
  between ordered contexts and worklists. This enables precise scope tracking
  of variables in judgments, and avoids the duplication of context information across
  judgments in worklists.

\begin{comment}
\jimmy{Notes @20190211 4 points of novalty:\\
1) Dealing with inference judgments and CPS-style chains\\
2) The form of the judgment itself with a single shared context\\
3) The way we deal with scope (which may follow from 2)\\
4) Immediate substitution (judgment list)
}
\end{comment}

\end{itemize}

\section{Overview}

This section starts with a discussion on DK's declarative type system.
Then it introduces several techniques that have been used in algorithmic
formulations, and which have influenced our own algorithmic design.
Finally we introduce the novelties of our new algorithm.
In particular the support for inference judgments in
worklist judgments, and a new form of worklist judgment
that unifies \emph{ordered contexts} and the worklists themselves. 

% DK's declarative system TODO point to background

\subsection{DK's Declarative System}

\begin{figure}[t]
	\[
	\begin{array}{l@{\qquad}lcl}
	\text{Type variables}\qquad&a, b
	\\
	\text{Types}\qquad&A, B, C &::=&\quad 1 \mid a \mid \all A \mid A\to B\\
	\text{Monotypes}\qquad&\tau,\sigma &::=&\quad 1 \mid a \mid \tau\to \sigma
	\\
	\text{Expressions}\qquad&e &::=&\quad x \mid () \mid \lam e \mid e_1~e_2 \mid (e:A)
	\\
	\text{Contexts}\qquad&\Psi &::=&\quad \nil \mid \Psi, a \mid \Psi, x:A
	\end{array}
	\]
	\Description{Syntax of Declarative System}
	\caption{Syntax of Declarative System (Extends Figure~\ref{fig:ITP:decl:syntax})}
\end{figure}

\begin{figure}[t]
\centering \framebox{$\Psi \vdash A$} Well-formed declarative type
\begin{gather*}
\inferrule*[right=$\mathtt{wf_d unit}$]
    {~}{\Psi\vdash 1}
\qquad
\inferrule*[right=$\mathtt{wf_d var}$]
    {a\in\Psi}{\Psi\vdash a}
\qquad
\inferrule*[right=$\mathtt{wf_d{\to}}$]
    {\Psi\vdash A\quad \Psi\vdash B}
    {\Psi\vdash A\to B}
\qquad
\inferrule*[right=$\mathtt{wf_d\forall}$]
    {\Psi, a\vdash A}
    {\Psi\vdash \forall a. A}
\end{gather*}

\centering \framebox{$\Psi \vdash e$} Well-formed declarative expression
\begin{gather*}
\inferrule*[right=$\mathtt{wf_d tmvar}$]
    {x:A\in\Psi}{\Psi\vdash x}
\qquad
\inferrule*[right=$\mathtt{wf_d tmunit}$]
    {~}{\Psi\vdash ()}
\qquad
\inferrule*[right=$\mathtt{wf_d abs}$]
    {\Psi,x:A\vdash e}
    {\Psi\vdash \lam e}
\\
\inferrule*[right=$\mathtt{wf_d app}$]
    {\Psi\vdash e_1 \quad \Psi\vdash e_2}
    {\Psi\vdash e_1~e_2}
\qquad
\inferrule*[right=$\mathtt{wf_d anno}$]
    {\Psi\vdash A \quad \Psi\vdash e}
    {\Psi\vdash (e:A)}
\end{gather*}

\centering \framebox{$\Psi \vdash A \le B$} Declarative subtyping
\begin{gather*}
\inferrule*[right=$\mathtt{{\le}Var}$]
    {a\in\Psi}{\Psi\vdash a\le a}
\qquad
\inferrule*[right=$\mathtt{{\le}Unit}$]
    {~}{\Psi \vdash 1 \le 1}
\qquad
\inferrule*[right=$\mathtt{{\le}{\to}}$]
    {\Psi \vdash B_1 \le A_1 \quad \Psi \vdash A_2 \le B_2}
    {\Psi\vdash A_1\to A_2 \le B_1\to B_2}
\\
\inferrule*[right=$\mathtt{{\le}\forall L}$]
    {\Psi\vdash \tau \quad \Psi\vdash [\tau/a] A \le B}
    {\Psi\vdash \all A \le B}
\qquad
\inferrule*[right=$\mathtt{{\le}\forall R}$]
    {\Psi, b\vdash A\le B}
    {\Psi\vdash A \le \all[b]B}
\end{gather*}
\Description{Declarative Well-formedness and Subtyping}
\caption{%Well-formedness of Declarative Types and 
Declarative Well-formedness and Subtyping}
\end{figure}

\begin{figure}[t]
\begin{tabular}{rl}
    \framebox{$\Psi \vdash e \Lto A$} & $e$ checks against input type $A$.\\[0.5mm]
    \framebox{$\Psi \vdash e \To A$} & $e$ synthesizes output type $A$.\\[0.5mm]
    \framebox{$\Psi \vdash \appInf{A}{e}{C}$} & Applying a function of type $A$ to $e$ synthesizes type $C$.
\end{tabular}
\begin{gather*}
\inferrule*[right=$\mathtt{DeclVar}$]
    {(x:A)\in\Psi}{\Psi\vdash x\To A}
\qquad
\inferrule*[right=$\mathtt{DeclSub}$]
%e \neq \lam e' \quad B \neq \all B' \quad 
    {\Psi\vdash e\To A \quad \Psi\vdash A\le B}
    {\Psi \vdash e\Lto B}
\\
\inferrule*[right=$\mathtt{DeclAnno}$]
    {\Psi \vdash A \quad \Psi\vdash e\Lto A}
    {\Psi\vdash (e:A)\To A}
\qquad
\inferrule*[right=$\mathtt{Decl1I}$]
    {~}{\Psi\vdash () \Lto 1}
\qquad
\inferrule*[right=$\mathtt{Decl1I{\To}}$]
    {~}{\Psi\vdash () \To 1}
\\
\inferrule*[right=$\mathtt{Decl\forall I}$]
    {\Psi,a \vdash e \Lto A}
    {\Psi\vdash e\Lto \all A}
\qquad
\inferrule*[right=$\mathtt{Decl\forall App}$]
    {\Psi \vdash \tau \quad \Psi\vdash \appInf{[\tau/a]A}{e}{C} }
    {\Psi\vdash \appInf{\all A}{e}{C}}
\\
\inferrule*[right=$\mathtt{Decl{\to}I}$]
    {\Psi,x:A \vdash e\Lto B}
    {\Psi\vdash \lam e \Lto A \to B}
\qquad
\inferrule*[right=$\mathtt{Decl{\to}I{\To}}$]
    {\Psi\vdash \sigma\to\tau \quad \Psi,x:\sigma \vdash e\Lto \tau}
    {\Psi\vdash \lam e \To \sigma\to\tau}
\\
\inferrule*[right=$\mathtt{Decl{\to} E}$]
    {\Psi\vdash e_1\To A \quad \Psi\vdash \appInf{A}{e_2}{C}}
    {\Psi\vdash e_1~e_2 \To C}
\qquad
\inferrule*[right=$\mathtt{Decl{\to}App}$]
    {\Psi\vdash e \Lto A}
    {\Psi\vdash \appInf{A \to C}{e}{C}}
\end{gather*}
\Description{Declarative Typing}
\caption{Declarative Typing}
\end{figure}

Subsection~\ref{subsec:dk:decl} introduces DK's
declarative subtyping and typing systems.
We also duplicate the rules here for the convenience of the reader.

\paragraph{Overlapping Rules}
A problem that we found in the declarative system is that
some of the rules overlap with each other.
Declarative subtyping rules $\mathtt{{\le}\forall L}$ and $\mathtt{{\le}\forall R}$
both match the conclusion $\Psi\vdash \all A \le \all B$.
In such a case, choosing $\mathtt{{\le}\forall R}$ first is always better,
since we introduce the type variable $a$ to the context earlier,
which gives more flexibility on the choice of $\tau$.
The declarative typing rule $\mathtt{DeclSub}$ overlaps with
both $\mathtt{Decl\forall I}$ and $\mathtt{Decl{\to}I}$.
However, we argue that more specific rules are always the best choices,
i.e. $\mathtt{Decl\forall I}$ and $\mathtt{Decl{\to}I}$ should have
higher priority than $\mathtt{DeclSub}$.

For example, $\Psi\vdash \lam x \Lto \all a\to a$ succeeds if derived from
Rule $\mathtt{Decl\forall I}$:
$$
\inferrule*[right={$\mathtt{Decl\forall I}$}]
	{\inferrule*[Right={$\mathtt{Decl{\to}I}$}]
		{\Psi,a,x:a \vdash x \Lto a}
		{\Psi,a \vdash \lam x \Lto a \to a}
	}
	{\Psi\vdash \lam x \Lto \all a \to a},
$$
but fails when applied to $\mathtt{DeclSub}$:
$$
\inferrule*[right={$\mathtt{DeclSub}$}]
	{
		\inferrule*[right=$\mathtt{Decl{\to}I{\To}}$]
			{\Psi \vdash \blue\sigma\to \blue\tau \quad \Psi,x:\blue\sigma\vdash e \Lto \blue\tau}
			{\Psi \vdash \lam x \To \blue\sigma\to \blue\tau}\quad
		\inferrule*[Right=$\mathtt{{\le}\forall R}$]
			{\inferrule*[Right=$\mathtt{{\le}{\to}}$]
				{
					\inferrule*[Right=$?$]
						{\textit{Impossible!} \\\\ a \notin \text{FV}(\blue\sigma)}
						{\Psi,a \vdash a \le \blue\sigma}
					\quad \Psi,a \vdash \blue\tau \le a
				}
				{\Psi,a \vdash \blue\sigma\to \blue\tau \le a \to a}
			}
			{\Psi\vdash \blue\sigma\to \blue\tau\le \all a \to a}
	}
{\Psi\vdash \lam x \Lto \all a \to a}.
$$

Rule $\mathtt{Decl{\to}I}$ is also better at handling higher-order types.
When the lambda-expression to be inferred has a polymorphic input type,
such as $\all a \to a$,
$\mathtt{DeclSub}$ may not derive some judgments.
For example, $\Psi,id:\all a\to a \vdash \lam[f] f~id~(f~()) \Lto (\all a\to a) \to 1$
requires the argument of the lambda-expression to be a polymorphic type,
otherwise it could not be applied to both $id$ and $()$.
If Rule $\mathtt{DeclSub}$ was chosen for derivation,
the type of its argument is restricted by Rule $\mathtt{Decl{\to}I{\To}}$,
which is not a polymorphic type.
By contrast,
Rule $\mathtt{Decl{\to}I}$ keeps the polymorphic argument type $\all a\to a$,
and will successfully derive the judgment.

We will come back to this
topic in Section~\ref{sec:metatheory:non-overlapping}
and formally derive a system without overlapping rules.

%-------------------------------------------------------------------------------
\subsection{DK's Algorithm}\label{ssec:DK_Algorithm}

DK's algorithm version revolves around their notion of \emph{algorithmic context}.
\[
\begin{array}{l@{\qquad}lcl}
\text{Algorithmic Contexts}\qquad&\Gamma,\Delta,\Theta &::=&\quad \nil \mid
\Gamma, a \mid \Gamma, x:A \mid \Gamma, \al \mid \Gamma, \al = \tau \mid
\Gamma, \blacktriangleright_{\al}
\end{array}
\]
In addition to the regular (universally quantified) type variables $a$, the
algorithmic context also contains \emph{existential} type variables
$\al$. These are placeholders for monotypes $\tau$ that are still to
be determined by the inference algorithm. When the existential variable is
``solved'', its entry in the context is replaced by the assignment
$\al = \tau$.
A context application on a type, denoted by $[\Gamma]A$,
substitutes all solved existential type variables in $\Gamma$
with their solutions on type $A$.

All algorithmic judgments thread an algorithmic context. They have the form
$\Gamma \vdash \ldots \dashv \Delta$, where $\Gamma$ is the input context and
$\Delta$ is the output context:
$\Gamma \vdash A \le B \dashv \Delta$  for subtyping, 
$\Gamma \vdash e \Leftarrow A \dashv \Delta$  for type checking, and so on. 
The output context is a functional update of the input context that records newly
introduced existentials and solutions.

Solutions are incrementally propagated by applying the algorithmic output
context of a previous task as substitutions to the next task. This can be seen
in the subsumption rule:
\[
\inferrule*[right=$\mathtt{DK\_Sub}$]
  {\Gamma \vdash e \Rightarrow A \dashv \Theta \\ 
   \Theta \vdash [\Theta]A \le [\Theta]B \dashv \Delta
  }
  { \Gamma \vdash e \Leftarrow B \dashv \Delta}
\]
The inference task yields an output context $\Theta$ which is applied as a substitution
to the types $A$ and $B$ before performing the subtyping check to propagate any solutions
of existential variables that appear in $A$ and $B$.

\paragraph{Markers for scoping.}
The sequential order of entries in the algorithmic context, in combination with
the threading of contexts,  does not perfectly capture the scoping of all
existential variables. For this reason the DK algorithm uses scope markers
$\blacktriangleright_{\al}$ in a few places. An example is given in the following
rule:
\[
\inferrule*[right=$\mathtt{DK\_{\le}\forall L}$]
  {\Gamma,\blacktriangleright_{\al}, \al \vdash [\al/a]A \le B \dashv \Delta,\blacktriangleright_{\al},\Theta}
  {\Gamma \vdash \all A \le B \dashv \Delta}
\]
To indicate that the scope of $\al$ is local to the subtyping check
$[\al/a]A \le B$, the marker is pushed onto its input stack and popped
from the output stack together with the subsequent part $\Theta$, which may
refer to $\al$. 
(Remember that later entries may refer to earlier
ones, but not vice versa.) This way $\al$ does not escape its scope.
\begin{comment}
A type variable may have a similar functionality to the scoping markers.
An example rule that checks an expression against a polymorphic type is as follows:
$$
\inferrule*
{\Gm, a \vdash e \Lto A \dashv \Delta, a, \Theta}
{\Gm \vdash e \Lto \all A \dashv \Delta}
$$
\end{comment}

One may suggest that the marker $\blacktriangleright_{\al}$ is somewhat redundant,
since $\al$ already declares the scope.
However, in the following rule,
\[
\inferrule*[right=$\mathtt{DK\_InstLArr}$]
  {\Gamma[\al[2], \al[1], \al = \al[1] \to \al[2]] \vdash A_1 \le \al[1] \dashv \Theta \\
    \Theta \vdash \al[2] \le [\Theta]A_2 \dashv \Delta
  }
  {\Gamma[\al] \vdash \al \le A_1 \to A_2 \dashv \Delta}
\]
the algorithm introduces new existential variables right before $\al$.
In such case, the marker $\blacktriangleright_{\al}$ still appears
to the left of them.
Without the marker, it will be difficult to recycle the new existential variables
$\al[1]$ and $\al[2]$ properly, which should have the same scope of $\al$
and thus should be recycled together with $\al$.


%Due to the complication of its scoping control method,
%it is hard to argue the preciseness of scoping for each variable when designing algorithms.

At first sight, the DK algorithm would seem to be a good basis for mechanization. After all,
 it comes with a careful description and extensive manual proofs.
Unfortunately, we ran into several obstacles that have prompted us to formulate
a different, more mechanization-friendly algorithm.

%- - - - - - - - - - - - - - - - - - - - - - - - - - - - - - - - - - - - - - - - 
\paragraph{Broken Metatheory} % Talk about the lemmas that are false, show counterexample.
While going through the manual proofs of DK's algorithm, we found several
problems.  Indeed, two proofs of lemmas---Lemma 19 (Extension Equality
Preservation) and Lemma 14 (Subsumption)--- wrongly apply induction hypotheses
in several cases. Fortunately, we have found simple workarounds that fix these
proofs without affecting the appeals to these lemmas.

% The proof of Lemma 19 (Extension Equality Preservation) applies a wrong
% induction hypothesis for the ${\longrightarrow}\mathtt{Uvar}$ case.
% Fortunately the lemma could be proven without the well-formedness conditions
% $\Gm\vdash A$ and $\Gm\vdash B$.
% Similar problem happens for Lemma 14 (Subsumption), where many applications of
% induction hypotheses are not correct.
% For manual proofs, induction hypotheses are not automatically generated by programs,
% thus are easily misused.

More seriously, we have also found a lemma that simply does not hold:
Lemma 29 (Parallel Admissibility)\footnote{Ningning Xie found the issue with Lemma 29 in 2016 on an earlier attempt to mechanically
  formalize DK's algorithm. The authors acknowledged the problem after we contacted them through email.
  Although they briefly mentioned that it should be possible to use a weaker lemma instead they did
  not go into details.}.
%
This lemma is used to relate the algorithmic system and declarative system
before and after the instantiation procedure.
We believe that the general idea of the lemma is correct,
but the statement may fail when the sensitive ordering of variables
breaks the ``parallelism'' in some corner cases.
%
This lemma is a cornerstone of the two metatheoretical results 
of the algorithm, soundness, and completeness with respect to the declarative system.
In particular, both instantiation soundness (i.e. a part of subtyping
soundness) and typing completeness directly require the broken lemma.
Moreover, Lemma 54 (Typing Extension) also requires the broken lemma and is
itself used 13 times in the proof of typing soundness and completeness.
Unfortunately, we have not yet found a way to fix this problem.

In what follows, we briefly discuss the problem through counterexamples.
False lemmas are found in the manual proofs of DK' two papers
\citep{dunfield2013complete} and \citep{DunfieldIndexed}.
\begin{itemize}
    \item
        In the first paper, Lemma 29 on page 9 of its appendix says:
        \begin{lemma}[Parallel Admissibility of~\citep{dunfield2013complete}]~\\
        If $\Gamma_L \longrightarrow \Delta_L$ and
        $\Gamma_L, \Gamma_R \longrightarrow \Delta_L, \Delta_R$ then:
        \begin{enumerate}
            \item $\Gamma_L,\al,\Gamma_R \longrightarrow \Delta_L, \al, \Delta_R$
            \item If $\Delta_L \vdash \tau'$ then
                $\Gamma_L,\al,\Gamma_R \longrightarrow \Delta_L, \al = \tau', \Delta_R$.
            \item If $\Gamma_L \vdash \tau$ and $\Delta_L \vdash \tau'$ and
                $[\Delta_L]\tau = [\Delta_L]\tau'$, then
                $\Gamma_L, \al = \tau, \Gamma_R \longrightarrow \Delta_L, \al = \tau', \Delta_R$.
        \end{enumerate}
        \end{lemma}

        We give a counter-example to this lemma:\\
        Pick $\Gm_L = \nil, \Gm_R = \bt, \Delta_L = \bt, \Delta_R = \nil$, then both conditions
        $\nil\longrightarrow \bt$ and $\bt\longrightarrow \bt$ hold, but the first conclusion
        $\al,\bt \longrightarrow \bt,\al$ does not hold.
    \item
        In the second paper, as an extended work to the first paper, Lemma 26 on page 22 of its appendix says:
        \begin{lemma}[Parallel Admissibility of~\citep{DunfieldIndexed}]~\\
        If $\Gamma_L \longrightarrow \Delta_L$ and
        $\Gamma_L, \Gamma_R \longrightarrow \Delta_L, \Delta_R$ then:
        \begin{enumerate}
            \item $\Gamma_L,\al:\kappa,\Gamma_R \longrightarrow \Delta_L, \al:\kappa, \Delta_R$
            \item If $\Delta_L \vdash \tau' : \kappa$ then
                $\Gamma_L,\al:\kappa,\Gamma_R \longrightarrow \Delta_L, \al:\kappa = \tau', \Delta_R$.
            \item If $\Gamma_L \vdash \tau : \kappa$ and $\Delta_L \vdash \tau' types$ and
                $[\Delta_L]\tau = [\Delta_L]\tau'$, then
                $\Gamma_L, \al:\kappa = \tau, \Gamma_R \longrightarrow \Delta_L, \al:\kappa = \tau', \Delta_R$.
        \end{enumerate}
        \end{lemma}
        
        A similar counter-example is given:\\
        Pick $\Gm_L = \nil, \Gm_R = \bt:\star, \Delta_L = \bt:\star, \Delta_R = \nil$, then both conditions
        $\nil\longrightarrow \bt:\star$ and $\bt:\star\longrightarrow \bt:\star$ hold, but the first conclusion
        $\al:\kappa,\bt:\star \longrightarrow \bt:\star,\al:\kappa$ does not hold.
\end{itemize}

%- - - - - - - - - - - - - - - - - - - - - - - - - - - - - - - - - - - - - - - - 
\paragraph{Complex Scoping and Propagation}

Besides the problematic lemmas in DK's metatheory, there are other reasons to
look for an alternative algorithmic formulation of the type system that is more
amenable to mechanization. Indeed, two aspects that are particularly
challenging to mechanize are the scoping of universal and existential type
variables, and the propagation of the instantiation of existential type
variables across judgments. 
DK is already quite disciplined on these accounts, in particular compared to
traditional constraint-based type-inference algorithms like Algorithm $\mathcal{W}$~\citep{milner1978theory} which
features an implicit global scope for all type variables. Indeed, DK uses its
explicit and ordered context $\Gamma$ that tracks the relative scope of universal and
existential variables and it is careful to only instantiate existential
variables in a well-scoped manner.

Moreover, DK's algorithm carefully propagates instantiations by recording them
into the context and threading this context through all judgments. 
% two This means that every judgments takes two contexts---an input and an output
% context---rather than the conventional single context. The output context
% records any new variable instantations; to propagate these instantiations to
% the remaining judgments, their predecessor's output context---which is their
% input context---is applied to them as a substitution.
While this works well on paper, this approach is still fairly involved and thus
hard to reason about in a mechanized setting. Indeed, the instantiations have
to be recorded in the context and are applied incrementally to each remaining
judgment in turn, rather than immediately to all remaining judgments at once.
Also, as we have mentioned above, the stack discipline of the ordered contexts
does not mesh well with the use of output contexts; explicit marker entries are
needed in two places to demarcate the end of an existential variable's scope.
Actually, we found a scoping issue related to the subsumption rule $\mathtt{DK\_Sub}$,
which might cause existential variables to leak across judgments.
In Section~\ref{sec:discussion:scoping} we give a detailed discussion.
%A detailed discussion on this issue with a possible fix is .

The complications of scoping and propagation are compelling reasons
to look for another algorithm that is easier to
reason about in a mechanized setting.


%-------------------------------------------------------------------------------
\subsection{Judgment Lists}\label{sec:overview:list}
To avoid the problem of incrementally applying a substitution to remaining
tasks, we can find inspiration in the formulation of constraint solving
algorithms. For instance, the well-known unification
algorithm by \citet{unification} decomposes the problem of unifying two terms $s \stackrel{.}{=} t$ into a number
of related unification problems between pairs of terms $s_i \stackrel{.}{=} t_i$. These smaller
problems are not tackled independently, but kept together in a set $S$. 
The algorithm itself is typically formulated as a small-step-style state
transition system $S \rightarrowtail S'$ that rewrites the set of unification
problems until it is in solved form or until a contradiction has been found.
For instance, the variable elimination rule is written as:
\[
   x \stackrel{.}{=} t, S  ~~\rightarrowtail~~  x \stackrel{.}{=} t, [t/x]S   \qquad\qquad{if}~x \not\in t~\text{and}~{x \in S}
\]
Because the whole set is explicitly available, the variable $x$ can be
simultaneously substituted.

In the above unification problem, all variables are implicitly bound in the same
global scope. 
Some constraint solving algorithms for Hindley-Milner type
inference use similar ideas, but are more careful tracking the
scopes of variables~\citep{remy-attapl}.
% However they have separate phases for constraint generation and solving.
Recent unification algorithms for dependently-typed languages
are also more explicit about scopes. For instance, \citet{Reed2009} represents a unification
problem as $\Delta \vdash P$ where $P$ is a set of equations to be solved and $\Delta$ is
a (modal) context. \citet{Abel2011higher} even use multiple contexts within a unification problem.
Such a problem is denoted $\Delta \Vdash \mathcal{K}$ where the meta-context
$\Delta$ contains all the typings of meta-variables in the constraint set
$\mathcal{K}$. The latter consists of constraints like $\Psi \vdash M = N : C$
that are equipped with their individual context $\Psi$. While accurately tracking
the scoping of regular and meta-variables, this approach is not ideal because it
repeatedly copies contexts when decomposing a unification problem, and
later it has to 
substitute solutions into these copies.

% %-------------------------------------------------------------------------------
% \subsection{Small-Step Unification}
% % Mention some algorithms for unification for 
% % dependent types that use a small-step approach. Credit them later for some ideas that 
% % we also employ.
% There are literals that make use of small-step unification for dependently typed
% inference and reconstruction algorithms~\citep{Reed2009,Abel2011higher}.
% These approaches collect a list of constraints and process one at a time.
% Similar to DK's algorithm, unification variables are used to represent the types to guess.
% In order to solve unification variables to correctly scoped types
% (typically constrained by the context when the variable is introduced),
% the context information should be kept with the variables.
% Judgments that represent unifications of terms might also require such a context
% to keep the constraint collection well-formed.
% 
% As an example, Abel et al. defines their constraint $K$ by
% \begin{gather*}
% \begin{aligned}
% K &::= {\top} \mid {\bot} &&\text{Trivial constraint and inconsistency.}\\
%     & \ \mid \  \Psi\vdash M = N : C &&\text{Unify term $M$ with $N$.}\\
%     & \ \mid \  \Psi\mid R:A \vdash E = E' &&\text{Unify evaluation context $E$ with $E'$.}\\
%     & \ \mid \  \Psi\vdash u\leftarrow M: C &&\text{Solution for $u$ (metavariable) found.}
% \end{aligned}
% \end{gather*}
% where the unification constraints and metavariable solutions are bound by a proper context.
% A unification problem in their system is described by $\Delta \Vdash \overline{K}$,
% where $\Delta$ is a collection of metavariables with their defined scopes.
% This approach clearly states the scroping of each unification problem,
% therefore rules out invalid instantiations to metavariables.
% 
% However, such ``duplicated contexts'' are not ideal for our formalization.
% Following DK's algorithm, an existential variable $\al$ defined in the context
% could be decomposed into a function type $\al[1] \to \al[2]$,
% so the declaration of $\al$ should be replaced by two declarations $\al[1], \al[2]$.
% Such operation causes all the context that contains $\al$ to change,
% which brings difficulty on the synchronization of multiple contexts,
% as variable declarations are not centralized.
% 
% We find a way of encoding multiple contexts by ``compressing'' them into a single worklist.
% As our type inference rules applied to a judgment,
% we typically keep the context or add variable declarations to the old context
% before analysing its sub-judgment(s).
% This enables us to write the worklist judgment in such a form:
% $$\Gm, \{\text{variable declarations}\}, \{\text{judgment}\},
% \{\text{variable declarations}\}, \{\text{judgment}\}, \ldots .$$
% Typically, we simplify the right-most judgment and push back smaller tasks,
% where the old ``context'' automatically gets inherited.
% When some rules solve or partially solve an existential variable,
% we could easily propagate the solution to all the judgments,
% and safely modify its declaration, such as remove from the worklist.
% New judgment(s) may also introduce new \emph{local} variables by declaring
% that variable right before the new judgment(s).
% After the new judgments are satisfied, these local variables are properly recycled.
% 
% % \jimmy{TODO explain the duplicated context with some examples (their judgment form)}
% % Some issues to point out: Duplicated contexts (rather than shared contexts), which make 
% % the formalization harder since requires ``synchronizing'' things ...
% 
%-------------------------------------------------------------------------------
\subsection{Single-Context Worklist Algorithm for Subtyping}

As we have seen in Chapter~\ref{chap:ITP},
an algorithm based on \emph{worklist judgments} is mechanized
and shown to be correct with respect to DK's declarative
subtyping judgment.
This approach overcomes some problems in DK's algorithmic formulation
by using a worklist-based judgment of the form $$\Gamma \vdash \Omega$$
where $\Omega$ is a worklist (or sequence) of subtyping problems of the
form $A \leq B$.  The judgment is defined by case analysis on the first
element of $\Omega$ and recursively processes the worklist until it is empty.
Using both a single common ordered context $\Gamma$ and a worklist $\Omega$ greatly
simplifies propagating the instantiation of type variables in one
subtyping problem to the remaining ones.

Unfortunately, this work does not solve all problems. In particular, it has two
major limitations that prevent it from scaling to the whole DK system. 

\paragraph{Scoping Garbage} Firstly, the single common ordered context 
$\Gamma$ does not accurately reflect the type and unification variables
currently in scope. Instead, it is an overapproximation that steadily accrues
variables, and only drops those unification variables that are instantiated.
In other words, $\Gamma$ contains ``garbage'' that is no longer in scope.
This complicates establishing the relation with the declarative system.


\paragraph{No Inference Judgments} 
Secondly, and more importantly, the approach only deals with a judgment for
\emph{checking} whether one type is the subtype of another. While this may not
be so difficult to adapt to the \emph{checking} mode of term typing $\Gamma
\vdash e \Leftarrow A$, it clearly lacks the functionality to support the
\emph{inference} mode of term typing $\Gamma \vdash e \Rightarrow A$. Indeed,
the latter requires not only the communication of unification variable
instantiations from one typing problem to another, but also an inferred type.

%-------------------------------------------------------------------------------
\subsection{Algorithmic Type Inference for Higher-Ranked Types: Key Ideas}

Our new algorithmic type system builds on the work above, but
addresses the open problems.

\paragraph{Scope Tracking}
We avoid scoping garbage by blending the ordered context and the
worklist into a single syntactic sort $\Gamma$, our algorithmic worklist. This
algorithmic worklist interleaves (type and term) variables with \emph{work}
like checking or inferring types of expressions. The interleaving keeps track
of the variable scopes in the usual, natural way: each variable is in scope of
anything in front of it in the worklist. If there is nothing in front, the
variable is no longer needed and can be popped from the worklist. This way, no
garbage (i.e. variables out-of-scope) builds up.

%\jimmy{Comment A is not satisfied with the term "continuation passing style" for our definition.}

$$\begin{aligned}
\text{Algorithmic judgment chain}\qquad&\jg &::=&\quad A \le B \mid e\Lto A \mid e\To_{a} \jg \mid \appInfAlg{A}{e}\\
\text{Algorithmic worklist}\qquad&\Gm &::=&\quad \nil \mid \Gm, a \mid \Gm, \al \mid \Gm, x: A \mid \Gm \Vdash \jg\\
\end{aligned}$$

For example, suppose that the top judgment of the following worklist
checks the identity function against $\all a \to a$:
$$\Gm \Vdash \lam x \Lto \all a \to a$$
To proceed, two auxiliary variables $a$ and $x$ are introduced to help the type checking:
$$\Gm, a, x:a \Vdash x \Lto a$$
which will be satisfied after several steps, and the worklist becomes
$$\Gm, a, x:a$$
Since the variable declarations $a, x:a$ are only used for a judgment already processed,
they can be safely removed, leaving the remaining worklist $\Gm$ to be further reduced.

Our worklist can be seen as an all-in-one stack,
containing variable declarations and subtyping/ typing judgments.
The stack is an enriched form of ordered context,
and it has a similar variable scoping scheme.
%The well-formedness condition, shown in Figure~\ref{fig:alg:syntax},
%states that every variable must be declared prior to it usage.
%Therefore, if any variable declaration is the top element,
%then no usage should appear in the rest of the worklist, so it can be removed.

% $$\Gm \Vdash \all a \to a \le \all a \to a$$
% $$\Gm, a, \al \Vdash \al \to \al \le a \to a$$

\paragraph{Inference Judgments}
To express the DK's inference judgments, we use a novel form of work entries in
the worklist: our algorithmic judgment chains $\omega$. In its simplest form,
such a judgment chain is just a check, like a subtyping check $A \leq B$ or a
term typecheck $e \Leftarrow A$. 
However, the non-trivial forms of chains capture an
inference task together with the work that depends on its outcome. For
instance, a type inference task takes the form $e \Rightarrow_a \omega$.
This form expresses that we need to infer the type, say $A$, for expression $e$ and use it
in the chain $\omega$ by substituting it for the placeholder type variable $a$.
Notice that such $a$ binds a fresh type variable for the inner chain $\jg$,
which behaves similarly to the variable declarations in the context.
%\bruno{Removed CPS mention here. But if we remove it here should we remove it
%everywhere? Tom, your input would be good regarding this.}

Take the following worklist as an example
$$\al \Vdash \underline{(\lam x)~() \To_a a \le \al} , x:\al, \bt \Vdash \underline{\al \le \bt} \Vdash \underline{\bt \le 1}$$
There are three (underlined) judgment chains in the worklist,
where the first and second judgment chains (from the right) are two subtyping judgments,
and the third judgment chain, $(\lam x)~() \To_a a \le \al$,
is a sequence of an inference judgment followed by a subtyping judgment.

The algorithm first analyses the two subtyping judgments and
will find the best solutions $\al = \bt = 1$
(please refer to Figure~\ref{fig:alg} for detailed derivations).
Then we substitute every instance of $\al$ and $\bt$ with $1$,
so the variable declarations can be safely removed from the worklist.
Now we reduce the worklist to the following state
$$\nil \Vdash \underline{(\lam x)~() \To_a a \le 1}, x:1$$
which has a term variable declaration as the top element.
%Because no judgments before the declaration refer to $x$,
%as a basic property of an ordered context
%(Figure~\ref{fig:alg:syntax} shows its well-formedness criteria),
%we remove all declarations before the first judgment chain to collect unused variables.
%As a result all garbage variables are collected
%once the judgment chains refering to them are processed,
%just as what we have mentioned above about the scope tracking.
After removing the garbage term variable declaration from the worklist, we process the last remaining
inference judgment $(\lam x)~()\To~?$, with the unit type $1$ as its result.
Finally, the last judgment becomes $1 \le 1$, a trivial base case.


% Our algorithm borrows some ideas from previous work, while adding new ones. 
% A small-step style processing worklists; \emph{Judgment Chains}; others. 

% Importantly we now deal with inference judgment.


%
\section{Declarative System}

\begin{figure}[t]
    \begin{gather*}
    \begin{aligned}
        \text{Type variables}\qquad&a, b\\
        \text{Types}\qquad&A, B, C &::=&\quad 1 \mid \top \mid \bot \mid a \mid \all A \mid A\to B\\
        \text{Monotypes}\qquad&\tau &::=&\quad 1 \mid \top \mid \bot \mid a \mid \tau_1\to \tau_2\\
        \text{Expressions}\qquad&e &::=&\quad x \mid () \mid \lam e \mid e_1~e_2 \mid (e:A)\\
        \text{Context}\qquad&\Psi &::=&\quad \cdot \mid \Psi, a \mid \Psi, x:A
    \end{aligned}
    \end{gather*}
\Description{Declarative Syntax}
\caption{Declarative Syntax}\label{fig:top_decl_syntax}
\end{figure}

\paragraph{Syntax}
The syntax of the declarative system, shown in Figure~\ref{fig:top_decl_syntax},
is similar to the previous systems by having
a primitive type $1$, type variables $a$,
polymorphic types $\all A$ and function types $A \to B$.
Additionally, top and bottom types are introduced to the type system.
% intro of top/bot moved to intro

The well-formedness formalization of the system is standard
and almost identical to the previous systems,
therefore we omit the formal definitions.


\begin{figure}[t]
    \framebox{$\Psi \vdash A \le B$}
    \begin{gather*}
    \inferrule*[right=$\mathtt{{\le}Var}$]
    {a\in\Psi}{\Psi\vdash a\le a}
    \qquad
    \inferrule*[right=$\mathtt{{\le}Unit}$]
    {~}{\Psi \vdash 1 \le 1}
    \qquad
    \inferrule*[right=$\mathtt{{\le}{\to}}$]
    {\Psi \vdash B_1 \le A_1 \quad \Psi \vdash A_2 \le B_2}
    {\Psi\vdash A_1\to A_2 \le B_1\to B_2}
    \\
    \inferrule*[right=$\mathtt{{\le}\forall L}$]
    {\Psi\vdash \tau \quad \Psi\vdash [\tau/a] A \le B}
    {\Psi\vdash \all A \le B}
    \qquad
    \inferrule*[right=$\mathtt{{\le}\forall R}$]
    {\Psi, b\vdash A\le B}
    {\Psi\vdash A \le \all[b]B}
    \\
    \pmb{
    \inferrule*[right=$\mathtt{{\le}Top}$]
    {~}
    {A \le \top}
    }
    \qquad
    \pmb{
    \inferrule*[right=$\mathtt{{\le}Bot}$]
    {~}
    {\bot \le A}
    }
    \end{gather*}
\Description{Declarative Subtyping}
\caption{Declarative Subtyping}\label{fig:top_decl_subtyping}
\end{figure}

\paragraph{Declarative Subtyping}
Shown in Figure~\ref{fig:top_decl_subtyping},
the declarative subtyping extends the polymorphic subtyping relation
originally proposed by \citet{odersky1996putting}
by adding rules $\mathtt{{\le}Top}$ and $\mathtt{{\le}Bot}$,
defining the properties of the $\top$ and $\bot$ types, respectively.
Although the new rules seem quite simple,
they may increase the uncertainty of polymorphic instantiations.
For example, the subtyping judgment
\[\all a \to a \le \bot \to \top\]
accepts any well-formed instantiation on the polymorphic type $\all a \to a$.


\begin{figure}[t]
    \begin{tabular}{rl}
        \framebox{$\Psi \vdash e \Lto A$} & $e$ checks against input type $A$.\\[0.5mm]
        \framebox{$\Psi \vdash e \To A$} & $e$ synthesizes output type $A$.\\[0.5mm]
        \framebox{$\Psi \vdash \appInf{A}{e}{C}$} & Applying a function of type $A$ to $e$ synthesizes type $C$.
    \end{tabular}
    \begin{gather*}
    \inferrule*[right=$\mathtt{DeclVar}$]
        {(x:A)\in\Psi}{\Psi\vdash x\To A}
    \qquad
    \inferrule*[right=$\mathtt{DeclSub}$]
    %e \neq \lam e' \quad B \neq \all B' \quad 
        {\Psi\vdash e\To A \quad \Psi\vdash A\le B}
        {\Psi \vdash e\Lto B}
    \\
    \inferrule*[right=$\mathtt{DeclAnno}$]
        {\Psi \vdash A \quad \Psi\vdash e\Lto A}
        {\Psi\vdash (e:A)\To A}
    \qquad
    \inferrule*[right=$\mathtt{Decl1I{\To}}$]
        {~}{\Psi\vdash () \To 1}
    \\
    \inferrule*[right=$\mathtt{Decl1I}$]
        {~}{\Psi\vdash () \Lto 1}
    \qquad
    \inferrule*[right=$\mathtt{Decl\top}$]
        {\Psi \vdash e}
        {\Psi\vdash e \Lto \top}
    \qquad
    \inferrule*[right=$\mathtt{Decl{\bot}App}$]
        {\Psi\vdash e}
        {\Psi\vdash \appInf{\bot}{e}{\bot}}
    \\
    \inferrule*[right=$\mathtt{Decl\forall I}$]
        {\Psi,a \vdash e \Lto A}
        {\Psi\vdash e\Lto \all A}
    \qquad
    \inferrule*[right=$\mathtt{Decl\forall App}$]
        {\Psi \vdash \tau \quad \Psi\vdash \appInf{[\tau/a]A}{e}{C} }
        {\Psi\vdash \appInf{\all A}{e}{C}}
    \\
    \inferrule*[right=$\mathtt{Decl{\to}I}$]
        {\Psi,x:A \vdash e\Lto B}
        {\Psi\vdash \lam e \Lto A \to B}
    \qquad
    \inferrule*[right=$\mathtt{Decl{\to}I{\To}}$]
        {\Psi\vdash \sigma\to\tau \quad \Psi,x:\sigma \vdash e\Lto \tau}
        {\Psi\vdash \lam e \To \sigma\to\tau}
    \\
    \inferrule*[right=$\mathtt{Decl{\to} E}$]
        {\Psi\vdash e_1\To A \quad \Psi\vdash \appInf{A}{e_2}{C}}
        {\Psi\vdash e_1~e_2 \To C}
    \qquad
    \inferrule*[right=$\mathtt{Decl{\to}App}$]
        {\Psi\vdash e \Lto A}
        {\Psi\vdash \appInf{A \to C}{e}{C}}
    \end{gather*}
\Description{Declarative Typing}
\caption{Declarative Typing}\label{fig:top_decl_typing}
\end{figure}

\paragraph{Declarative Typing}

The declarative typing rules, shown in Figure~\ref{fig:top_decl_typing},
extend DK's higher-rank type system in order to support the top and bottom types.
Rule $\mathtt{Decl\top}$ allows any well-formed expression to check against $\top$.
Rule $\mathtt{Decl{\bot}App}$ returns the $\bot$ type
when a function of $\bot$ type is applied to any argument.
All other rules remain exactly the same as our previous work.

It's worth mentioning that the design of the two new rules
is driven by the subsumption property described in Section~\ref{sec:meta:decl}.
They maintain the property in presence of a more powerful declarative subtyping,
and we will discuss further later in that part.

\setcounter{algRuleCounter}{0}

\section{Algorithmic System}

This section introduces a novel algorithmic system that implements 
DK's declarative specification. The new algorithm extends the idea
of worklists proposed by \citet{itp2018} in two ways. Firstly,
unlike \citet{itp2018}'s worklists, the scope of variables is precisely tracked
and variables do not escape their scope. This is achieved by unifying algorithmic contexts and the worklists themselves.
Secondly, our algorithm also
accounts for the type system (and not just subtyping). To deal with
inference judgments that arise in the type system we employ a \emph{continuation
passing style} to enable the transfer of inferred information across
judgments in a worklist.

\subsection{Syntax and Well-Formedness}

Figure~\ref{fig:alg:syntax} shows the syntax and well-formedness
judgments used by the algorithm. Similarly to the declarative system 
the well-formedness rules are unsurprising and merely ensure 
well-scopedness.

\begin{figure}
\begin{gather*}
\begin{aligned}
% \text{Type variables}\qquad&a, b\\
\text{Existential variables}\qquad&\al, \bt
\\
%\text{Types}\qquad&A', B', C' &::=&\quad 1 \mid a \mid \forall x. A' \mid A'\to B'\\
% \text{Mono-types}\qquad&\tau &::=&\quad 1 \mid a \mid \tau_1\to \tau_2\\
%\text{Context}\qquad&\Psi &::=&\quad \nil \mid \Psi, a \mid \Psi, x:A' \color{red} \\[2mm]
% \text{Declarative Judgments}\qquad&\jo &::=&\quad \nil \mid \P\vdash A' \le B' \jc \jo\\
% &&&\quad \mid \P\vdash e\Leftarrow A' \jc \jo \mid \P\vdash e\To A' \jc \jo \mid \P\vdash A' \bullet e\TTo C' \jc \jo\\
% \noalign{\jimmy{\text{What do the markers do: $\G \vdash,a \vdash,b\vdash \j_1;\j_2;\j_3$ means $\G,a,b\vdash \j_1 \land \G,a\vdash \j_1,\j_2 \land \G\vdash \j_3$}}} \\[2mm]
\text{Algorithmic types}\qquad&A, B, C &::=& \quad 1 \mid a \mid \all A \mid A\to B \mid \al % \quad 1 \mid a \mid \al \mid \forall a. A \mid A\to B
\\
% \text{Terms}\qquad&e&::=&\quad x \mid () \mid \lam{e} \mid e_1~e_2 \mid (e:A)
% \\[2mm]
\text{Algorithmic judgment chain}\qquad&\jg &::=&\quad A \le B \mid e\Lto A \mid e\To_{a} \jg \mid \appInfAlg{A}{e}
\\
\text{Algorithmic worklist}\qquad&\Gm &::=&\quad \nil \mid \Gm, a \mid \Gm, \al \mid \Gm, x: A \mid \Gm \Vdash \jg%\\
% \text{Declarative worklist}\qquad&\Om &::=&\quad \nil \mid \Om, a \mid \Om, x: A \mid \Om \Vdash \jg
\end{aligned}
\end{gather*}

\centering \framebox{$\Gm \vdash A$} Well-formed algorithmic type
\begin{gather*}
\inferrule*[right=$\mathtt{wf\_ unit}$]
    {~}{\Gm\vdash 1}
\quad
\inferrule*[right=$\mathtt{wf\_ var}$]
    {a\in\Gm}{\Gm\vdash a}
\quad
\inferrule*[right=$\mathtt{wf\_ exvar}$]
    {\al\in\Gm}{\Gm\vdash \al}
\quad
\inferrule*[right=$\mathtt{wf\_{\to}}$]
    {\Gm\vdash A\quad \Gm\vdash B}
    {\Gm\vdash A\to B}
\quad
\inferrule*[right=$\mathtt{wf\_\forall}$]
    {\Gm, a\vdash A}
    {\Gm\vdash \forall a. A}
\end{gather*}

\centering \framebox{$\Gm \vdash e$} Well-formed algorithmic expression
\begin{gather*}
\inferrule*[right=$\mathtt{wf\_ tmvar}$]
    {x:A\in\Gm}{\Gm\vdash x}
\qquad
\inferrule*[right=$\mathtt{wf\_ tmunit}$]
    {~}{\Gm\vdash ()}
\qquad
\inferrule*[right=$\mathtt{wf\_ abs}$]
    {\Gm,x:A\vdash e}
    {\Gm\vdash \lam e}
\\
\inferrule*[right=$\mathtt{wf\_ app}$]
    {\Gm\vdash e_1 \quad \Gm\vdash e_2}
    {\Gm\vdash e_1~e_2}
\qquad
\inferrule*[right=$\mathtt{wf\_ anno}$]
    {\Gm\vdash A \quad \Gm\vdash e}
    {\Gm\vdash (e:A)}
\end{gather*}

\framebox{$\Gm\vdash\jg$} Well-formed algorithmic judgment
\begin{gather*}
\inferrule*[right=$\mathtt{wf{\le}}$]
{\Gm\vdash A \\ \Gm\vdash B}
{\Gm\vdash A\le B}
\qquad
\inferrule*[right=$\mathtt{wf{\Lto}}$]
{\Gm\vdash e \\ \Gm\vdash A}
{\Gm\vdash e \Lto A}
\\
\inferrule*[right=$\mathtt{wf{\To}}$]
{\Gm\vdash e \\ \Gm, a\vdash \jg}
{\Gm\vdash e \To_a \jg}
\qquad
\inferrule*[right=$\mathtt{wf{\TTo}}$]
{\Gm\vdash A \\ \Gm, a\vdash \jg \\ \Gm\vdash e}
{\Gm\vdash \appInfAlg{A}{e}}
\end{gather*}

\framebox{$\text{wf }\Gm$} Well-formed algorithmic worklist
\begin{gather*}
\inferrule*[right=$\mathtt{wf\nil}$]
{~}{\text{wf }\nil}
\qquad
\inferrule*[right=$\mathtt{wf_a}$]
{\text{wf }\Gm}
{\text{wf }\Gm, a}
\qquad
\inferrule*[right=$\mathtt{wf_{\al}}$]
{\text{wf }\Gm}
{\text{wf }\Gm, \al}
\qquad
\inferrule*[right=$\mathtt{wf_{of}}$]
{\text{wf }\Gm \\ \Gm\vdash A}
{\text{wf }\Gm, x:A}
\qquad
\inferrule*[right=$\mathtt{wf_{\jg}}$]
{\text{wf }\Gm \\ \Gm\vdash\jg}
{\text{wf }\Gm \Vdash\jg}
\end{gather*}
\Description{Extended Syntax and Well-Formedness for the Algorithmic System}
\caption{Extended Syntax and Well-Formedness for the Algorithmic System}\label{fig:alg:syntax}
\end{figure}

\paragraph{Existential Variables} 
The algorithmic system inherits the syntax of terms and types from 
the declarative system. It only introduces one additional feature.
In order to find unknown types $\tau$ in the declarative system, the
algorithmic system extends the declarative types $A$ with \emph{existential variables} $\al, \bt$.
They behave like unification variables,
but their scope is restricted by their position in the
algorithmic worklist rather than being global.
Any existential variable $\al$ should only be solved to
a type that is well-formed with respect to the worklist to which $\al$ has been added.
The point is that the monotype $\tau$, represented by the corresponding existential variable,
is always well-formed under the corresponding declarative context.
In other words, the position of $\al$'s reflects the well-formedness restriction of $\tau$'s.

%% A \le B \mid e\Lto A \mid e\To_{a} \jg \mid \appInfAlg{A}{e}

\paragraph{Judgment Chains} 
Judgment chains $\jg$, or judgments for short, are the core components of our algorithmic
type-checking. There are four kinds
of judgments in our system: subtyping ($A \le B$), checking ($e\Lto
A$), inference ($e\To_{a} \jg$) and
application inference ($\appInfAlg{A}{e}$).  Subtyping and checking are relatively simple,
since their result is only success or failure. However both inference and
application inference return a type that is used in subsequent judgments. We use a
continuation-passing-style encoding to accomplish this. For example, the judgment
chain $e \To_a (a \le B)$ contains two judgments: first we want to
infer the type of the expression $e$, and then check if that type is a
subtype of $B$. The \emph{unknown} type of $e$ is represented by a
type variable $a$, which is used as a placeholder in the second judgment to denote the 
type of $e$.

\paragraph{Worklist Judgments} Our algorithm has a non-standard form.
We combine traditional contexts and judgment(s) into a single sort, the \emph{worklist} $\Gm$.
The worklist is an \emph{ordered} collection of both variable bindings and judgments. The order captures the scope:
only the objects that come after a variable's binding in the worklist can refer to it.
For example, $[\nil, a, x:a \Vdash x \Lto a]$ is a valid worklist,
but $[\nil \Vdash \underline{x} \Lto \underline{a}, x:\underline{a}, a]$ is not
(the underlined symbols refer to out-of-scope variables).

\paragraph{Hole Notation}
We use the syntax $\Gm[\Gm_M]$ to denote the worklist $\Gm_L,\Gm_M,\Gm_R$,
where $\Gm$ is the worklist $\Gm_L,\bullet,\Gm_R$ with a hole ($\bullet$).
Hole notations with the same name implicitly share the same structure $\Gm_L$ and $\Gm_R$.
A multi-hole notation splits the worklist into more parts.
For example, $\Gm[\al][\bt]$ means $\Gm_1,\al,\Gm_2,\bt,\Gm_3$.

\subsection{Algorithmic System}

\newcounter{algRuleCounter}
\newcommand \algrule {\stepcounter{algRuleCounter}\rrule{\arabic{algRuleCounter}}}

% TODO rules too long
\begin{figure}[htp]
\hfill \framebox{$\Gm\rto \Gm'$} \hfill $\Gm$ reduces to $\Gm'$.
%\jimmy{Whenever there is a change on judgment count, marker should be properly added/removed. Decidability: all valid judgment chain should be reduced to nil.}
\begin{gather*}
\begin{aligned}
\Gm, a &\algrule \Gm \qquad
\Gm, \al \algrule \Gm \qquad
\Gm, x:A \algrule \Gm
\\[3mm]
\Gm \Vdash 1\le 1 &\algrule \Gm\\
\Gm \Vdash a\le a &\algrule \Gm\\
\Gm \Vdash \al\le \al &\algrule \Gm\\
\Gm \Vdash A_1\to A_2 \le B_1\to B_2 &\algrule \Gm \Vdash A_2 \le B_2 \Vdash B_1\le A_1\\
\Gm \Vdash \all A\le B &\algrule \Gm,\al \Vdash [\al/a]A\le B \quad\text{when } B \neq \all B'\\
\Gm \Vdash A\le \all[b]B &\algrule \Gm,b \Vdash A\le B
\\[3mm]
%\\
%\text{Let } \color{red}\Gm[\al \toto B // G_M] &:= \Gm_L, G_M, [B/\al]\Gm_R, \text{ when } \Gm[\al] = \Gm_L,\al,\Gm_R \ (G_M\text{ defaults to }\nil)\\
\Gm[\al] \Vdash \al \le A\to B &\algrule [\al[1]\to\al[2]/\al] (\Gm[\al[1], \al[2]] \Vdash \al[1]\to \al[2] \le A \to B)\\
 &\qquad\qquad \text{when }\al\notin FV(A)\cup FV(B)\\
\Gm[\al] \Vdash A\to B \le \al &\algrule [\al[1]\to \al[2]/\al] (\Gm[\al[1], \al[2]] \Vdash A \to B \le \al[1]\to \al[2])\\
 &\qquad\qquad \text{when }\al\notin FV(A)\cup FV(B)
 \\[3mm]
\Gm[\al][\bt] \Vdash \al \le \bt &\algrule [\al/\bt](\Gm[\al][])\\
\Gm[\al][\bt] \Vdash \bt \le \al &\algrule [\al/\bt](\Gm[\al][])\\
\Gm[a][\bt] \Vdash a \le \bt &\algrule [a/\bt](\Gm[a][])\\
\Gm[a][\bt] \Vdash \bt \le a &\algrule [a/\bt](\Gm[a][])\\
\Gm[\bt] \Vdash 1 \le \bt &\algrule [1/\bt](\Gm[])\\
\Gm[\bt] \Vdash \bt \le 1 &\algrule [1/\bt](\Gm[])
\\[3mm]
\Gm \Vdash e \Lto B &\algrule \Gm \Vdash e\To_a a\le B \quad
    \text{when } e \neq \lam e' \text{ and } B \neq \all B'\\
% \Gm \Vdash () \Lto 1 &\algrule \Gm\\
\Gm \Vdash e\Lto \all A &\algrule \Gm,a \Vdash e\Lto A\\
\Gm \Vdash \lam e \Lto A\to B &\algrule \Gm, x:A  \Vdash e \Lto B\\
\Gm[\al] \Vdash \lam e \Lto \al &\algrule [\al[1]\to \al[2] / \al](\Gm[\al[1],\al[2]], x:\al[1] \Vdash e \Lto \al[2])
% \quad\text{\jimmy{Additional}}
\\[3mm]
\Gm \Vdash x\To_a \jg &\algrule \Gm \Vdash [A/a] \jg \quad \text{when } (x:A)\in \Gm\\
\Gm \Vdash (e:A)\To_a \jg &\algrule \Gm \Vdash [A/a]\jg \Vdash e \Lto A\\
\Gm \Vdash ()\To_a \jg &\algrule \Gm \Vdash [1/a]\jg\\
\Gm \Vdash \lam e \To_a \jg &\algrule
    \Gm,\al,\bt \Vdash [\al\to\bt/a]\jg, x:\al \Vdash e\Lto \bt\\
\Gm \Vdash e_1\ e_2 \To_a \jg &\algrule \Gm \Vdash e_1\To_b (\appInfAlg{b}{e_2})
\\[3mm]
\Gm \Vdash \appInfAlg{\all A}{e} &\algrule \Gm,\al \Vdash \appInfAlg{[\al/a]A}{e}\\
\Gm \Vdash \appInfAlg{A\to C}{e} &\algrule \Gm \Vdash [C/a]\jg \Vdash e \Lto A\\
\Gm[\al] \Vdash \appInfAlg{\al}{e} &\algrule
    [\al[1]\to\al[2]/\al](\Gm[\al[1], \al[2]] \Vdash \appInfAlg{\al[1]\to\al[2]}{e})
%	[\al[1]\to\al[2]/\al](\Gm[\al[1], \al[2]] \Vdash [\al[2]/a]\jg \Vdash e\Lto \al[1])\\
% &\color{magenta} \makebox[0pt]{\qquad or} \phantom{{}\rrule{}{}}
\end{aligned}
\end{gather*}
\Description{Algorithmic Typing}
\caption{Algorithmic Typing}\label{fig:alg}
\end{figure}

The algorithmic typing reduction rules, defined in Figure~\ref{fig:alg}, have
the form $\Gm \rto \Gm'$.
% The worklists $\Gm, \Gm'$ contain both variable declarations and judgments.
The reduction process treats the worklist as a stack.  In every step it pops
the first judgment from the worklist for processing and possibly pushes new
judgments onto the worklist.  The syntax $\Gm\redto \Gm'$ denotes multiple
reduction steps. 

$$\inferrule*[right=$\mathtt{{\redto} id}$]
{~}{\Gm\redto\Gm}
\qquad
\inferrule*[right=$\mathtt{{\redto} step}$]
{\Gm \rto \Gm_1 \quad \Gm_1 \redto \Gm'}{\Gm \redto \Gm'}$$

\noindent In the case that $\Gm\redto\nil$ this corresponds to successful type checking.

% \jimmy{The implicit freshness conditions for algorithmic system}
Please note that when a new variable is introduced in the right-hand side worklist $\Gm'$,
we implicitly pick a fresh one,
since the right-hand side can be seen as the premise of the reduction.

Rules 1-3 pop variable declarations that are essentially garbage.
That is variables that are
out of scope for the remaining judgments in the worklist.
All other rules concern a judgment at the front of the worklist. Logically we
can discern 6 groups of rules.

\paragraph{{\bf 1. Algorithmic subtyping}}
We have six subtyping rules (Rules 4-9) that are similar to their
declarative counterparts. For instance, Rule 7 consumes a subtyping
judgment and pushes two back to the worklist.  Rule 8 differs from
declarative Rule $\mathtt{{\le}{\forall}L}$ by introducing an existential
variable $\al$ instead of guessing the monotype $\tau$
instantiation. Each existential variable is later solved to a
monotype $\tau$ with the same context, so the final solution $\tau$ of
$\al$ should be well-formed under $\Gm$.

\paragraph{Worklist Variable Scoping}
Rules 8 and 9 involve variable declarations and demonstrate how our
worklist treats variable scopes. Rule 8 introduces an existential
variable $\al$ that is only visible to the judgment $[\al/a]A \le B$.
Reduction continues until all the subtyping judgments in front of $\al$
are satisfied.  Finally we can safely remove $\al$ since no occurrence
of $\al$ could have leaked into the left part of the worklist.  Moreover,
the algorithm garbage-collects the $\al$
variable at the right time: it leaves the environment immediately
after being unreferenced completely for sure.

\paragraph{Example} Consider the derivation of the subtyping judgment
$(1\to 1)\to 1 \le (\all 1\to 1) \to 1$:
\begin{gather*}
\begin{aligned}
  & \nil\vdash (1\to 1)\to 1 \le (\all 1 \to 1) \to 1\\
\rrule{7}  & \nil \Vdash 1 \le 1 \Vdash \all 1 \to 1 \le 1 \to 1\\
\rrule{8}  & \nil \Vdash 1 \le 1 ,\al \Vdash 1 \to 1 \le 1 \to 1\\
\rrule{7}  & \nil \Vdash 1 \le 1 ,\al \Vdash 1 \le 1 \Vdash 1 \le 1\\
\rrule{4} & \nil \Vdash 1 \le 1 ,\al \Vdash 1 \le 1\\
\rrule{4} & \nil \Vdash 1 \le 1 ,\al\\
\rrule{2} & \nil \Vdash 1 \le 1\\
\rrule{4} & \nil
\end{aligned}
\end{gather*}
First, the subtyping of two function types is split into two judgments by Rule 7:
a covariant subtyping on the return type and a contravariant subtyping on the argument type.
Then we apply Rule 8 to reduce the $\forall$ quantifier on the left side.
The rule introduces an existential variable $\al$ to the context (even though 
the type $\all 1 \to 1$ does not actually refer to the quantified type
variable $a$).
In the following 3 steps we satisfy the judgment $1 \to 1 \le 1 \to 1$ by Rules 7, 4 and 4.

Now the existential variable $\al$, introduced before but still unsolved,
is at the top of the worklist and Rule 2 garbage-collects it.
The process is carefully designed within the algorithmic rules:
when $\al$ is introduced earlier by Rule 8,
we foresee the recycling of $\al$ after all the judgments (potentially)
requiring $\al$ have been processed.
Finally Rule 4 reduces one of the base cases and finishes the subtyping derivation.

\paragraph{\bf 2. Existential decomposition.}
Rules 10 and 11 are algorithmic versions of Rule $\mathtt{{\le}{\to}}$; they
both partially instantiate $\al$ to function types.
The domain $\al[1]$ and range $\al[2]$ of the new function type are not determined:
they are fresh existential variables with the same scope as $\al$.
We replace $\al$ in the worklist with  $\al[1], \al[2]$.
To propagate the instantiation to the rest of the worklist and maintain well-formedness,
every reference to $\al$ is replaced by $\al[1] \to \al[2]$.
The \emph{occurs-check} condition prevents divergence as usual.
For example, without it $\al \le 1 \to \al$ would diverge.

\paragraph{\bf 3. Solving existentials} Rules 12-17 are base cases where an existential variable is solved.
They all remove an existential variable and substitute the
variable with its solution in the remaining worklist. Importantly the rules
respect the scope of existential variables. For example, Rule 12 
states that an existential variable $\al$ can be solved with another
existential variable $\bt$ only if $\bt$ occurs after $\al$.

One may notice that the subtyping relation for simple types is just equivalence,
which is true according to the declarative system.
The DK's system works in a similar way.

\paragraph{\bf 4. Checking judgments.}
Rules 18-21 deal with checking judgments.
Rule 18 is similar to $\mathtt{DeclSub}$, but rewritten in the
continuation-passing-style.
The side conditions $e \neq \lam e'$ and $B \neq \all B'$ 
prevent overlap with Rules 19, 20 and 21;
this is further discussed at the end of this section.
% \bruno{The side condition needs to be explained, and some mention of
%   the overlapping is probably needed here.}
Rules 19 and 20 adapt their declarative counterparts to the worklist style.
Rule 21 is a special case of $\mathtt{Decl\to I}$,
dealing with the case when the input type is an existential variable,
representing a monotype \emph{function} as in the declarative system
(it must be a function type, since the expression $\lam e$ is a function).
% \bruno{Rule 21
%   deserves some more explanation. What is the motivation for this
%   rule? You may want to give some concrete example to motivate it.}
The same instantiation technique as in Rules 10 and 11 applies.
The declarative checking rule $\mathtt{Decl1I}$ does not have a direct counterpart in the algorithm, 
because Rules 18 and 24 can be combined to give the same result.

\paragraph{Rule 21 Design Choice}
The addition of Rule 21 is a design choice we have made to simplify the side
condition of Rule 18 (which avoids overlap). It also streamlines the algorithm
and the metatheory as we now treat all cases
where we can see that an existential variable should be instantiated to a
function type  (i.e., Rules 10, 11, 21 and 29) uniformly.

The alternative would have been to omit Rule 21 and drop the condition on $e$
in Rule 18. The modified Rule 18 would then handle $\Gm \Vdash \lam e \Lto \al$
and yield $\Gm \Vdash \lam e \To_a a \le \al$, which would be further processed
by Rule 25 to $\Gm, \bt[1], \bt[2] \Vdash \bt[1] \to \bt[2] \le \al, x:\bt[1] \Vdash e \Lto \bt[2]$.
As a subtyping constraint between monotypes is simply equality, $\bt[1] \to \bt[2] \le \al$
must end up equating $\bt[1] \to \bt[2]$ with $\al$ and thus have the same effect as Rule 21, but in a more roundabout
fashion.

In comparison,
DK's algorithmic subsumption rule has no restriction on the expression $e$,
and they do not have a rule that explicitly handles the case $\lam e \Lto \al$.
Therefore the only way to check a lambda function against an existential variable
is by applying the subsumption rule, which further breaks into
type inference of a lambda function and a subtyping judgment.
% A combination of Rules $18'$ (Rule 18 without the restriction on $e$) and 25 and further derivations
% may result in a similar effect to Rule 21.
% Rule 21 guesses a monotype for the lambda function $\lam e$, so as Rule 25.
% The only difference is that a subtyping relation is introduced by Rule $18'$,
% and is to be processed after the type inference:
% $$\Gm \Vdash \lam e \Lto \al \rrule{18'} \Gm \Vdash \lam e \To_a a \le \al
% \rrule{25} \Gm, \bt[1], \bt[2] \Vdash \bt[1] \to \bt[2] \le \al, x:\bt[1] \Vdash e \Lto \bt[2]$$
% Since two monotypes with a valid subtyping relation indicates equality,
% eventually $\bt[1] \to \bt[2]$ and $\al$ should be unified with each other.
% DK's algorithm takes a similar approach to such variant,
% and we believe that they are equivalent.
% 
% Choosing Rule 21 simplifies the side condition on Rule 18,
% the metatheory and the design: whenever an input type could be a function type,
% the algorithmic rule should always take care of
% the possibility when the type being a single existential variable.
% This pattern applies to all the possible cases, including Rules 10, 11, 21 and 29.

\paragraph{\bf 5. Inference judgments.}
Inference judgments behave differently compared with subtyping and checking judgments:
they \emph{return} a type instead of only accepting or rejecting.
For the algorithmic system, where guesses are involved,
it may happen that the output type of an inference judgment refers to new existential variables,
such as Rule 25.
In comparison to Rule 8 and 9, where new variables are only referred by the sub-derivation,
Rule 25 introduces variables $\al, \bt$ that affect the remaining judgment chain.
This rule is carefully designed so that the output variables are bound by earlier declarations,
thus the well-formedness of the worklist is preserved,
and the garbage will be collected at the correct time.
By making use of the continuation-passing-style judgment chain,
inner judgments always share the context with their parent judgment.

\begin{comment}
Old text:
The design of our judgment chain is closely related to the shape of
the judgments
\jimmy{requires further clarification}
\bruno{What property? Not very clear}.
Subtyping and checking do not return anything, so variables cannot
leak anyway, as applied to Rules 8 and 9.
\bruno{Is the discussion that follows in the right place? We just jump
  to
  rule 26. Perhaps we can wait until we talk about inference to
  discuss those issues?}
However, inference and application inference may return a type that contains new variables.
Take Rule 26 as an example, if it simply returns $\al \to \bt$,
passes that to the next judgment and continues the type-checking process,
variables $\al$ and $\bt$ in the next judgment are out of scope
and will break the well-formedness of the worklist.
\end{comment}

Rules 22-26 deal with type inference judgments, written in continuation-passing-style.
When an inference judgment succeeds with type $A$,
the algorithm continues to work on the inner-chain $\jg$ by
assigning $A$ to its placeholder variable $a$.
Rule 23 infers an annotated expression by changing into checking mode,
therefore another judgment chain is created.
Rule 24 is a base case,
where the unit type $1$ is inferred and thus passed to its child judgment chain.
Rule 26 infers the type of an application by
firstly inferring the type of the function $e_1$,
and then leaving the rest work to an application inference judgment,
which passes $a$, representing the return type of the application,
to the remainder of the judgment chain $\jg$.

Rule 25 infers the type of a lambda expression by introducing $\al, \bt$
as the input and output types of the function, respectively.
After checking the body $e$ under the assumption $x:\al$,
the return type might reflect more information than simply $\al \to \bt$
through propagation when existential variables are solved or partially solved.
The variable scopes are maintained during the process:
the assumption of argument type ($x:\al$) is recycled after checking against the function body;
the existential variables used by the function type ($\al,\bt$) are only visible in the remaining chain $[\al\to\bt/a]\jg$.
The recycling process of Rule 25 differs from DK's corresponding rule significantly,
and we further discuss the differences in Section~\ref{sec:discussion:scoping}.

% \bruno{You have some note ``differs'' on the rules. Should that
%   deserve an explanation here?}

\paragraph{\bf 6. Application inference judgments}
Finally, Rules 27-29 deal with application inference judgments.
Rules 27 and 28 behaves similarly to declarative rules $\mathtt{Decl\forall App}$ and $\mathtt{Decl\to App}$.
Rule 29 instantiates $\al$ to the function type $\al[1] \to \al[2]$, just like Rules 10, 11 and 21.

\paragraph{Example}
Figure~\ref{fig:alg:sample} shows a sample derivation of the algorithm.
It checks the application $(\lam x)~()$ against the unit type.
According to the algorithm, we apply Rule 18 (subsumption), changing to inference mode.
Type inference of the application breaks into two steps by Rule 26:
first we infer the type of the function,
and then the application inference judgment helps to determine the return type.
In the following 5 steps the type of the identity function, $\lam x$, is inferred to be $\al \to \al$:
checking the body of the lambda function (Rule 25),
switching from check mode to inference mode (Rule 18),
inferring the type of a term variable (Rule 22),
solving a subtyping between existential variables (Rule 12) and
garbage collecting the term variable $x$ (Rule 3).

After that, Rule 28 changes the application inference judgment to
a check of the argument against the input type $\al$ and returns the output type $\al$.
Checking $()$ against the existential variable $\al$ solves $\al$ to the unit type $1$
through Rules 18, 24 and 16.
Immediately after $\al$ is solved, the algorithm replaces every occurrence of $\al$ with $1$.
Therefore the worklist remains $1 \le 1$, which is finished off by Rule 4.
Finally, the empty worklist indicates the success of the whole derivation.

\begin{figure}
\begin{gather*}
\begin{aligned}
           & \nil \Vdash (\lam x)~() \Lto 1\\
\rrule{18} & \nil \Vdash (\lam x)~() \To_a a\le 1\\
\rrule{26} & \nil \Vdash (\lam x) \To_b (\appInfAlg{b}{()}[a][a\le 1])\\
\rrule{25} & \nil,\al,\bt \Vdash \appInfAlg{\al\to\bt}{()}[a][a\le 1], x:\al \Vdash x\Lto \bt\\
\rrule{18} & \nil,\al,\bt \Vdash \appInfAlg{\al\to\bt}{()}[a][a\le 1], x:\al \Vdash x\To_b b\le \bt\\
\rrule{22} & \nil,\al,\bt \Vdash \appInfAlg{\al\to\bt}{()}[a][a\le 1], x:\al \Vdash \al\le \bt\\
\rrule{12} & \nil,\al \Vdash \appInfAlg{\al\to\al}{()}[a][a\le 1], x:\al\\
\rrule{3}  & \nil,\al \Vdash \appInfAlg{\al\to\al}{()}[a][a\le 1]\\
\rrule{28} & \nil,\al \Vdash \al\le 1 \Vdash () \Lto \al\\
\rrule{18} & \nil,\al \Vdash \al\le 1 \Vdash () \To_a a\le \al\\
\rrule{24} & \nil,\al \Vdash \al\le 1 \Vdash 1\le \al\\
\rrule{16} & \nil \Vdash 1\le 1\\
\rrule{4}  & \nil
\end{aligned}
\end{gather*}
\Description{A Sample Derivation for Algorithmic Typing}
\caption{A Sample Derivation for Algorithmic Typing}
\label{fig:alg:sample}
\end{figure}

In summary, our type checking algorithm accepts $(\lam x)~() \Lto 1$.

\paragraph{Non-overlapping and Deterministic Reduction}
An important feature of our algorithmic rules is that they are directly
implementable. Indeed, although written in the form of reduction rules, they do
not overlap and are thus deterministic.
%This has the added benefit that our % INCORRECT!
%algorithm runs in polynomial time, which makes it suitable for practical use.

Consider in particular Rules 8 and 9, which correspond to the declarative rules
$\mathtt{{\le}\forall L}$ and $\mathtt{{\le}\forall R}$. While those
declarative rules both match the goal $\all A\le \all[b]B$,
we have eliminated this overlap in the algorithm by restricting Rule 8
($B\neq\all B'$) and thus always applying Rule 9 to $\all A\le \all[b]B$.

Similarly, the declarative rule $\mathtt{DeclSub}$ overlaps highly with the
other checking rules. Its algorithmic counterpart is Rule 18. Yet, we have
avoided the overlap with other algorithmic checking rules by adding
side-conditions to Rule 18, namely $e\neq\lam e'$ and $B\neq\all B'$.

These restrictions have not been imposed arbitrarily:
we formally prove that the restricted algorithm is still complete.
In Section~\ref{sec:metatheory:non-overlapping} we discuss the relevant metatheory,
with the help of a non-overlapping version of the declarative system.



\section{Metatheory}

In this section we present several properties formally verified.
For the declarative system, the typing subsumption and subtyping
transitivity lemmas are discussed in detail.
The algorithmic system is proven to be sound with respect
to the declarative system via a transfer relation.
A partial completeness theorem is shown under the rank-1 restriction.
We then briefly describe the challenges we face when proving termination.
Lastly, proof statistics of Abella are discussed.

\subsection{Declarative Properties}\label{sec:meta:decl}

\paragraph{The Typing Subsumption Lemma.}
An important desired property for a type system is \emph{checking subsumption},
which basically says that any expression can
check against any super type of its actual type.
Since our bidirectional type system defines the checking mode, inference mode and
application inference mode mutually,
we formalize the generalized \emph{typing subsumption}.

First of all, we give the definition of worklist subtyping,
which is used to further generalize the typing subsumption lemma.
This is necessary because rules like $\mathtt{Decl{\to}I}$
will push the argument type $A$ into the context,
thus when checking against a super type of $A \to B$, say $C \to D$,
will cause the bind of $x$ in the context to a subtype of $A$ (since $C \le A$).

\begin{definition}[Worklist Subtyping]
    Worklist subtyping compares the type of variables bound in the worklist.
    $\Psi <: \Psi'$ iff each binding in $\Psi$ is converted to one with a super type.
    \begin{gather*}
        \inferrule*[right=$\mathtt{<:nil}$]
            {~}{\cdot <: \cdot}
        \qquad
        \inferrule*[right=$\mathtt{<:ty}$]
            {\Psi <: \Psi'}
            {\Psi, a <: \Psi', a}
        \\
        \inferrule*[right=$\mathtt{<:of}$]
            {\Psi' \vdash A \le B \\ \Psi <: \Psi'}
            {\Psi, x:A <: \Psi', x:B}
        \qquad
        \inferrule*[right=$\mathtt{<:\omega}$]
            {\Psi <: \Psi'}
            {\Psi \Vdash \jg <: \Psi' \Vdash \jg}
    \end{gather*}
\end{definition}

A basic property of worklist subtyping is that they acts similarly when
dealing with subtyping between well-formed types.
\begin{lemma}[Worklist Subtyping Equivalence]~\\
    Given $\Psi <: \Psi'$, $\Psi \vdash A \le B \Longleftrightarrow \Psi' \vdash A \le B$.
\end{lemma}

Finally, we give the statement of typing subsumption lemma,
which is generalized by the worklist subtyping relation.

\begin{lemma}[Typing Subsumption]
    Given $\Psi <: \Psi'$,
    \begin{enumerate}[1)]
        \item If $\Psi' \vdash e \Lto A$ and $\Psi' \vdash A \le B$, then $\Psi \vdash e \Lto B$;
        \item If $\Psi' \vdash e \To A$, then $\exists B$ s.t. $\Psi' \vdash B \le A$ and $\Psi \vdash e \To B$.
        \item If $\Psi' \vdash \appInf{C}{e}{A}$ and $\Psi' \vdash D \le C$, then
            $\exists B$ s.t. $\Psi' \vdash B \le A$ and $\Psi \vdash \appInf{D}{e}{B}$.
    \end{enumerate}
\end{lemma}

\begin{proof}
    By induction on the following size measure (lexicographical order on a 3-tuple):
    \begin{itemize}
        \item Checking ($e \Lto A$): $\langle |e|, 1, |A|_\forall + |B|_\forall \rangle$
        \item Inference ($e \To A$): $\langle |e|, 0, 0 \rangle$
        \item Application inference ($\appInf{A}{e}{C}$): $\langle |e|, 2, |C|_\forall + |D|_\forall \rangle$
    \end{itemize}
    Most of the cases are straightforward.
    When rule $\mathtt{{\le}\forall L}$ is applied for the subtyping predicate
    like $\Psi' \vdash A \le B$,
    a mono-type substitution is performed on $\all A$,
    resulting in $[\tau/a]A$.
    Since $\tau$ is a mono-type, the result type reduces the number of $\forall$'s,
    and thus reduces the size measure.
\end{proof}

Interestingly, the two new declarative rules
$\mathtt{Decl\top}$ and $\mathtt{Decl{\bot}App}$ are
discovered when we were trying to prove the property instead of
before exploring the meta-theory.
Given the typing and subtyping judgments $\Psi \vdash e \Lto A$ and $\Psi \vdash A \le \top$,
we should derive $\Psi \vdash e \Lto \top$ from the lemma,
therefore Rule $\mathtt{Decl\top}$ is required,
saying that any expression can be checked against the top type.
Similarly, the most general type $\bot$,
being able to convert to any type due to Rule $\mathtt{{\le}Bot}$,
can be converted to any function type,
or simply the most general one $\top \to \bot$,
which accepts any input and returns the $\bot$ type,
resulting in the derivation $\Psi \vdash \appInf{\bot}{e}{\bot}$.
From the lemma we can also derive that by
$\Psi \vdash \appInf{C}{e}{A}$, $\Psi \vdash \bot \le C$ and $\Psi \vdash \bot \le A$.

With the addition of Rules $\mathtt{Decl\top}$ and $\mathtt{Decl{\bot}App}$,
we can prove the typing subsumption lemma.
To the best of the authors' knowledge,
they are the minimal set of rules that make the lemma hold.


\paragraph{Transitivity of Subtyping}

The transitivity lemma for declarative subtyping is a commonly expected property.
The proof depends on the following subtyping derivation size relation and an auxiliary lemma.

\begin{definition}[Subtyping Derivation Size]
    \begin{gather*}
        \begin{aligned}
            |1 \le 1| &= 0\\
            |a \le a| &= 0\\
            |A \le \top| &= 0\\
            |\bot \le B| &= 0\\
            |A_1 \to A_2 \le B_1 \to B_2| &= |B_1 \le A_1| + |A_2 \le B_2| + 1\\
            |\all A \le B| &= |[\tau/a]A \le B| + 1\\
            |A \le \all B| &= |A \le B| + 1
        \end{aligned}
    \end{gather*}
\end{definition}

\begin{lemma}[Monotype Subtyping Substitution]
    If $\Psi \vdash \tau$ and $\Psi, a, \Psi_R \vdash A \le B$, then
    $\Psi, [\tau/a]\Psi_R \vdash [\tau/a]A \le [\tau/a]B$.
\end{lemma}

\begin{proof}
    A routine induction on the subtyping relation $\Psi, a, \Psi_R \vdash A \le B$
    finishes the proof.
\end{proof}

\begin{corollary}[Monotype Subtyping Substitution for Type Variables]
    \label{cor:subtyping_subst_mono}
    If $\Psi \vdash \tau$ and $\Psi, a \vdash A \le B$, then
    $\Psi \vdash [\tau/a]A \le [\tau/a]B$.
\end{corollary}

The above lemma and corollary reveal the fact
that a type variable occured in the subtyping relation
represents \emph{arbitrary} well-formed monotype.
And it also explains the difference in treatment of polymorphic types between
Rules $\mathtt{{\le}{\forall}L}$ and $\mathtt{{\le}{\forall}R}$:
Rule $\mathtt{{\le}{\forall}R}$ is in fact equivalent to:
$$
\inferrule*[right=$\mathtt{{\le}{\forall}R'}$]
    {\forall \tau \text{ s.t. } \Psi \vdash \tau \Longrightarrow \Psi \vdash A \le [\tau/b]B}
    {\Psi \vdash A \le \all[b] B}
$$

Finally, with the size measure defined and required lemma proven,
we can obtain the transitivity lemma for declarative subtyping.

\begin{lemma}[Subtyping Transitivity]
    If $\Psi \vdash A \le B$ and $\Psi \vdash B \le C$ then
    $\Psi \vdash A \le C$.
\end{lemma}

\begin{proof}
    Induction on the lexicographical order defined by
    $\langle |B|_\forall, |A \le B| + |B \le C| \rangle$.
    Most cases preserve the first element of the size measures $|B|_\forall$,
    and are relatively easy to prove.
    The difficult case is when $B$ is a polymorphic type,
    when the conditions are $\Psi \vdash A \le \all B$ and $\Psi \vdash \all B \le C$.
    They are derived through rules $\mathtt{{\le}\forall L}$ and
    $\mathtt{{\le}\forall R}$, respectively.
    Therefore, we have $\Psi, a \vdash A \le B$ and $\Psi \vdash [\tau/a] B \le C$.
    To exploit the induction hypothesis, the contexts should be unified.
    By Corollary~\ref{cor:subtyping_subst_mono}, $\Psi \vdash A \le [\tau/a]B$.
    Notice that the freshness condition is implicit for rule $\mathtt{{\le}\forall L}$.
    Clearly, $|[\tau/a]B|_\forall < |\all B|_\forall$, i.e. the first size measure decreases.
    By induction hypothesis we get $\Psi \vdash A \le C$ and finishes this case.
\end{proof}

\subsection{Transfer}

\begin{figure}[t]
    \begin{gather*}
    \begin{aligned}
    \text{Declarative worklist}\qquad&\Om &::=&\quad \nil \mid \Om, a \mid \Om, x: A \mid \Om \Vdash \jg
    \end{aligned}
    \end{gather*}
    \hfill \framebox{$\Gm \sto \Om$} \hfill $\Gm$ instantiates to $\Om$.
    \begin{gather*}
    \inferrule*[right=$\mathtt{{\sto}}\Om$]
    {~}
    {\Om \sto \Om}
    \quad
    \inferrule*[right=$\mathtt{{\sto}\al}$]
    {\Om\vdash\tau \\ \Om,[\tau/\al]\Gm \sto \Om}
    {\Om,\al,\Gm \sto \Om}
    \end{gather*}
    \Description{Declarative Worklists and Instantiation}
    \caption{Declarative Worklists and Instantiation}
    \label{fig:top:trans}
\end{figure}

Follow the approach of Section~\ref{sec:metatheory},
the transfer relation and the declarative instantiation relation are defined
in Figure~\ref{fig:top:trans}.

Similarly, Lemmas~\ref{lem:top:insert} and \ref{lem:top:extract}
generalizing Rule $\mathtt{{\sto}\al}$ hold as well.

\begin{lemma}[Insert]\label{lem:top:insert}
If $\Gm_L, [\tau/\al]\Gm_R \sto \Om$ and $\Gm_L\vdash \tau$
, then $\Gm_L, \al, \Gm_R \sto \Om$.
\end{lemma}
\begin{lemma}[Extract]\label{lem:top:extract}
If $\Gm_L, \al, \Gm_R \sto \Om$
, then there exists $\tau$ s.t. $\Gm_L\vdash\tau$ and $\Gm_L, [\tau/\al]\Gm_R \sto \Om$.
\end{lemma}

\begin{figure}[ht]
\hfill \framebox{$\|\Om\|$} \hfill Judgment erasure.
\begin{gather*}
\begin{aligned}
\|\nil\| &= \nil\\
\|\Om,a\| &= \|\Om\|, a\\
\|\Om,x:A\| &= \|\Om\|, x:A\\
\|\Om\Vdash\jg\| &= \|\Om\|
\end{aligned}
\end{gather*}

\hfill \framebox{$\Om \rto \Om'$} \hfill Declarative transfer.
\begin{gather*}
\begin{aligned}
\Om,a &\rto \Om \\  \Om,x:A & \rto \Om\\
\Om\Vdash A\le B &\rto \Om &\text{ when } \|\Om\| \vdash A\le B\\
\Om\Vdash e\Lto A &\rto \Om & \text{ when } \|\Om\| \vdash e\Lto A\\
\Om\Vdash e\To_a \jg &\rto \Om\Vdash[A/a]\jg & \text{ when } \|\Om\| \vdash e\To A\\
\Om\Vdash \appInfAlg{A}{e} &\rto \Om\Vdash[C/a]\jg & \text{ when } \|\Om\| \vdash \appInf{A}{e}{C}\\
\end{aligned}
\end{gather*}
\Description{Declarative Transfer}
\caption{Declarative Transfer}
\label{fig:top:decl:worklist}
\end{figure}

Figure~\ref{fig:top:decl:worklist} defines a relation $\Om \rto \Om'$,
checking that every judgment entry in the worklist
holds using a corresponding declarative judgment.

\subsection{Soundness}

Our algorithm is sound with respect to the declarative system.
For any worklist $\Gm$ that reduces successfully,
there is a valid instantiation $\Om$ that transfers all judgments
to the declarative system.
\begin{theorem}[Soundness]
If \emph{wf }$\Gm$ and $\Gm \redto \nil$,
then there exists $\Om$ s.t. $\Gm \sto \Om$ and $\Om \redto \nil$.
\end{theorem}

Soundness is a basic desired property of a type inference algorithm,
which ensures that the algorithm is always producing
valid declarative derivations when the judgments are accepted.

\subsection{Partial Completeness of Subtyping: Rank-1 Restriction}

The algorithm is incomplete due to the subtyping rules 14, 15, 20 and 21.
However, subtyping is complete with respect to the declarative system in a rank-1 setting.

\paragraph{Declarative Rank-1 Restriction}

Rank-1 types are also named type schemes in Hindley-Milner type system.
$$\begin{aligned}
    \text{Declarative Type Schemes}\qquad&\sigma &::=&\quad \all \sigma \mid \tau\\
\end{aligned}$$
In other words, the universal quantifiers only appear in the top level
of all polymorphic types.

For declarative subtyping, a judgment must be of form $\sigma_1 \le \sigma_2$.

\subsection{Algorithmic Rank-1 Restriction (Partial Completeness)}

The algorithmic mono-types and type schemes are defined as following:
$$\begin{aligned}
    \text{Algorithmic Mono-types}\qquad&\tau_A &::=&\quad
        1 \mid \top \mid \bot \mid a \mid A\to B \mid \al\\
    \text{Algorithmic Type Schemes}\qquad&\sigma_A &::=&\quad \all \sigma_A \mid \tau_A\\
\end{aligned}$$

Starting from the declarative judgment $\sigma_1 \le \sigma_2$,
the algorithmic derivation might involve different other kinds of judgments.
The following derivation, as an example, shows how a rank-1 judgment derives.

$$\begin{aligned}
           & \cdot \Vdash \all a \to a \le \all[b] (b \to b) \to (b \to b)\\
    \rrule{8} & b \Vdash \all a \to a \le (b \to b) \to (b \to b)\\
    \rrule{7} & b, \al \Vdash \al \to \al \le (b \to b) \to (b \to b)\\
    \rrule{6} & b, \al \Vdash \al \le b \to b \Vdash b \to b \le \al\\
    \rrule{} & \cdots
\end{aligned}$$

In this derivation, we begin from a judgment of the form $\sigma \le \sigma$.
After rule 8 is applied, the judgment becomes $\sigma \le \tau$,
since the right-hand-side polymorphic type is reduced to a declarative mono-type.
Then, rule 7 introduces existential variables to the left-hand-side,
resulting in a judgment like $\tau_A \le \tau$,
or $\sigma_A \le \tau$ in a more general case.
Finally, rule 6 breaks a judgment between functions into two sub-judgments,
which swaps the positions of the argument types
and creates a judgment like $\tau \le \tau_A$.
Notice that $\sigma_A$ is not possible to occur to the right
because the function type may not contain any polymorphic types as its argument type.

After a detailed analysis on the judgments derivations,
we found that the only possible judgments that a rank-1 declarative subtyping judgment
might step to belong to the following two categories:
$$\sigma_A \le \sigma \quad\text{or}\quad \tau \le \sigma_A$$
All the possible judgment types shown above fall into these categories.
For example, $\tau_A \le \tau$ is a special form of $\sigma_A \le \sigma$,
and $\tau \le \tau_A$ belongs to $\tau \le \sigma_A$.

An interesting observation is that $\al \le \bt$ does not belong to either category,
neither does $\al \le A \to B$ when $\al \in \text{FV}(A \to B)$.
Therefore, in the rank-1 setting, both cases of incompleteness never occur,
and our algorithm is complete.

\begin{theorem}[Completeness of Rank-1 Subtyping]
    Given $\Psi \vdash \sigma_1 \le \sigma_2$,
    \begin{itemize}
        \item If $\Gm \Vdash \sigma_A \le \sigma \sto \Psi \Vdash \sigma_1 \le \sigma_2$
            \\then $\Gm \Vdash \sigma_A \le \sigma \redto \nil$;
        \item If $\Gm \Vdash \tau \le \sigma_A \sto \Psi \Vdash \sigma_1 \le \sigma_2$
            \\then $\Gm \Vdash \tau \le \sigma_A \redto \nil$.
    \end{itemize}
\end{theorem}

\subsection{Termination}

The measure used in Chapter~\ref{chap:ICFP} no longer works because subtyping judgments like
$$\al \le \bot \to \top$$
cause $\al$ to split into $\al[1] \to \al[2]$, without solving any part of it,
resulting in an increased number of existential variables
and possibly increased complexity of the worklist through the size-increasing substitution
$\{\al := \al[1] \to \al[2]\}$.

We have performed a large set of tests on generated subtyping judgments
that are consist of algorithmic mono types,
and all judgements terminated within a reasonable number of derivation depth.
Unfortunately, we have not yet find any formal proof for the termination statement.

\subsection{Formalization in the Abella Proof Assistant}

We have chosen the Abella (v2.0.7-dev
\footnote{We use a forked version \url{https://github.com/JimmyZJX/abella} by only enhancing the Abella prover with a handy
``applys'' tactic.}
) proof assistant~\citep{AbellaDesc} to develop our formalization.
Equipped with HOAS, Abella ease the formalization and proof tasks a lot
compared with various libraries in Coq.
Additionally, our algorithm heavily use eager substitutions,
and Abella greatly simplifies relavent proofs thanks to its built-in
substitution representation and higher-order unification algorithms.

\paragraph{Statistics of the Proof}
The proof script consists of 7,301 lines of Abella code with a total of
48 definitions and 592 theorems.
Figure~\ref{table:top:proof_statistics} briefly summarizes the contents of each file.
The files are linearly dependent due to limitations of Abella.

\begin{table}[t]
    \renewcommand{\arraystretch}{1.2}
    \caption{Statistics for the proof scripts}
    \centering\begin{tabular}{@{}lrrl@{}}
    \toprule
        File(s) & LOC & \#Thm & Description\\
    \midrule
        olist.thm, nat.thm  &   311 & 57  & Basic data structures\\
        typing.thm          &   273 & 7   & Declarative \& algorithmic system, debug examples\\
        decl.thm            &   241 & 33  & Basic declarative properties\\
        order.thm           &   274 & 27  & The $|\cdot|_\forall$ measure; decl. subtyping strengthening\\
        alg.thm             &   699 & 82  & Basic algorithmic properties\\
        trans.thm           &   635 & 53  & \makecell[l]{Worklist instantiation and declarative transfer;\\
                                Lemmas~\ref{lem:insert}, \ref{lem:extract}}\\
        declTyping.thm      & 1,087 & 76  & \makecell[l]{Non-overlapping declarative system; \\
                                Lemmas~\ref{lem:inv_allR}, \ref{lem:inv_chkAll},
                                    \ref{lem:inv_chkLam}, \ref{lem:subsumption}}\\
        soundness.thm       & 1,206 & 81  & Soundness theorem; aux. lemmas on transfer\\
        dcl.thm             &   417 & 12  & Non-overlapping declarative worklist \\
        scheme.thm          & 1,113 & 98  &
                                Type scheme (rank-1 restriction)\\
        completeness.thm    & 1,045 & 63  &
                                Completeness theorem; aux. lemmas and relations\\
    \midrule
        \emph{Total}        & 7,301 & 592 & (48 definitions in total)\\
    \bottomrule
    \end{tabular}
    \label{table:top:proof_statistics}
\end{table}




\section{Discussion}

% \subsection{Overlapping Rules} Already discussed in previous sections

This section discusses some insights that we gained from our work and contrasts
the scoping mechanisms we have employed with those in DK's algorithm.
We also discuss a way to improve the precision of their scope tracking.
Furthermore we discuss and sketch an extension of our algorithm with
an elaboration to a target calculus, and discuss an extension of our algorithm
with scoped type variables~\citep{scoped-type-variables}.

\begin{comment}
\subsection{Implementation}
Anything to say about the implementation? Do we have one?
\end{comment}

\subsection{Contrasting Our Scoping Mechanisms with DK's}\label{sec:discussion:scoping}

A nice feature of our worklists is that, simply by interleaving variable declarations and
judgment chains, they make the scope of variables
precise.  DK's algorithm deals with garbage collecting variables in a
different way: through type variable or existential variable
markers (as discussed in Section~\ref{ssec:DK_Algorithm}).  Despite
the sophistication employed in DK's algorithm to keep scoping precise,
there is still a chance that unused existential variables leak their
scope to an output context and accumulate indefinitely.
For example, the derivation of the judgment $(\lam x)~() \Lto 1$ is as follows
$$
\inferrule*
{
    \inferrule*
    {
        \inferrule*
        {
            \inferrule*
            {\ldots x \To \al \ldots \\ \ldots \al \le \bt \ldots}
            {\Gm, \al, \bt, x:\al \vdash x \Lto \bt \dashv \Gm_1, x:\al}
        }
        {\Gm \vdash \lam x \To \al \to \bt \dashv \Gm_1}
        \\
        \inferrule*
        {\ldots () \Lto \al \ldots}
        {\Gm_1 \vdash \appInf{\al \to \al}{()}{\al} \dashv \Gm_2}
    }
    {\Gm \vdash (\lam x)~() \To \al \dashv \Gm_2}
    \\
    \inferrule*
    {~}
    {\Gm_2 \vdash 1 \le 1 \dashv \Gm_2}
}
{\Gm \vdash (\lam x)~() \Lto 1 \dashv \Gm, \al = 1, \bt = \al}
$$
where $\Gm_1 := (\Gm, \al, \bt = \al$) solves $\bt$,
and $\Gm_2 := (\Gm, \al = 1, \bt = \al$) solves both $\al$ and $\bt$.

If the reader is not familiar with DK's algorithm,
he/she might be confused about the inconsistent types across judgment.
As an example, $(\lam x)~()$ synthesizes $\al$,
but the second premise of the subsumption rule uses $1$ for the result.
This is because a context application $[\Gm,\al=1,\bt=\al]\al = 1$ happens between the premises.

The existential variables $\al$ and $\bt$ are clearly not used after the subsumption rule,
but according to the algorithm, they are kept in the context
until some parent judgment recycles a block of variables,
or to the very end of a type inference task.
In that sense, DK's algorithm does not control the scoping of variables precisely.
%However, it is a minor issue that does not affect the soundness and completeness properties.

Two rules we may blame for not garbage collecting correctly are the inference rule for lambda
functions and an application inference rule:
$$
\inferrule*[right=$\mathtt{DK\_{\to}I{\To}}$]
{\Gamma, \al, \bt, x:\al \vdash e \Lto \bt \dashv \Delta, x:\al, \Theta}
{\Gamma \vdash \lam e \To \al \to \bt \dashv \Delta}
\qquad
\inferrule*[right=$\mathtt{DK\_\forall App}$]
{\Gamma, \al \vdash [\al/a] \appInf{A}{e}{C} \dashv \Delta}
{\Gamma \vdash \appInf{\all A}{e}{C} \dashv \Delta}
$$
In contrast, Rule 25 of our algorithm collects the existential variables
right after the second judgment chain,
and Rule 27 collects one existential variable similarly:
\[\Gm \Vdash \lam e \To_a \jg \rrule{25}
\Gm,\al,\bt \Vdash [\al\to\bt/a]\jg, x:\al \Vdash e\Lto \bt\]
\[\Gm \Vdash \appInfAlg{\all A}{e} \rrule{27} \Gm,\al \Vdash \appInfAlg{[\al/a]A}{e}\]
It seems impossible to achieve a similar outcome in DK's system by only modifying these two rules.
Taking $\mathtt{DK\_{\to}I{\To}}$ as an example,
the declaration or solution for $\al$ and $\bt$ may be referred to by subsequent judgments.
Therefore leaving $\al$ and $\bt$ in the output context is the only choice,
when the subsequent judgments cannot be consulted.

To fix the problem, one possible modification is on the algorithmic subsumption rule,
as garbage collection for a checking judgment is always safe:
$$
\inferrule*[right=$\mathtt{DK\_Sub}$]
{\Gamma, \blacktriangleright_{\al} \vdash e \To A \dashv \Theta \\
\Theta \vdash [\Theta]A \le [\Theta]B \dashv \Delta, \blacktriangleright_{\al}, \Delta'}
{\Gamma \vdash e \Lto B \dashv \Delta}
$$
Here we employ the markers in a way they were originally not intended for.
We create a dummy fresh existential $\al$ and add a marker to the input context of the inference judgment.
After the subtyping judgment is processed we look for the marker and drop everything afterwards.
We pick this rule because it is the only one where a checking judgment calls an inference judgment.
That is the point where our algorithm recycles variables---right after a judgment chain is satisfied.
After applying this patch, to the best of our knowledge,
DK's algorithm behaves equivalently to our algorithm in terms of variable scoping.
However, this exploits markers in a way they were not intended to be used and seems ad-hoc.

% \subsection{Scoped type variables} Not discussed for this paper

%-------------------------------------------------------------------------------
\subsection{Elaboration}

Type-inference algorithms are often extended with an
associated elaboration. For example, for languages with implicit
polymorphism, it is common to have an elaboration to a variant of
System F~\citep{reynolds1983types}, which recovers type information and explicit type
applications. Therefore a natural question is whether our algorithm
can also accommodate such elaboration.
While our algorithmic reduction does not elaborate to System F,
we believe that it is not difficult to extend the algorithm with a (type-directed) elaboration.
We explain the rough idea as follows.

Since the judgment form of our algorithmic worklist contains a collection of judgments,
elaboration expressions are also generated as a list of equal length to
the number of judgments (\emph{not judgment chains}) in the worklist.
As usual, subtyping judgments translate to coercions (denoted by $f$ and
represented by System F functions),
all three other types of judgments translate to terms in System F (denoted by $t$).

Let $\Phi$ be the elaboration list, containing translated type coercions and terms:
$$\Phi ::= \nil \mid \Phi, f \mid \Phi, t$$

Then the form of our algorithmic judgment becomes:
$$\Gamma \hookrightarrow \Phi$$

We take Rule 18 as an example, rewriting small-step reduction in a relational style,
$$
\inferrule*[right=$\mathtt{Translation\_18}$]
{\Gamma \Vdash e \To_a a \le B \hookrightarrow \Phi, f, t}
{\Gamma \Vdash e \Lto B \hookrightarrow \Phi, f t}
$$
As is shown in the conclusion of the rule,
a checking judgment at the top of the worklist corresponds to a top element for elaboration.
The premise shows that one judgment chain may relate to more than one elaboration elements,
and that the outer judgment, being processed before inner ones,
elaborates to the top element in the elaboration list.

Moreover, existential variables need special treatment, since they may be solved at any point,
or be recycled if not solved within their scopes.
If an existential variable is solved, we not only propagate the solution to the other judgments,
but also replace occurrences in the elaboration list.
If an existential variable is recycled, meaning that it is not constrained,
we can pick any well-formed monotype as its solution.
The unit type $1$, as the simplest type in the system, is a good choice.

\subsection{Lexically-Scoped Type Variables}
We have further extended the type system with support for
lexically-scoped type variables~\citep{scoped-type-variables}.
Our Abella formalization for this extension proves
all the metatheory we discuss in Section~\ref{sec:metatheory}.

From a
practical point of view, this extension allows the implementation of a function
to refer to type variables from its type signature. For example,
$$(\lam{\lam[y] (x:a)}) : \all[a~b] a \to b \to a$$
has an annotation $(x:a)$ that refers to the type variable $a$ in the type signature.
This is not a surprising feature, since the declarative system already accepts similar programs
$$
\inferrule*[right=$\mathtt{DeclAnno}$]
{
    \Psi \vdash \all A \quad
    \inferrule*[right=$\mathtt{Decl\forall I}$]
    {\Psi,a \vdash e \Lto A}
    {\Psi\vdash e\Lto \all A}
}
    {\Psi\vdash (e: \all A)\To \all A}
$$

The main issue is the well-formedness condition.
Normally $\Psi \vdash (e : A)$ follows from $\Psi \vdash e$ and $\Psi \vdash A$.
However, when $A = \all A'$, the type variable $a$ is not in scope at $e$,
therefore $\Psi \vdash e$ is not derivable.
To address the problem, we add a new syntactic form that explicitly binds a
type variable simulatenously in a function and its annotation.
$$
\begin{array}{l@{\qquad}lcl}
\text{Expressions}\qquad&e &::=&\quad \ldots \mid \Lambda a.~e:A
\end{array}
$$

This new type-lambda syntax $\Lambda a.~e:A$ actually annotates its body $e$ with $\all A$,
while making $a$ visible inside the body of the function.
The well-formedness judgments are extended accordingly:
$$
\inferrule*[right=$\mathtt{wf_d}\Lambda$]
    {\Psi,a \vdash e \quad \Psi,a \vdash A}
    {\Psi \vdash \Lambda a.~e:A}
\qquad
\inferrule*[right=$\mathtt{wf\_}\Lambda$]
    {\Gm,a \vdash e \quad \Gm,a \vdash A}
    {\Gm \vdash \Lambda a.~e:A}
$$

Corresponding rules are introduced for both the declarative system and the algorithmic system:
$$
\inferrule*[right=$\mathtt{Decl}\Lambda$]
    {\Psi,a \vdash A \quad \Psi,a \vdash e\Lto A}
    {\Psi\vdash \Lambda a.~e:A \To \all A}
$$
$$\Gm \Vdash \Lambda a.~e:A \To_b \jg \algrule \Gm \Vdash [(\all A)/b]\jg, a \Vdash e \Lto A$$

In practice, programmers would not write the syntax $\Lambda a.~e : A$ directly.
The ScopedTypeVariables extension of Haskell is effective only
when the type signature is explicitly universally quantified
(which the compiler translates into an expression similar to $\Lambda a.~e : A$);
otherwise the program means the normal syntax $e : \all A$
and may not later refer to the type variable $a$.

We have proven all three desired properties for the extended system,
namely soundness, completeness and decidability.


%%%%%%%%%%%%%%%%%%%%%%%%%%%%%%%%%%%%%%%%%%%%%%%%%%%%%%%%%%%%%%%%%%%%%%%%
\chapter{Related Work}
\label{chap:related}
%%%%%%%%%%%%%%%%%%%%%%%%%%%%%%%%%%%%%%%%%%%%%%%%%%%%%%%%%%%%%%%%%%%%%%%%

Throughout the thesis, we have already discussed much of the closest related work.
In this section we summarize the key differences and novelties,
and discuss some other related work.

\section{Higher-Ranked Polymorphic Type Inference Algorithms}

\subsection{Predicative Algorithms}

Higher-ranked polymorphism is a convenient and practical feature of
programming languages.  Since full type-inference for System F is
undecidable~\citep{wells1999typability}, various decidable partial
type-inference algorithms were developed.
% No need to cite here: the following text is describing the point.
The declarative formulation of subtyping in Chapters~\ref{chap:ITP} and \ref{chap:ICFP}
(and later extended in Chapter~\ref{chap:Top}),
originally proposed by \citet{odersky1996putting}, is \emph{predicative}:
$\forall$'s only instantiate to monotypes.  The monotype restriction
on instantiation is considered reasonable and practical for most
programs, except for those that require sophisticated forms of
higher-order polymorphism.
In those cases, type annotations may guide the type system to
accept lambda functions of higher-ranked types.

In addition to OL's type system,
the bidirectional system proposed by DK~\citep{dunfield2013complete}
accepts even better type annotations, which also allow polymorphic types.
Such annotations also improve readability of the program,
and are not much of a burden in practice.
DK's algorithm is shown to be sound, complete and decidable in 70 pages of manual proofs.
Though carefully written, some of the proofs are incorrect
(see discussion in Section~\ref{ssec:DK_Algorithm} and
\ref{sec:metatheory:non-overlapping}),
which creates difficulties when formalizing them in a proof assistant.
In their follow-up work \citet{DunfieldIndexed} enrich the bidirectional higher-ranked system with
existentials and indexed types.
With a more complex declarative system, they developed a proof of over 150 pages.
It is even more difficult to argue its correctness for every single detail
within such a big development.
Unfortunately, we find that their Lemma 26 (Parallel Admissibility) appears to have the same issue 
as lemma 29 in \citep{dunfield2013complete}: the conclusion is false. We also discuss
the issue in more detail in Section~\ref{ssec:DK_Algorithm}.

\citet{jones2007practical} developed another higher-ranked predicative bidirectional type system.
Their subtyping relation is enriched with \emph{deep skolemisation},
which is more general than ours and allows more valid relations.
For example,
$$\all{\all[b] a \to b \to b} \le \all a \to (\all[b] b \to b)$$
this subtyping relation does not hold in OL's subtyping relation,
because the instantiation of $b$ in the left-hand-side type
happens strictly before the introduction of
the variable $b$ in the right-hand-side type.
Deep skolemisation can be achieved through
a pre-processing that extracts all the $\forall$'s
in the return position of a function type to the top level.
After such pre-processing, the right-hand-side type becomes
$\all {\all[b] a \to b \to b}$ and the subtyping relation holds.

In terms of handling applications where type parameters are implicit,
they do not use the application inference judgment as DK's type system.
Instead, they employ a complicated mechanism for implicit instantiation
taken care by the unification process for the algorithm.
A manual proof is given, showing that the algorithm is sound and
complete with respect to their declarative specification.

In a more recent work, \citet{xie2018letarguments} proposed a variant of a
bidirectional type inference system for a predicative system with higher-ranked types.
Type information flows from arguments to
functions with an additional \emph{application} mode. This variant 
allows more higher-order typed programs to be inferred without additional annotations.
Following the new mode, the let-generalization of the Hindley-Milner system
is well supported as syntactic sugar. The formalization includes some
mechanized proofs for the declarative type system, but all proofs regarding
the algorithmic type system are manual.

\subsection{Impredicative Algorithms}

Impredicative System F allows instantiation with polymorphic types,
but unfortunately its subtyping system is already undecidable~\citep{tiuryn1996subtyping}.
Works on partial impredicative type-inference algorithms
navigate a variety of design tradeoffs for a decidable algorithm.
Generally speaking, such algorithms tend to be more complicated,
and thus less adopted in real-world programming languages.

$\text{ML}^\text{F}$~\citep{le2003ml,remy2008from,Botlan2009recasting}
extends types of System F with a form of bounded quantification
and proposes an impredicative system.
The type inference algorithm always infers principle types given proper type annotations.
Through the technique called ``monomorphic abstraction of polymorphic types'',
polymorphic instantiations are expressed by constraints
on type variables of type schemes.
An annotation is only needed when an argument of a lambda function
is used polymorphically.
Moreover, their type system is robust against a wide range of program transformations,
including let-expansion, let-reduction and $\eta$-expansion.
However, the extended type structure of $\text{ML}^\text{F}$
introduces non-compatible types with System F,
and it complicates the metatheory and implementation of the type system.

HMF~\citep{leijen2008hmf} takes a slightly different approach
by not extending types with bounds,
therefore programmers still work with plain System F types.
The algorithm works similarly compared to $\text{ML}^\text{F}$,
except that it does not output richer types.
Instead, when there are ambiguity and no principal type can be inferred,
being a conservative algorithm,
HMF prefers predicative instantiations
to maintain backward the compatibility of HM.
Soon after HMF, HML~\citep{leijen2009flexible} is proposed as
an extension of HMF with flexible types.
The system eliminates rigid quantifications ($\alpha = \sigma$)
of $\text{ML}^\text{F}$ and only keeps the flexible quantifications ($\alpha \ge \sigma$).
Furthermore, a variant ``Rigid HML'' is presented
by restricting let-bindings to System F types only,
so that type annotations can still stay inside System F.
Similar to HMF, they both require type annotations at higher-ranked
types or ambiguous implicit instantiations.

FPH~\citep{vytiniotis2008fph} employ a box notion \framebox{$\sigma$} to
encapsulate polymorphic instantiations internally in the algorithm.
A type inside a box indicates that the instantiation is impredicative.
Therefore an inferred type with a box type in it means that
incomparable System F types may be produced,
and that is rejected before entering the environment.
Although more annotations might be required compared with other approaches,
the algorithm of FPH is much simpler,
thanks to the simple syntax and subtyping relation of box types.


Guarded Impredicative Polymorphism~\citep{Serrano2018} was recently proposed
as an improvement on GHC's type inference algorithm with impredicative instantiation.
They make use of local information in $n$-ary applications to
infer polymorphic instantiations with a relatively simple specification and unification algorithm.
Although not all impredicative instantiations can be handled well,
their algorithm is already quite useful in practice.

A recent follow-up work, the Quick Look~\citep{quicklook2020},
takes a more conservative approach.
The algorithm will try its best to infer impredicative instantiations,
and only use the instantiation when it is the best one.
In order to do so,
the algorithm also needs to analyse all the argument types during a function call,
and make use of the \emph{guardedness} property to decide the principality of
the inferred instantiation.
When Quick Look cannot give the best instantiation,
it will instead try predicative inference as HM.
This conservative approach makes sure that the algorithm does not infer bad types
and leads the subsequent type checking in wrong directions.
In the meantime, they treat the function arrow ($\to$)
as invariant like normal type constructors.
This change significantly restricts the subtyping relation and thus
simplifies guesses for implicit parameters.
Manual $\eta$-expansions are required to regain
the co- and contravariance of function types.

FreezeML~\citep{FreezeML} is another recent work that extends the ML type system
with impredicative instantiations.
A special syntax of expression, the explicit freezing $\lceil x \rceil$,
is added to guide the type system.
Freezed variables are prevented to be instantiated by the type system,
therefore variables of polymorphic types may force the type inference algorithm
to instantiate impredicatively.
In combination with the let-generalization rule, programmers may also encode
explicit generalization and explicit instantiation.
This work provides a nice means for programmers to control how type inference algorithm
deals with impredicative instantiations.

\section{Type Inference Algorithms with Subtyping}
Type systems in presence of subtyping usually encounter constraints that
are not simply equalities as in HM.
Therefore constraint solvers used in HM, where unifications are based on equality,
cannot be easily extended to support subtyping.
Instead, constraints are usually collected as subtyping relations
and may delay resolving as the constraints accumulate.
\citet{EIFRIG1995,Eifrig1995sound} proposed systems that are based on
\emph{constraint types}, i.e. types expressed together with a set of constraints
$\tau \mid \{\overline{\tau_1 \le \tau_2}\}$.
Their type checking algorithm checks at each step whether
each constraint in the closure of constraint set is \emph{consistent},
which is a set of rules that prevent obvious contradictions appear,
such as $\text{Int} \le \text{Bool}$.
Our attempt in Section~\ref{subsec:lazy_subst} is similar to this idea,
where we exhaustively check if every subtyping relation derived from the bounds is valid.
However, our algorithm tries to solve the bounds into concrete types,
and it turns out that it never terminates in some complex cases.

Constraint types improve the expressiveness of the type system,
yet the type inference algorithm can be quite slow.
The size of the constraint is linear in the program size,
and the closure can grow to cubic size.
\citet{pottier1998phd} proposed three methods to simplify constraints in his Ph.D. thesis,
aiming at improving the efficiency of type inference algorithms and
improving the readability of the resulting types.
By \emph{canonization}, constraints are converted to canonical forms
with the introduction of new meta-variables;
\emph{garbage collection} eliminates constraints that do not affect the type;
finally, \emph{minimization} shares nodes within the constraint graph.

Inspired by the simplification strategies of Pottier's,
MLsub~\citep{mlsub} suggest that the data flow on the constraint graph
can reflect directly on types extended by a richer type system,
where lattices operations are used to represent constraints imposed on type variables.
Polar types distinguish between input types and output types,
and pose different restrictions on them.
As a result, constraints can easily be transformed into canonical forms,
and the bi-unification algorithm can solve them by simple substitutions.
One can also view the MLsub system as a different way to encode the constraint types:
instead of a set of constraints that is stated along with the type,
types themselves now contain subtyping constraints with the help of lattice operations.
Similarly, the size of constraints may grow with the size of program
and affect the readability.
Therefore, various simplification algorithms should be used in real applications.
A more recent work, the Simple-sub~\citep{Parreaux2020simple}, further simplifies
the algorithm of MLsub and is implemented in 500 lines of code.
While being equivalent to MLsub, it is a more efficient variant.

\section{Techniques Used in Type Inference Algorithms}

\subsection{Ordered Contexts in Type Inference}
\citet{gundry2010type} revisit algorithm $\mathcal{W}$ and
propose a new unification algorithm with the help of ordered contexts.
Similar to DK's algorithm, information of meta-variables flows from input contexts to output contexts.
Not surprisingly, its information increase relation has a similar role to DK's context extension.
Our algorithm, in contrast,
eliminates output contexts and solution records ($\al = \tau$),
simplifying the information propagation process through immediate substitution
by collecting all the judgments in a single worklist.

\subsection{The Essence of ML Type Inference}
Constraint-based type inference is adopted by \citet{remy-attapl} for
ML type systems, which do not employ higher-ranked polymorphism. An
interesting feature of their algorithm is that it keeps precise
scoping of variables, similarly to our approach.  Their algorithm is
divided into constraint generation and solving phases (which are
typical of constraint-based algorithms). Furthermore an intermediate
language is used to describe constraints and their constraint solver
utilizes a stack to track the state of the solving process.  In
contrast, our algorithm has a single phase, where the judgment chains
themselves act as constraints, thus no separate constraint language is
needed.

%A possible advantage of a separate constraint language is to provide
%a general and reuseable interface
%that interprets the desired properties for type inference.

\subsection{Lists of Judgments in Unification}
Some work~\citep{Reed2009,Abel2011higher} adopts a similar idea to this paper
in work on unification for dependently typed languages. Similarly to our work
the algorithms need to be very careful about scoping, since the order of variable
declarations is fundamental in a dependently typed setting. 
Their algorithms simplify a collection of unification constraints progressively in a single-step style.
In comparison, our algorithm mixes variable declarations with judgments,
resulting in a simpler judgment form,
while processing them in a similar way.
One important difference is that contexts are
duplicated in their unification judgments, which complicates the unification process,
since the information of each local context needs to be synchronized.
Instead we make use of the nature of ordered context to control the scope of unification variables.
While their algorithms focus only on unification,
our algorithm also deals with other types of judgments like synthesis.
A detailed discussion is given in Section~\ref{sec:overview:list}.





\section{Mechanical Formalization of Polymorphic Type Systems}

% \paragraph{Mechanical Formalization of Polymorphic Subtyping}
% In all previous work on type inference for higher-ranked polymorphism
% (predicative and impredicative) discussed above, proofs and
% metatheory for the algorithmic aspects are manual. The only partial effort on mechanizing algorithmic
% aspects of type inference
% for higher-ranked types is
% the Abella formalization of \emph{polymorphic subtyping} by \citet{itp2018}.
% The judgment form of worklist $\Gm \vdash \Omega$ used in the formalization simplifies
% the propagation of existential variable instantiations.
% However, the approach has two main drawbacks:
% it does not collect unused variable declarations effectively;
% and the simple form of judgment cannot handle inference modes, which output types.
% The new worklist introduced in this paper inherits the simplicity of propagating instantiations,
% but overcomes both of the issues by mixing judgments with declarations
% and using the continuation-passing-style judgment chains. Furthermore,
% we formalize the complete bidirectional type system by
% \citet{dunfield2013complete}, whereas Zhao et al. only formalize
% the subtyping relation. 

% \paragraph{Mechanical Formalizations of Other Type-Inference Algorithms}
Since the publication of the \textsc{POPLMark} challenge~\citep{aydemir2005mechanized},
many theorem provers and packages provide new methods for dealing
with variable binding~\citep{aydemir2008engineering,urban2008nominalTech,chlipala2008parametric}.
More and more type systems are formalized with these tools.
However, mechanizing certain algorithmic aspects, like unification and
constraint solving, has received very little attention and is still challenging.
Moreover, while most tools support local (input) contexts in a neat way,
many practical type-inference algorithms require
more complex binding structures with output contexts or various forms of constraint solving procedures.

Algorithm $\mathcal{W}$,
as one of the classic type inference algorithms for polymorphic type systems,
has been manually proven to be sound and complete
with respect to the Hindley-Milner type system~\citep{hindley1969principal,milner1978theory,damas1982principal}.
After around 15 years, the algorithm was formally verified by
\citet{naraschewski1999type} in Isabelle/HOL~\citep{nipkow2002isabelle}.
The treatment of new variables was tricky at that time, while the overall structure follows the
structure of Damas's manual proof closely.
%Another complication is the encoding of substitutions,
%which they chose to formalize as a function instead of an association list.
Later on, other researchers~\citep{dubois2000proving,dubois1999certification}
formalized algorithm $\mathcal{W}$ in Coq~\citep{Coq}.
Nominal techniques~\citep{urban2008nominalTech} in Isabelle/HOL have been
developed to help programming language formalizations, and are used for a similar
verification~\citep{urban2008nominal}. Moreover, Garrigue~\citep{garrigue2015certified}
mechanized a type inference algorithm,
with the help of locally nameless~\citep{LocallyNameless},
for Core ML extended with structural polymorphism and recursion.



\section{Conclusion and Future Work}

In this paper we have shown how to mechanise an algorithmic subtyping relation
for higher-order polymorphism, together with its proofs of soundness,
completeness and decidability, in the Abella proof assistant. 
In ongoing work we are extending our mechanisation with a bidirectional type
inference algorithm. The main difficulty there is communicating the
instantiations of existential variables from the subtyping algorithm to the
type inference. To make this possible we are exploring a continuation passing
style formulation, which generalises the worklist approach.
Another possible extension is to have the algorithm return an explicit witness
for the subtyping as part of type-directed elaboration into System F.


%% Acknowledgments
\begin{acks}                            %% acks environment is optional
                                        %% contents suppressed with 'anonymous'
  %% Commands \grantsponsor{<sponsorID>}{<name>}{<url>} and
  %% \grantnum[<url>]{<sponsorID>}{<number>} should be used to
  %% acknowledge financial support and will be used by metadata
  %% extraction tools.

We sincerely thank the anonymous reviewers for their insightful
comments. Ningning Xie found the issue with Lemma 29 in DK's formalization that we reported 
on this article. This work has been sponsored by the Hong Kong Research
Grant Council projects number 17210617 and 17258816, and by the
Research Foundation - Flanders.
\end{acks}


%% Bibliography
\bibliography{main}

%% Appendix
% \newpage
% \appendix
% 
\section{Technical Discussions on DK's Manual Proofs}

\subsection{Broken Lemmas in DK's Papers}
\label{appendix:false_lemmas}

We found false lemmas in the manual proofs of DK' two papers
\cite{dunfield2013complete} and \cite{DunfieldIndexed}.
\begin{itemize}
    \item
        In the first paper, Lemma 29 on page 9 of its appendix says:
        \begin{lemma}[Parallel Admissibility of~\cite{dunfield2013complete}]~\\
        If $\Gamma_L \longrightarrow \Delta_L$ and
        $\Gamma_L, \Gamma_R \longrightarrow \Delta_L, \Delta_R$ then:
        \begin{enumerate}
            \item $\Gamma_L,\al,\Gamma_R \longrightarrow \Delta_L, \al, \Delta_R$
            \item If $\Delta_L \vdash \tau'$ then
                $\Gamma_L,\al,\Gamma_R \longrightarrow \Delta_L, \al = \tau', \Delta_R$.
            \item If $\Gamma_L \vdash \tau$ and $\Delta_L \vdash \tau'$ and
                $[\Delta_L]\tau = [\Delta_L]\tau'$, then
                $\Gamma_L, \al = \tau, \Gamma_R \longrightarrow \Delta_L, \al = \tau', \Delta_R$.
        \end{enumerate}
        \end{lemma}

        We give a counter-example to this lemma:\\
        Pick $\Gm_L = \nil, \Gm_R = \bt, \Delta_L = \bt, \Delta_R = \nil$, then both conditions
        $\nil\longrightarrow \bt$ and $\bt\longrightarrow \bt$ hold, but the first conclusion
        $\al,\bt \longrightarrow \bt,\al$ does not hold.
    \item
        In the second paper, as an extended work to the first paper, Lemma 26 on page 22 of its appendix says:
        \begin{lemma}[Parallel Admissibility of~\cite{DunfieldIndexed}]~\\
        If $\Gamma_L \longrightarrow \Delta_L$ and
        $\Gamma_L, \Gamma_R \longrightarrow \Delta_L, \Delta_R$ then:
        \begin{enumerate}
            \item $\Gamma_L,\al:\kappa,\Gamma_R \longrightarrow \Delta_L, \al:\kappa, \Delta_R$
            \item If $\Delta_L \vdash \tau' : \kappa$ then
                $\Gamma_L,\al:\kappa,\Gamma_R \longrightarrow \Delta_L, \al:\kappa = \tau', \Delta_R$.
            \item If $\Gamma_L \vdash \tau : \kappa$ and $\Delta_L \vdash \tau' types$ and
                $[\Delta_L]\tau = [\Delta_L]\tau'$, then
                $\Gamma_L, \al:\kappa = \tau, \Gamma_R \longrightarrow \Delta_L, \al:\kappa = \tau', \Delta_R$.
        \end{enumerate}
        \end{lemma}
        
        A similar counter-example is given:\\
        Pick $\Gm_L = \nil, \Gm_R = \bt:\star, \Delta_L = \bt:\star, \Delta_R = \nil$, then both conditions
        $\nil\longrightarrow \bt:\star$ and $\bt:\star\longrightarrow \bt:\star$ hold, but the first conclusion
        $\al:\kappa,\bt:\star \longrightarrow \bt:\star,\al:\kappa$ does not hold.
\end{itemize}

\subsection{A Fix to DK's Subsumption Lemma}
\label{appendix:subsumption}

Let's first recall the lemma:
\begin{lemma}[Subsumption]
Given $\Psi' \le \Psi$:
\begin{enumerate}
    \item If $\Psi \vdash e \Lto A$ and $\Psi \vdash A \le A'$ then $\Psi' \vdash e \Lto A'$;
    \item If $\Psi \vdash e \To B$ then there exists
        $B'$ s.t. $\Psi \vdash B' \le B$ and $\Psi' \vdash e \To B'$;
    \item If $\Psi \vdash \appInf{A}{e}{C}$ and $\Psi \vdash A' \le A$,
        then there exists $C'$ s.t. $\Psi \vdash C' \le C$ and $\Psi' \vdash \appInf{A'}{e}{C'}$.
\end{enumerate}
\end{lemma}

\paragraph{Problems in DK's Manual Proof}
We tried to follow DK's manual proof for this lemma.
While being written in a clear format and providing enough details,
the proof is not fully accepted by the proof assistant.
Specifically, the first two applications of induction hypotheses on page 22 are not accepted.
Either of them seems to use a slightly different ``induction hypothesis'' than the true one.

\paragraph{A Fix to the Proof}
We make use of our worklist measures for the induction.
Recall that $|e|_e$ measures term size;
and the judgment measure counts checking as 2, inference as 1 and application inference as 3;
and $|A|_\forall$ counts the number of $\forall$'s in a type.

The proof is by a nested mutual induction on the lexicographical order of the measures
$$\langle |e|_e, |\cdot|_\Leftrightarrow, |A|_\forall + |A'|_\forall \rangle,$$
where the second measure is 2 for checking (1), 1 for inference (2) and 3 for application inference (3);
and the third measure does not apply to case (2) since no $A$ is used.

All but two cases can be finished easily following the declarative typing derivation,
and the proof shares a similar structure to DK's.
One tricky case related to Rule $\mathtt{Decl\forall I}$ indicates that $A$ has the shape $\all A_0$,
thus the subtyping relation derives from either $\mathtt{{\le}\forall L}$ or $\mathtt{{\le}\forall R}$.
For each of the case, the third measure $|A|_\forall + |A'|_\forall$ decreases
(the $\mathtt{{\le}\forall L}$ case requires a type substitution lemma obtaining
$\Psi \vdash e \Lto [\tau/a]A_0$ from the typing derivation).

Another tricky case is $\mathtt{Decl{\to}I}$.
When the subtyping relation is derived from $\mathtt{{\le}{\to}}$,
a simple application of induction hypothesis finishes the proof.
When the subtyping relation is derived from $\mathtt{{\le}\forall R}$,
$|A'|_\forall$ decreases, and thus the induction hypothesis finishes this case.


\end{document}
