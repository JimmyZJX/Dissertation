
\section{Backtracking Algorithm}

\subsection{Syntax}

\begin{figure}[ht]
    \begin{gather*}
    \begin{aligned}
        \text{Existential variables}\qquad&\al, \bt\\
        \text{Types}\qquad&A, B, C &::=&\quad 1 \mid \top \mid \bot \mid a \mid \all A \mid A\to B \mid \al\\
        \text{Algorithmic judgment chain}\qquad&\jg &::=&\quad A \le B \mid e\Lto A \mid e\To_{a} \jg \mid \appInfAlg{A}{e}\\
        \text{Algorithmic worklist}\qquad&\Gm &::=&\quad \cdot \mid \Gm, a \mid \Gm, \al \mid \Gm \Vdash \jg
    \end{aligned}
    \end{gather*}
\Description{Algorithmic Syntax}
\caption{Algorithmic Syntax}\label{fig:top_alg_syntax}
\end{figure}

The algorithmic syntax is shown in Figure~\ref{fig:top_alg_syntax}.
Compared with the declarative system, existential variables $\al, \bt$ are used
as placeholders for unsolved mono-types.
The judgment chain $\jg$ and worklist context $\Gm$ are defined in the same way as the ICFP work.

The well-formedness relation is almost the same as that of the ICFP work.
The hole notation is also inherited.

\paragraph{The Solve Notation}
We define a set of auxiliary substitution functions in the form $\{\al := \tau_A\}$
to improve readability of our algorithmic definitions.
Basically it refers to a global substitution when solving an existential variable.

A detailed definition of the notation is shown as follows:
\begin{gather*}
    \begin{aligned}
        \{\al := \tau\}~(\Gm_L, \al, \Gm_R) &= \Gm_L, [\tau/\al]\Gm_R & \tau = 1, \top, \bot \text{ or }a\\
        \{\al := \bt\}~(\Gm_L, \al, \Gm_R) &= \Gm_L, [\bt/\al]\Gm_R & \bt \in \Gm_L\\
        \{\al := \al[1] \to \al[2]\}~(\Gm_L, \al, \Gm_R) &= \Gm_L, \al[1], \al[2], [\al[1] \to \al[2]/\al]\Gm_R\\
    \end{aligned}
\end{gather*}

\subsection{Algorithmic Subtyping}

\begin{figure}[!ht]
    \noindent\framebox{$\Gm\rto \Gm'$} $\Gm$ reduces to $\Gm'$.
    \begin{gather*}
    \begin{aligned}
    \Gm, a &\algrule \Gm\\
    \Gm, \al &\algrule \Gm\\
    \Gm, x:A &\algrule \Gm\\
    % \\
    % \pmb{
    % (\Gm, L \le \al \le U)} &\algrule \Gm \Vdash l_1 \le u_1 \Vdash \ldots \Vdash l_n \le u_m
    % 	\quad \color{blue} \text{Add $nm$ judgments}\\
    \\
    \Gm \Vdash 1\le 1 &\algrule \Gm\\
    \Gm \Vdash a\le a &\algrule \Gm\\
    \Gm \Vdash A_1\to A_2 \le B_1\to B_2 &\algrule \Gm \Vdash A_2 \le B_2 \Vdash B_1\le A_1\\
    \Gm \Vdash \forall a. A\le B &\algrule \Gm, \al \Vdash [\al/a]A\le B\\
    \Gm \Vdash A\le \forall b. B &\algrule \Gm,b \Vdash A\le B\\
    \Gm \Vdash A\le \top &\algrule \Gm\\
    \Gm \Vdash \bot \le B &\algrule \Gm\\
    \\
    \Gm[\al] \Vdash \al\le \al &\algrule \Gm\\
    \\
    % \multispan2{\text{\jimmy{These 2 rules introduce non-determinicity}}}\\
    \Gm[\al] \Vdash A \le \al &\algrule \{\al := \top\}~\Gm[\al]\\
    \Gm[\al] \Vdash \al \le A &\algrule \{\al := \bot\}~\Gm[\al]\\
    \\
    \Gm[\al] \Vdash \al \le A\to B &\algrule
    \{\al := \al[1] \to \al[2]\}~(\Gm[\al] \Vdash \al \le A\to B)\\
    &\qquad\qquad \text{when } \al\notin FV(A \to B)\\
    \Gm[\al] \Vdash A\to B \le \al &\algrule
    \{\al := \al[1] \to \al[2]\}~(\Gm[\al] \Vdash A\to B \le \al)\\
    %\Gm[\al[1], \al[2], L \cup \{\al[1] \to \al[2]\} \le \al \le U] \Vdash A \to B \le \al[1] \to \al[2]\\
    &\qquad\qquad \text{when } \al\notin  FV(A \to B)\\
    \\
    \Gm[a][\bt] \Vdash a \le \bt &\algrule \{\bt := a\}~\Gm[a][\bt]\\
    \Gm[a][\bt] \Vdash \bt \le a &\algrule \{\bt := a\}~\Gm[a][\bt]\\
    \Gm[\bt] \Vdash 1 \le \bt &\algrule \{\bt := 1\}~\Gm[\bt]\\
    \Gm[\bt] \Vdash \bt \le 1 &\algrule \{\bt := 1\}~\Gm[\bt]\\
    \\
    \Gm[\al][\bt] \Vdash \al \le \bt &\algrule \{\bt := \al\}~\Gm[\al][\bt]\\
    \Gm[\al][\bt] \Vdash \bt \le \al &\algrule \{\bt := \al\}~\Gm[\al][\bt]\\
    \end{aligned}
    \end{gather*}
\Description{Algorithmic Garbage Collection and Subtyping}
\caption{Algorithmic Garbage Collection and Subtyping}\label{fig:top_alg_subtyping}
\end{figure}

Figure~\ref{fig:top_alg_subtyping} describes the algorithmic rules for subtyping.
The relation is stated in a small-step ``reduction'' form, i.e. in each step,
the worklist is analysed from the right-hand-side and reduced according to the top judgment.
The overall procedure succeeds iff the worklist eventually reduces to $\cdot$ (the empty worklist).


We categorize them into 7 groups according to their behavior:
\begin{enumerate}
    \item Rules 1-3 are basic garbage collection rules.
        Given that the worklist $\Gm$ is well-formed,
        no reference of a variable should occur before its declaration.
        Therefore removing the declaration in the top position does not break well-formedness.

        An existential variable that is unsolved in the top position indicates
        that it is not constrained,
        thus picking any well-formed mono-type as its solution is acceptable.
        In our algorithmic formalization, we simply drop the existential variable.

    \item Rules 4-10 directly correspond to the declarative subtyping rules.
        With no top-level existential variables,
        there are nothing to guess immediately,
        and thus the algorithm behaves just like the declarative system.

    \item Rule 11 is a base case in the algorithmic system.
        The declarative reflexivity property suggests that any solution is acceptable,
        thus the judgment holds without any constraint.
    
    \item Rules 12-13 are important rules that requires backtracking techniques
        for implementation.
        These rules \emph{overlaps} with all the remaining rules when solving
        an existential variable.
        In other words, they simply try if $\top$ or $\bot$ satisfies the constraints
        in parallel with other possibilities.
    
    \item Rules 14-15 compares an existential variable $\al$ with a function type,
        resulting in solving the $\al$ by $\al[1] \to \al[2]$.
        The freshness condition rules out the possibility
        when there is a cyclic dependency.
        For example, the judgment
        \[\al \le 1 \to \al\]
        is satisfied with either of these solutions to $\al$:
        \[ \bot, 1 \to \bot, \top \to \bot, 1 \to 1 \to \bot, \ldots \]
        However, we argue that comparing $\al$ with a function type
        that contains $\al$ itself is hardly useful in practice,
        and most of the solutions are meaningless.
        Therefore, in our algorithm, only the $\bot$ solution is considered
        with rule 13.
        The condition of rule 14 rejects the judgement for further analysis
        and thus does not produce more solutions.
        This is one source of incompleteness of our algorithm with respect to the declarative specification.
    
    \item Rules 16-19 solve existential variables against a type variable or the unit type.
        For instance, the judgment
        \[\al \le 1\]
        only have two solutions: $\al = 1$ or $\al = \bot$.
        In similar cases, one of the solutions is produced by rule 12 or 13,
        and the other one is given by one of the rules in this group.
        Additional well-formedness check is performed when type variables are encountered;
        a solution to an existential variable must be well-formed in the context
        before the existential variable is defined.
    
    \item Rules 20-21 deals with subtyping judgments
        that compares two different existential variables.
        The only difference between them is the variable order.
        Similar to the type variable case in the previous group,
        existential variables must solve to another one defined earlier.
        With rules 12, 13, 20 and 21, a judgment like
        \[\al \le \bt\]
        could possibly give any of the following solutions:
        \[ \al = \bt ~(\text{or } \bt = \al)
            \text{ or } \al = \bot \text{ or } \bt = \top \]
        Those are good attempts, but unfortunately,
        they does not cover the complete set of possibilities.
        The following example worklist
        \[ \al, \bt \Vdash \bt \le 1 \to 1 \Vdash \al \le \bt \]
        has a solution $\al = 1 \to \top, \bt = 1 \to 1$ missed by our algorithm.
        Similar situations happen when the judgments are specifically ordered;
        if $\bt \le 1 \to 1$ is the top-most judgment,
        the algorithm will not miss this solution.

        Although such treatment for existential variable solving is incomplete in theory,
        a large number of practical unification is simply equality,
        and the algorithm completely handles these programs.
        Other programs that exploit complex guessing involving subtyping,
        the programmer may put annotation when the type inference algorithm
        does not find the optimal solution.
\end{enumerate}


\subsection{Algorithmic Typing}

\begin{figure}[!ht]
\noindent\framebox{$\Gm\rto \Gm'$} $\Gm$ reduces to $\Gm'$ (continued).
\begin{gather*}
\begin{aligned}
    \Gm \Vdash e \Lto B &\algrule \Gm \Vdash e\To_a a\le B \quad
    \text{when } e \neq \lam e' \text{ and } B \neq \all B'\\
    \Gm \Vdash e\Lto \all A &\algrule \Gm,a \Vdash e\Lto A\\
    \Gm \Vdash \lam e \Lto A\to B &\algrule \Gm, x:A  \Vdash e \Lto B\\
    \Gm[\al] \Vdash \lam e \Lto \al &\algrule \{\al := \al[1]\to \al[2]\}~\Gm, x:\al[1] \Vdash e \Lto \al[2]\\
    \Gm \Vdash e \Lto \top &\algrule \Gm\\
    \Gm[\al] \Vdash e \Lto \al &\algrule \{\al := \top\}~\Gm[\al]
    % \quad\text{\jimmy{Additional}}
    \\[3mm]
    \Gm \Vdash x\To_a \jg &\algrule \Gm \Vdash [A/a] \jg \quad \text{when } (x:A)\in \Gm\\
    \Gm \Vdash (e:A)\To_a \jg &\algrule \Gm \Vdash [A/a]\jg \Vdash e \Lto A\\
    \Gm \Vdash ()\To_a \jg &\algrule \Gm \Vdash [1/a]\jg\\
    \Gm \Vdash \lam e \To_a \jg &\algrule
    \Gm,\al,\bt \Vdash [\al\to\bt/a]\jg, x:\al \Vdash e\Lto \bt\\
    \Gm \Vdash e_1\ e_2 \To_a \jg &\algrule \Gm \Vdash e_1\To_b (\appInfAlg{b}{e_2})
    \\[3mm]
    \Gm \Vdash \appInfAlg{\all A}{e} &\algrule \Gm,\al \Vdash \appInfAlg{[\al/a]A}{e}\\
    \Gm \Vdash \appInfAlg{A\to C}{e} &\algrule \Gm \Vdash [C/a]\jg \Vdash e \Lto A\\
    \Gm \Vdash \appInfAlg{\bot}{e} &\algrule \Gm \Vdash [\bot/a]\jg\\
    \Gm[\al] \Vdash \appInfAlg{\al}{e} &\algrule
        \{\al := \al[1]\to \al[2]\}~\Gm \Vdash \appInfAlg{\al[1]\to\al[2]}{e}\\
    \Gm[\al] \Vdash \appInfAlg{\al}{e} &\algrule
        \{\al := \bot\}~\Gm[\al] \Vdash [\bot/a]\jg
\end{aligned}
\end{gather*}
\Description{Algorithmic Typing}
\caption{Algorithmic Typing}\label{fig:top_alg_typing}
\end{figure}

The algorithmic typing rules split into three groups,
according to the category of the top-most judgment.

\begin{enumerate}
    \item Checking mode. Rules 22-27 reduce top-level checking judgments.
            Rules 22, 23, 24 and 26 directly reflects how the declarative system behaves.
            Rule 25 splits $\al$ into $\al[1] \to \al[2]$ to mimic the same logic as rule 24,
            since a lambda expression must be of an function type.
            Rule 27 is another backtracking rule, which overlaps with all other checking rules.
            Solving $\al$ to $\top$ in such case reflects
            the declarative rule $\mathtt{Decl{\top}}$ when the unknown type is $\top$.
            Furthermore, we eliminate the algorithmic counterpart of declarative rule
            $\mathtt{Decl1I}$, because a combination of Rules 22 and 30 already
            accepts the judgment $() \Lto 1$.
    
    \item Inference mode. Rules 28-32 reduce inference judgments.
            Like our previous work, encoding of the return type is by
            an explicit substitution on the binder of a judgment chain.
            Rules 28, 29, 30 and 32 corresponds to the declarative system straightforwardly.
            Rule 31 guesses an unannotated lambda with a mono function type $\al \to \bt$.
            Writing in a slightly different way may help improve readability:
            \[ \Gm \Vdash \lam e \To_a \jg \rrule{31'}
                \Gm, \al, \bt \Vdash [\al \to \bt / a]\jg \Vdash \lam e \Lto \al \to \bt \]
            Rule 32 illustrates how the judgment chain works, with a continuation-passing-style
            encoding of the type inference task.
    
    \item Application inference mode. Rules 33-37 reduce application inference judgments.
            Each of these judgments accepts an input function type and an argument expression
            and produces the expected return type.
            Rules 33, 34 and 35 are direct translation from the declarative rules.
            Rule 33, specifically, enables implicit parametric polymorphism via
            existential variable solving.
            Rules 36 and 37 deal with cases where a single existential variable
            $\al$ as a function type.
            Rule 36 splits $\al$ into a function unknown type $\al[1] \to \al[2]$.
            Rule 37 tries the solution $\al = \bot$ and returns the $\bot$ type.
            Both of the rules are not so useful in reality,
            since it is derived from a parametric function type like $\all a$,
            which behaves like the the $\bot$ type.


\end{enumerate}

