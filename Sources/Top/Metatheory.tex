
\section{Metatheory}

\subsection{Declarative Properties}\label{sec:meta:decl}

An important desired property for a type system is \emph{checking subsumption},
which basically says that any expression can
check against any super type of its actual type.
Since our bidirectional type system defines the checking mode, inference mode and
application inference mode mutually,
we formalize the generalized \emph{typing subsumption}.

\begin{definition}[Worklist Subtyping]
    Worklist subtyping compares the type of variables bound in the worklist.
    $\Gm <: \Gm'$ iff each binding in $\Gm$ is converted to one with a super type.
    \begin{gather*}
        \inferrule*[right=$\mathtt{<:nil}$]
            {~}{\cdot <: \cdot}
        \qquad
        \inferrule*[right=$\mathtt{<:ty}$]
            {\Gm <: \Gm'}
            {\Gm, a <: \Gm', a}
        \\
        \inferrule*[right=$\mathtt{<:of}$]
            {\Gm' \vdash A \le B \\ \Gm <: \Gm'}
            {\Gm, x:A <: \Gm', x:B}
        \qquad
        \inferrule*[right=$\mathtt{<:\omega}$]
            {\Gm <: \Gm'}
            {\Gm \Vdash \jg <: \Gm' \Vdash \jg}
    \end{gather*}
\end{definition}

A basic property of worklist subtyping is that they acts similarly when viewed as contexts.
\begin{lemma}[Worklist Subtyping Equivalence]
    Given $\Gm <: \Gm'$, $\Gm \vdash A \le B \Longleftrightarrow \Gm' \vdash A \le B$.
\end{lemma}

\begin{lemma}[Typing Subsumption]
    Given $\Gm <: \Gm'$,
    \begin{enumerate}[1)]
        \item If $\Gm' \vdash e \Lto A$ and $\Gm' \vdash A \le B$, then $\Gm \vdash e \Lto B$;
        \item If $\Gm' \vdash e \To A$, then $\exists B$ s.t. $\Gm' \vdash B \le A$ and $\Gm \vdash e \To B$.
        \item If $\Gm' \vdash \appInf{C}{e}{A}$ and $\Gm' \vdash D \le C$, then
            $\exists B$ s.t. $\Gm' \vdash B \le A$ and $\Gm \vdash \appInf{D}{e}{B}$.
    \end{enumerate}
\end{lemma}

\begin{proof}
    By induction on the following size measure (lexicographical order on a 3-tuple):
    \begin{itemize}
        \item Checking ($e \Lto A$): $\langle |e|, 1, |A|_\forall + |B|_\forall \rangle$
        \item Inference ($e \To A$): $\langle |e|, 0, 0 \rangle$
        \item Application inference ($\appInf{A}{e}{C}$): $\langle |e|, 2, |C|_\forall + |D|_\forall \rangle$
    \end{itemize}
    Most of the cases are straightforward.
    When rule $\mathtt{{\le}\forall L}$ is applied for the subtyping predicate
    like $\Gm' \vdash A \le B$,
    a mono-type substitution is performed on $\all A$,
    resulting in $[\tau/a]A$.
    Since $\tau$ is a mono-type, the result type reduces the number of $\forall$'s,
    and thus reduces the size measure.
\end{proof}

The two new declarative rules $\mathtt{Decl\top}$ and $\mathtt{Decl{\bot}App}$ are
discovered through it and verified against it.
To the best of the authors' knowledge,
they are the minimal set of rules that keep the lemma hold.


\paragraph{Subtyping Transitivity}

The transitivity lemma for declarative subtyping is a commonly expected property.
The proof depends on the following subtyping derivation size relation and an auxiliary lemma.

\begin{definition}[Subtyping Derivation Size]
    \begin{gather*}
        \begin{aligned}
            |1 \le 1| &= 0\\
            |a \le a| &= 0\\
            |A \le \top| &= 0\\
            |\bot \le B| &= 0\\
            |A_1 \to A_2 \le B_1 \to B_2| &= |B_1 \le A_1| + |A_2 \le B_2| + 1\\
            |\all A \le B| &= |[\tau/a]A \le B| + 1\\
            |A \le \all B| &= |A \le B| + 1
        \end{aligned}
    \end{gather*}
\end{definition}

\begin{lemma}[Subtyping Substiting Mono-type]
    If $\Psi \vdash \tau$ and $\Psi, a, \Psi_R \vdash A \le B$, then
    $\Psi, [\tau/a]\Psi_R \vdash [\tau/a]A \le [\tau/a]B$.
\end{lemma}

\begin{proof}
    A routine induction on the subtyping relation $\Psi, a, \Psi_R \vdash A \le B$
    finishes the proof.
\end{proof}

\begin{corollary}[Subtyping Substiting Type Var with a Mono-type]
    \label{cor:subtyping_subst_mono}
    If $\Psi \vdash \tau$ and $\Psi, a \vdash A \le B$, then
    $\Psi \vdash [\tau/a]A \le [\tau/a]B$.
\end{corollary}

\begin{lemma}[Subtyping Transitivity]
    If $\Psi \vdash A \le B$ and $\Psi \vdash B \le C$ then
    $\Psi \vdash A \le C$.
\end{lemma}

\begin{proof}
    Induction on $\langle |B|_\forall, |A \le B| + |B \le C| \rangle$.
    Most cases preserve the first element of the size measures $|B|_\forall$,
    and are relatively easy to prove.
    The difficult case is when $B$ is a polymorphic type,
    when the conditions are $\Psi \vdash A \le \all B$ and $\Psi \vdash \all B \le C$.
    They are derived through rules $\mathtt{{\le}\forall L}$ and
    $\mathtt{{\le}\forall R}$, respectively.
    Therefore, we have $\Psi, a \vdash A \le B$ and $\Psi \vdash [\tau/a] B \le C$.
    To exploit the induction hypothesis, the contexts should be unified.
    By Corollary~\ref{cor:subtyping_subst_mono}, $\Psi \vdash A \le [\tau/a]B$.
    Notice that the freshness condition is implicit for rule $\mathtt{{\le}\forall L}$.
    Clearly, $|[\tau/a]B|_\forall < |\all B|_\forall$, i.e. the first size measure decreases.
    By induction hypothesis we get $\Psi \vdash A \le C$ and finishes this case.
\end{proof}

\subsection{Soundness}

Our algorithm is sound with respect to the declarative system.
For any worklist $\Gm$ that reduces successfully,
there is a valid instantiation $\Om$ that transfers all judgments
to the declarative system.
\begin{theorem}[Soundness]
If \emph{wf }$\Gm$ and $\Gm \redto \nil$,
then there exists $\Om$ s.t. $\Gm\sto\Om$ and $\Om\redto\nil$.
\end{theorem}

\subsection{Partial Completeness of Subtyping: Rank-1 Restriction}

The algorithm is incomplete due to the subtyping rules 14, 15, 20 and 21.
However, subtyping is complete with respect to the declarative system in rank-1 setting.

\paragraph{Declarative Rank-1 Restriction}

Rank-1 types are also named type schemes in Hindley-Milner type system.
$$\begin{aligned}
    \text{Declarative Type Schemes}\qquad&\sigma &::=&\quad \all \sigma \mid \tau\\
\end{aligned}$$
In other words, the universal quantifiers only appear in the top level
of all polymorphic types.

For declarative subtyping, a judgment must be of form $\sigma_1 \le \sigma_2$.

\subsection{Algorithmic Rank-1 Restriction (Partial Completeness)}

The algorithmic mono-types and type schemes are defined as following:
$$\begin{aligned}
    \text{Algorithmic Mono-types}\qquad&\tau_A &::=&\quad
        1 \mid \top \mid \bot \mid a \mid A\to B \mid \al\\
    \text{Algorithmic Type Schemes}\qquad&\sigma_A &::=&\quad \all \sigma_A \mid \tau_A\\
\end{aligned}$$

Starting from the declarative judgment $\sigma_1 \le \sigma_2$,
the algorithmic derivation might involve different other kinds of judgments.
The following derivation, as an example, shows how a rank-1 judgment derives.

$$\begin{aligned}
           & \cdot \Vdash \all a \to a \le \all[b] (b \to b) \to (b \to b)\\
    \rrule{8} & b \Vdash \all a \to a \le (b \to b) \to (b \to b)\\
    \rrule{7} & b, \al \Vdash \al \to \al \le (b \to b) \to (b \to b)\\
    \rrule{6} & b, \al \Vdash \al \le b \to b \Vdash b \to b \le \al\\
    \rrule{} & \cdots
\end{aligned}$$

In this derivation, we begin from a judgment of the form $\sigma \le \sigma$.
After rule 8 is applied, the judgment becomes $\sigma \le \tau$,
since the right-hand-side polymorphic type is reduced to a declarative mono-type.
Then, rule 7 introduces existential variables to the left-hand-side,
resulting in a judgment like $\tau_A \le \tau$,
or $\sigma_A \le \tau$ in a more general case.
Finally, rule 6 breaks a judgment between functions into two sub-judgments,
which swaps the positions of the argument types
and creates a judgment like $\tau \le \tau_A$.
Notice that $\sigma_A$ is not possible to occur to the right
because the function type may not contain any polymorphic types as its argument type.

\jimmy{TODO the detailed analysis, perhaps a graph?}
After a detailed analysis on the judgments derivations,
we found that the only possible judgments that a rank-1 declarative subtyping judgment
might step to belong to the following two categories:
$$\sigma_A \le \sigma \quad\text{or}\quad \tau \le \sigma_A$$

An interesting observation is that $\al \le \bt$ does not belong to either category,
neither does $\al \le A \to B$ when $\al \in \text{FV}(A \to B)$.
Therefore, in the rank-1 setting, both cases of incompleteness never occur,
and our algorithm is complete.

\begin{theorem}[Completeness of Rank-1 Subtyping]
    Given $\Om \vdash \sigma_1 \le \sigma_2$,
    \begin{itemize}
        \item If $\Gm \Vdash \sigma_A \le \sigma \sto \Om \Vdash \sigma_1 \le \sigma_2$
            \\then $\Gm \Vdash \sigma_A \le \sigma \redto \nil$;
        \item If $\Gm \Vdash \tau \le \sigma_A \sto \Om \Vdash \sigma_1 \le \sigma_2$
            \\then $\Gm \Vdash \tau \le \sigma_A \redto \nil$.
    \end{itemize}
\end{theorem}

\subsection{Termination}

The measure used in ICFP no longer works because subtyping judgments like
$$\al \le \bot \to \top$$
cause $\al$ to split into $\al[1] \to \al[2]$, without solving any part of it,
resulting in an increased number of existential variables
and possibly increased complexity of the worklist through the size-increasing substitution
$\{\al := \al[1] \to \al[2]\}$.

We have performed a large set of tests on generated subtyping judgments
that are consist of algorithmic mono types,
and all judgements terminated within a reasonable number of derivation depth.
Unfortunately, we have not yet find any formal proof for the termination statement.

\subsection{Formalization in the Abella Proof Assistant}
