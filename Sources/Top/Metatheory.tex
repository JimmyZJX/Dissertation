
\section{Metatheory}

In this section we present several properties that are formally verified.
For the declarative system, the typing subsumption and subtyping
transitivity lemmas are discussed in detail.
The algorithmic system is proven to be sound with respect
to the declarative system via a transfer relation.
A partial completeness theorem is shown under the rank-1 restriction.
We then briefly describe the challenges we face when proving termination.
Lastly, proof statistics of Abella are discussed.

\subsection{Declarative Properties}\label{sec:meta:decl}

\paragraph{The Typing Subsumption Lemma.}
An important desired property for a type system is \emph{checking subsumption},
which basically says that any expression can
check against any super type of its actual type.
Since our bidirectional type system defines the checking mode, inference mode and
application inference mode mutually,
we formalize the generalized \emph{typing subsumption}.

First of all, we give the definition of worklist subtyping,
which is used to further generalize the typing subsumption lemma.
This is necessary because rules like $\mathtt{Decl{\to}I}$
will push the argument type $A$ into the context,
thus when checking against a super type of $A \to B$, say $C \to D$,
will cause the bind of $x$ in the context to a subtype of $A$ (since $C \le A$).

\begin{definition}[Worklist Subtyping]
    Worklist subtyping compares the type of variables bound in the worklist.
    $\Psi <: \Psi'$ iff each binding in $\Psi$ is converted to one with a super type.
    \begin{gather*}
        \inferrule*[right=$\mathtt{<:nil}$]
            {~}{\cdot <: \cdot}
        \qquad
        \inferrule*[right=$\mathtt{<:ty}$]
            {\Psi <: \Psi'}
            {\Psi, a <: \Psi', a}
        \\
        \inferrule*[right=$\mathtt{<:of}$]
            {\Psi' \vdash A \le B \\ \Psi <: \Psi'}
            {\Psi, x:A <: \Psi', x:B}
        \qquad
        \inferrule*[right=$\mathtt{<:\omega}$]
            {\Psi <: \Psi'}
            {\Psi \Vdash \jg <: \Psi' \Vdash \jg}
    \end{gather*}
\end{definition}

A basic property of worklist subtyping is that they acts similarly when
dealing with subtyping between well-formed types.
\begin{lemma}[Worklist Subtyping Equivalence]~\\
    Given $\Psi <: \Psi'$, $\Psi \vdash A \le B \Longleftrightarrow \Psi' \vdash A \le B$.
\end{lemma}

Finally, we give the statement of typing subsumption lemma,
which is generalized by the worklist subtyping relation.

\begin{lemma}[Typing Subsumption]
    Given $\Psi <: \Psi'$,
    \begin{enumerate}[1)]
        \item If $\Psi' \vdash e \Lto A$ and $\Psi' \vdash A \le B$, then $\Psi \vdash e \Lto B$;
        \item If $\Psi' \vdash e \To A$, then $\exists B$ s.t. $\Psi' \vdash B \le A$ and $\Psi \vdash e \To B$.
        \item If $\Psi' \vdash \appInf{C}{e}{A}$ and $\Psi' \vdash D \le C$, then
            $\exists B$ s.t. $\Psi' \vdash B \le A$ and $\Psi \vdash \appInf{D}{e}{B}$.
    \end{enumerate}
\end{lemma}

\begin{proof}
    By induction on the following size measure (lexicographical order on a 3-tuple):
    \begin{itemize}
        \item Checking ($e \Lto A$): $\langle |e|, 1, |A|_\forall + |B|_\forall \rangle$
        \item Inference ($e \To A$): $\langle |e|, 0, 0 \rangle$
        \item Application inference ($\appInf{A}{e}{C}$): $\langle |e|, 2, |C|_\forall + |D|_\forall \rangle$
    \end{itemize}
    Most of the cases are straightforward.
    When rule $\mathtt{{\le}\forall L}$ is applied for the subtyping predicate
    like $\Psi' \vdash A \le B$,
    a mono-type substitution is performed on $\all A$,
    resulting in $[\tau/a]A$.
    Since $\tau$ is a mono-type, the result type reduces the number of $\forall$'s,
    and thus reduces the size measure.
\end{proof}

Interestingly, the two new declarative rules
$\mathtt{Decl\top}$ and $\mathtt{Decl{\bot}App}$ are
discovered when we were trying to prove the property instead of
before exploring the meta-theory.
Given the typing and subtyping judgments $\Psi \vdash e \Lto A$ and $\Psi \vdash A \le \top$,
we should derive $\Psi \vdash e \Lto \top$ from the lemma,
therefore Rule $\mathtt{Decl\top}$ is required,
saying that any expression can be checked against the top type.
Similarly, the most general type $\bot$,
being able to convert to any type due to Rule $\mathtt{{\le}Bot}$,
can be converted to any function type,
or simply the most general one $\top \to \bot$,
which accepts any input and returns the $\bot$ type,
resulting in the derivation $\Psi \vdash \appInf{\bot}{e}{\bot}$.
From the lemma we can also derive that by
$\Psi \vdash \appInf{C}{e}{A}$, $\Psi \vdash \bot \le C$ and $\Psi \vdash \bot \le A$.

With the addition of Rules $\mathtt{Decl\top}$ and $\mathtt{Decl{\bot}App}$,
we can prove the typing subsumption lemma.
To the best of the author's knowledge,
they are the minimal set of rules that make the lemma hold.


\paragraph{Transitivity of Subtyping}

The transitivity lemma for declarative subtyping is a commonly expected property.
The proof depends on the following subtyping derivation size relation and an auxiliary lemma.

\begin{definition}[Subtyping Derivation Size]
    \begin{gather*}
        \begin{aligned}
            |1 \le 1| &= 0\\
            |a \le a| &= 0\\
            |A \le \top| &= 0\\
            |\bot \le B| &= 0\\
            |A_1 \to A_2 \le B_1 \to B_2| &= |B_1 \le A_1| + |A_2 \le B_2| + 1\\
            |\all A \le B| &= |[\tau/a]A \le B| + 1\\
            |A \le \all B| &= |A \le B| + 1
        \end{aligned}
    \end{gather*}
\end{definition}

\begin{lemma}[Monotype Subtyping Substitution]
    If $\Psi \vdash \tau$ and $\Psi, a, \Psi_R \vdash A \le B$, then
    $\Psi, [\tau/a]\Psi_R \vdash [\tau/a]A \le [\tau/a]B$.
\end{lemma}

\begin{proof}
    A routine induction on the subtyping relation $\Psi, a, \Psi_R \vdash A \le B$
    finishes the proof.
\end{proof}

\begin{corollary}[Monotype Subtyping Substitution for Type Variables]
    \label{cor:subtyping_subst_mono}
    If $\Psi \vdash \tau$ and $\Psi, a \vdash A \le B$, then
    $\Psi \vdash [\tau/a]A \le [\tau/a]B$.
\end{corollary}

The above lemma and corollary reveal the fact
that a type variable occurred in the subtyping relation
represents an \emph{arbitrary} well-formed monotype.
And it also explains the difference in treatment of polymorphic types between
Rules $\mathtt{{\le}{\forall}L}$ and $\mathtt{{\le}{\forall}R}$:
Rule $\mathtt{{\le}{\forall}R}$ is in fact equivalent to:
$$
\inferrule*[right=$\mathtt{{\le}{\forall}R'}$]
    {\forall \tau \text{ s.t. } \Psi \vdash \tau \Longrightarrow \Psi \vdash A \le [\tau/b]B}
    {\Psi \vdash A \le \all[b] B}
$$

Finally, with the size measure defined and required lemma proven,
we can obtain the transitivity lemma for declarative subtyping.

\begin{lemma}[Subtyping Transitivity]
    If $\Psi \vdash A \le B$ and $\Psi \vdash B \le C$ then
    $\Psi \vdash A \le C$.
\end{lemma}

\begin{proof}
    Induction on the lexicographical order defined by
    $\langle |B|_\forall, |A \le B| + |B \le C| \rangle$.
    Most cases preserve the first element of the size measures $|B|_\forall$,
    and are relatively easy to prove.
    The difficult case is when $B$ is a polymorphic type,
    when the conditions are $\Psi \vdash A \le \all B$ and $\Psi \vdash \all B \le C$.
    They are derived through rules $\mathtt{{\le}\forall L}$ and
    $\mathtt{{\le}\forall R}$, respectively.
    Therefore, we have $\Psi, a \vdash A \le B$ and $\Psi \vdash [\tau/a] B \le C$.
    To exploit the induction hypothesis, the contexts should be unified.
    By Corollary~\ref{cor:subtyping_subst_mono}, $\Psi \vdash A \le [\tau/a]B$.
    Notice that the freshness condition is implicit for rule $\mathtt{{\le}\forall L}$.
    Clearly, $|[\tau/a]B|_\forall < |\all B|_\forall$, i.e. the first size measure decreases.
    By induction hypothesis we get $\Psi \vdash A \le C$ and finish this case.
\end{proof}

\subsection{Transfer}

\begin{figure}[t]
    \begin{gather*}
    \begin{aligned}
    \text{Declarative worklist}\qquad&\Om &::=&\quad \nil \mid \Om, a \mid \Om, x: A \mid \Om \Vdash \jg
    \end{aligned}
    \end{gather*}
    \hfill \framebox{$\Gm \sto \Om$} \hfill $\Gm$ instantiates to $\Om$.
    \begin{gather*}
    \inferrule*[right=$\mathtt{{\sto}}\Om$]
    {~}
    {\Om \sto \Om}
    \quad
    \inferrule*[right=$\mathtt{{\sto}\al}$]
    {\Om\vdash\tau \\ \Om,[\tau/\al]\Gm \sto \Om}
    {\Om,\al,\Gm \sto \Om}
    \end{gather*}
    \Description{Declarative Worklists and Instantiation}
    \caption{Declarative Worklists and Instantiation}
    \label{fig:top:trans}
\end{figure}

Follow the approach of Section~\ref{sec:metatheory},
the transfer relation and the declarative instantiation relation are defined
in Figure~\ref{fig:top:trans}.

Similarly, Lemmas~\ref{lem:top:insert} and \ref{lem:top:extract}
generalizing Rule $\mathtt{{\sto}\al}$ hold as well.

\begin{lemma}[Insert]\label{lem:top:insert}
If $\Gm_L, [\tau/\al]\Gm_R \sto \Om$ and $\Gm_L\vdash \tau$
, then $\Gm_L, \al, \Gm_R \sto \Om$.
\end{lemma}
\begin{lemma}[Extract]\label{lem:top:extract}
If $\Gm_L, \al, \Gm_R \sto \Om$
, then there exists $\tau$ s.t. $\Gm_L\vdash\tau$ and $\Gm_L, [\tau/\al]\Gm_R \sto \Om$.
\end{lemma}

\begin{figure}[ht]
\hfill \framebox{$\|\Om\|$} \hfill Judgment erasure.
\begin{gather*}
\begin{aligned}
\|\nil\| &= \nil\\
\|\Om,a\| &= \|\Om\|, a\\
\|\Om,x:A\| &= \|\Om\|, x:A\\
\|\Om\Vdash\jg\| &= \|\Om\|
\end{aligned}
\end{gather*}

\hfill \framebox{$\Om \rto \Om'$} \hfill Declarative transfer.
\begin{gather*}
\begin{aligned}
\Om,a &\rto \Om \\  \Om,x:A & \rto \Om\\
\Om\Vdash A\le B &\rto \Om &\text{ when } \|\Om\| \vdash A\le B\\
\Om\Vdash e\Lto A &\rto \Om & \text{ when } \|\Om\| \vdash e\Lto A\\
\Om\Vdash e\To_a \jg &\rto \Om\Vdash[A/a]\jg & \text{ when } \|\Om\| \vdash e\To A\\
\Om\Vdash \appInfAlg{A}{e} &\rto \Om\Vdash[C/a]\jg & \text{ when } \|\Om\| \vdash \appInf{A}{e}{C}\\
\end{aligned}
\end{gather*}
\Description{Declarative Transfer}
\caption{Declarative Transfer}
\label{fig:top:decl:worklist}
\end{figure}

Figure~\ref{fig:top:decl:worklist} defines a relation $\Om \rto \Om'$,
checking that every judgment entry in the worklist
holds using a corresponding declarative judgment.

\subsection{Soundness}

Our algorithm is sound with respect to the declarative system.
For any worklist $\Gm$ that reduces successfully,
there is a valid instantiation $\Om$ that transfers all judgments
to the declarative system.
\begin{theorem}[Soundness]
If \emph{wf }$\Gm$ and $\Gm \redto \nil$,
then there exists $\Om$ s.t. $\Gm \sto \Om$ and $\Om \redto \nil$.
\end{theorem}

Soundness is a basic desired property of a type inference algorithm,
which ensures that the algorithm is always producing
valid declarative derivations when the judgments are accepted.

\subsection{Partial Completeness of Subtyping: Rank-1 Restriction}

The algorithm is incomplete due to the subtyping rules 14, 15, 20 and 21.
However, subtyping is complete with respect to the declarative system in a rank-1 setting.

\paragraph{Declarative Rank-1 Restriction}

Rank-1 types are also named type schemes in the Hindley-Milner type system.
$$\begin{aligned}
    \text{Declarative Type Schemes}\qquad&\sigma &::=&\quad \all \sigma \mid \tau\\
\end{aligned}$$
In other words, the universal quantifiers only appear in the top level
of all polymorphic types.

For declarative subtyping, a judgment must be of form $\sigma_1 \le \sigma_2$.

\subsection{Algorithmic Rank-1 Restriction (Partial Completeness)}

The algorithmic mono-types and type schemes are defined as following:
$$\begin{aligned}
    \text{Algorithmic Mono-types}\qquad&\tau_A &::=&\quad
        1 \mid \top \mid \bot \mid a \mid A\to B \mid \al\\
    \text{Algorithmic Type Schemes}\qquad&\sigma_A &::=&\quad \all \sigma_A \mid \tau_A\\
\end{aligned}$$

Starting from the declarative judgment $\sigma_1 \le \sigma_2$,
the algorithmic derivation might involve different other kinds of judgments.
The following derivation, as an example, shows how a rank-1 judgment derives.

$$\begin{aligned}
           & \cdot \Vdash \all a \to a \le \all[b] (b \to b) \to (b \to b)\\
    \rrule{8} & b \Vdash \all a \to a \le (b \to b) \to (b \to b)\\
    \rrule{7} & b, \al \Vdash \al \to \al \le (b \to b) \to (b \to b)\\
    \rrule{6} & b, \al \Vdash \al \le b \to b \Vdash b \to b \le \al\\
    \rrule{} & \cdots
\end{aligned}$$

In this derivation, we begin from a judgment of the form $\sigma \le \sigma$.
After Rule 8 is applied, the judgment becomes $\sigma \le \tau$,
since the right-hand-side polymorphic type is reduced to a declarative mono-type.
Then, Rule 7 introduces existential variables to the left-hand-side,
resulting in a judgment like $\tau_A \le \tau$,
or $\sigma_A \le \tau$ in a more general case.
Finally, Rule 6 breaks a judgment between functions into two sub-judgments,
which swaps the positions of the argument types
and creates a judgment like $\tau \le \tau_A$.
Notice that $\sigma_A$ is not possible to occur to the right
because the function type may not contain any polymorphic types as its argument type.

After a detailed analysis on the judgments derivations,
we found that the only possible judgments that a rank-1 declarative subtyping judgment
might step to belong to the following two categories:
$$\sigma_A \le \sigma \quad\text{or}\quad \tau \le \sigma_A$$
All the possible judgment types shown above fall into these categories.
For example, $\tau_A \le \tau$ is a special form of $\sigma_A \le \sigma$,
and $\tau \le \tau_A$ belongs to $\tau \le \sigma_A$.

An interesting observation is that $\al \le \bt$ does not belong to either category,
neither does $\al \le A \to B$ when $\al \in \text{FV}(A \to B)$.
Therefore, in the rank-1 setting, both cases of incompleteness never occur,
and our algorithm is complete.

\begin{theorem}[Completeness of Rank-1 Subtyping]
    Given $\Psi \vdash \sigma_1 \le \sigma_2$,
    \begin{itemize}
        \item If $\Gm \Vdash \sigma_A \le \sigma \sto \Psi \Vdash \sigma_1 \le \sigma_2$
            \\then $\Gm \Vdash \sigma_A \le \sigma \redto \nil$;
        \item If $\Gm \Vdash \tau \le \sigma_A \sto \Psi \Vdash \sigma_1 \le \sigma_2$
            \\then $\Gm \Vdash \tau \le \sigma_A \redto \nil$.
    \end{itemize}
\end{theorem}

\subsection{Termination}

The measure used in Chapter~\ref{chap:ICFP} no longer works because subtyping judgments like
$$\al \le \bot \to \top$$
cause $\al$ to split into $\al[1] \to \al[2]$, without solving any part of it,
resulting in an increased number of existential variables
and possibly increased complexity of the worklist through the size-increasing substitution
$\{\al := \al[1] \to \al[2]\}$.

We have performed a large set of tests on generated subtyping judgments
that are consist of algorithmic monotypes,
and all judgments terminated within a reasonable number of derivation depth.
Unfortunately, we have not yet find any formal proof for the termination statement.

\subsection{Formalization in the Abella Proof Assistant}

We have chosen the Abella (v2.0.7-dev
\footnote{We use a forked version \url{https://github.com/JimmyZJX/abella} by only enhancing the Abella prover with a handy
``applys'' tactic.}
) proof assistant~\citep{AbellaDesc} to develop our formalization.
Equipped with HOAS, Abella eases the formalization and proof tasks a lot
compared with various libraries in Coq.
Additionally, our algorithm heavily uses eager substitutions,
and Abella greatly simplifies relevant proofs thanks to its built-in
substitution representation and higher-order unification algorithms.

The reader may find the source code of the proof at \url{https://github.com/JimmyZJX/Dissertation/tree/main/src/Subtyping}.

\paragraph{Statistics of the Proof}
The proof script consists of 7,301 lines of Abella code with a total of
48 definitions and 592 theorems.
Figure~\ref{table:top:proof_statistics} briefly summarizes the contents of each file.
The files are linearly dependent due to the limitations of Abella.

\begin{table}[t]
    \renewcommand{\arraystretch}{1.2}
    \caption{Statistics for the proof scripts}
    \centering\begin{tabular}{@{}lrrl@{}}
    \toprule
        File(s) & LOC & \#Thm & Description\\
    \midrule
        olist.thm, nat.thm  &   311 & 57  & Basic data structures\\
        typing.thm          &   273 & 7   & Declarative \& algorithmic system, debug examples\\
        decl.thm            &   241 & 33  & Basic declarative properties\\
        order.thm           &   274 & 27  & The $|\cdot|_\forall$ measure; decl. subtyping strengthening\\
        alg.thm             &   699 & 82  & Basic algorithmic properties\\
        trans.thm           &   635 & 53  & \makecell[l]{Worklist instantiation and declarative transfer;\\
                                Lemmas~\ref{lem:insert}, \ref{lem:extract}}\\
        declTyping.thm      & 1,087 & 76  & \makecell[l]{Non-overlapping declarative system; \\
                                Lemmas~\ref{lem:inv_allR}, \ref{lem:inv_chkAll},
                                    \ref{lem:inv_chkLam}, \ref{lem:subsumption}}\\
        soundness.thm       & 1,206 & 81  & Soundness theorem; aux. lemmas on transfer\\
        dcl.thm             &   417 & 12  & Non-overlapping declarative worklist \\
        scheme.thm          & 1,113 & 98  &
                                Type scheme (rank-1 restriction)\\
        completeness.thm    & 1,045 & 63  &
                                Completeness theorem; aux. lemmas and relations\\
    \midrule
        \emph{Total}        & 7,301 & 592 & (48 definitions in total)\\
    \bottomrule
    \end{tabular}
    \label{table:top:proof_statistics}
\end{table}



