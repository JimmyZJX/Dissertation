
\section{Declarative System}

\begin{figure}[t]
    \begin{gather*}
    \begin{aligned}
        \text{Type variables}\qquad&a, b\\
        \text{Types}\qquad&A, B, C &::=&\quad 1 \mid \top \mid \bot \mid a \mid \all A \mid A\to B\\
        \text{Monotypes}\qquad&\tau &::=&\quad 1 \mid \top \mid \bot \mid a \mid \tau_1\to \tau_2\\
        \text{Expressions}\qquad&e &::=&\quad x \mid () \mid \lam e \mid e_1~e_2 \mid (e:A)\\
        \text{Context}\qquad&\Psi &::=&\quad \cdot \mid \Psi, a \mid \Psi, x:A
    \end{aligned}
    \end{gather*}
\Description{Declarative Syntax}
\caption{Declarative Syntax}\label{fig:top_decl_syntax}
\end{figure}

\paragraph{Syntax}
The syntax of the declarative system, shown in Figure~\ref{fig:top_decl_syntax},
is similar to the previous systems by having
a primitive type $1$, type variables $a$,
polymorphic types $\all A$ and function types $A \to B$.
Additionally, top and bottom types are introduced to the type system.
% intro of top/bot moved to intro

The well-formedness formalization of the system is standard
and almost identical to the previous systems,
therefore we omit the formal definitions.


\begin{figure}[t]
    \framebox{$\Psi \vdash A \le B$}
    \begin{gather*}
    \inferrule*[right=$\mathtt{{\le}Var}$]
    {a\in\Psi}{\Psi\vdash a\le a}
    \qquad
    \inferrule*[right=$\mathtt{{\le}Unit}$]
    {~}{\Psi \vdash 1 \le 1}
    \qquad
    \inferrule*[right=$\mathtt{{\le}{\to}}$]
    {\Psi \vdash B_1 \le A_1 \quad \Psi \vdash A_2 \le B_2}
    {\Psi\vdash A_1\to A_2 \le B_1\to B_2}
    \\
    \inferrule*[right=$\mathtt{{\le}\forall L}$]
    {\Psi\vdash \tau \quad \Psi\vdash [\tau/a] A \le B}
    {\Psi\vdash \all A \le B}
    \qquad
    \inferrule*[right=$\mathtt{{\le}\forall R}$]
    {\Psi, b\vdash A\le B}
    {\Psi\vdash A \le \all[b]B}
    \\
    \pmb{
    \inferrule*[right=$\mathtt{{\le}Top}$]
    {~}
    {A \le \top}
    }
    \qquad
    \pmb{
    \inferrule*[right=$\mathtt{{\le}Bot}$]
    {~}
    {\bot \le A}
    }
    \end{gather*}
\Description{Declarative Subtyping}
\caption{Declarative Subtyping}\label{fig:top_decl_subtyping}
\end{figure}

\paragraph{Declarative Subtyping}
Shown in Figure~\ref{fig:top_decl_subtyping},
the declarative subtyping extends the polymorphic subtyping relation
originally proposed by \citet{odersky1996putting}
by adding rules $\mathtt{{\le}Top}$ and $\mathtt{{\le}Bot}$,
defining the properties of the $\top$ and $\bot$ types, respectively.
Although the new rules seem quite simple,
they may increase the uncertainty of polymorphic instantiations.
For example, the subtyping judgment
\[\all a \to a \le \bot \to \top\]
accepts any well-formed instantiation on the polymorphic type $\all a \to a$.


\begin{figure}[t]
    \begin{tabular}{rl}
        \framebox{$\Psi \vdash e \Lto A$} & $e$ checks against input type $A$.\\[0.5mm]
        \framebox{$\Psi \vdash e \To A$} & $e$ synthesizes output type $A$.\\[0.5mm]
        \framebox{$\Psi \vdash \appInf{A}{e}{C}$} & Applying a function of type $A$ to $e$ synthesizes type $C$.
    \end{tabular}
    \begin{gather*}
    \inferrule*[right=$\mathtt{DeclVar}$]
        {(x:A)\in\Psi}{\Psi\vdash x\To A}
    \qquad
    \inferrule*[right=$\mathtt{DeclSub}$]
    %e \neq \lam e' \quad B \neq \all B' \quad 
        {\Psi\vdash e\To A \quad \Psi\vdash A\le B}
        {\Psi \vdash e\Lto B}
    \\
    \inferrule*[right=$\mathtt{DeclAnno}$]
        {\Psi \vdash A \quad \Psi\vdash e\Lto A}
        {\Psi\vdash (e:A)\To A}
    \qquad
    \inferrule*[right=$\mathtt{Decl1I{\To}}$]
        {~}{\Psi\vdash () \To 1}
    \\
    \inferrule*[right=$\mathtt{Decl1I}$]
        {~}{\Psi\vdash () \Lto 1}
    \qquad
    \inferrule*[right=$\mathtt{Decl\top}$]
        {\Psi \vdash e}
        {\Psi\vdash e \Lto \top}
    \qquad
    \inferrule*[right=$\mathtt{Decl{\bot}App}$]
        {\Psi\vdash e}
        {\Psi\vdash \appInf{\bot}{e}{\bot}}
    \\
    \inferrule*[right=$\mathtt{Decl\forall I}$]
        {\Psi,a \vdash e \Lto A}
        {\Psi\vdash e\Lto \all A}
    \qquad
    \inferrule*[right=$\mathtt{Decl\forall App}$]
        {\Psi \vdash \tau \quad \Psi\vdash \appInf{[\tau/a]A}{e}{C} }
        {\Psi\vdash \appInf{\all A}{e}{C}}
    \\
    \inferrule*[right=$\mathtt{Decl{\to}I}$]
        {\Psi,x:A \vdash e\Lto B}
        {\Psi\vdash \lam e \Lto A \to B}
    \qquad
    \inferrule*[right=$\mathtt{Decl{\to}I{\To}}$]
        {\Psi\vdash \sigma\to\tau \quad \Psi,x:\sigma \vdash e\Lto \tau}
        {\Psi\vdash \lam e \To \sigma\to\tau}
    \\
    \inferrule*[right=$\mathtt{Decl{\to} E}$]
        {\Psi\vdash e_1\To A \quad \Psi\vdash \appInf{A}{e_2}{C}}
        {\Psi\vdash e_1~e_2 \To C}
    \qquad
    \inferrule*[right=$\mathtt{Decl{\to}App}$]
        {\Psi\vdash e \Lto A}
        {\Psi\vdash \appInf{A \to C}{e}{C}}
    \end{gather*}
\Description{Declarative Typing}
\caption{Declarative Typing}\label{fig:top_decl_typing}
\end{figure}

\paragraph{Declarative Typing}

The declarative typing rules, shown in Figure~\ref{fig:top_decl_typing},
extend DK's higher-ranked type system in order to support the top and bottom types.
Rule $\mathtt{Decl\top}$ allows any well-formed expression to check against $\top$.
Rule $\mathtt{Decl{\bot}App}$ returns the $\bot$ type
when a function of $\bot$ type is applied to any argument.
All other rules remain exactly the same as the system of
Chapters~\ref{chap:ITP} and \ref{chap:ICFP}.

It's worth mentioning that the design of the two new rules
is driven by the subsumption property described in Section~\ref{sec:meta:decl}.
They maintain the property in presence of a more powerful declarative subtyping,
and we will discuss further later in that part.

\setcounter{algRuleCounter}{0}
