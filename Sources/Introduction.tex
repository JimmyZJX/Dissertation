%%%%%%%%%%%%%%%%%%%%%%%%%%%%%%%%%%%%%%%%%%%%%%%%%%%%%%%%%%%%%%%%%%%%%%%%
\chapter{Introduction}
%%%%%%%%%%%%%%%%%%%%%%%%%%%%%%%%%%%%%%%%%%%%%%%%%%%%%%%%%%%%%%%%%%%%%%%%


\section{Type Systems and Type Inference Algorithms}

Statically typed programming languages are widely used nowadays.
Programs are categorized by various types before it is compiled and executed.
Type errors caught before execution usually indicate potential bugs,
letting the programmers to realize and correct in advance.

In early stages, programming languages like C is built on a simple explicit type system,
where only primitive types, function types and records are supported.
People soon realize the need to generalize similar programs that have different types when used.
For example, a typical way to define a swap function on integers is
\begin{verbatim}
  void swap(int *x, int *y) {
    int t = *x; *x = *y; *y = *t;
  }
\end{verbatim}
Similarly, the swap function for float values can be defined as
\begin{verbatim}
  void swap(float *x, float *y) {
    float t = *x; *x = *y; *y = *t;
  }
\end{verbatim}
which mostly shares a the same body with the above function,
except that the types changes from \verb|int| to \verb|float|.
Therefore, a type-safe approach to get a generic function for free is by
text generation (pre-processing) macros:
\begin{verbatim}
  #define SWAP(X, Y, T) { T _tmp = X; X = Y; Y = _tmp; }
\end{verbatim}

Macros provides a generic way to program while maintaining type-safety,
but the drawback of macro programming is also obvious:
the type information for the abstraction is neither declared nor verified by
the type system.
In fact, users are responsible for the correctness of an expression after substitution.

In comparison, C++ provides a better way to define such a generic function:
\begin{verbatim}
  template<typename T> void swap(T& t1, T& t2) {
    T tmp(t1); t1 = t2; t2 = tmp;
  }
\end{verbatim}
where \verb|T| denotes the generic type that programmers may arbitrarily pick.
The C++ \verb|swap| function utilize the feature of type system, template,
to improve modularity.
In other words, introduction of good features of type systems
accepts more meaningful programs.

On the other hand, being able to accept all syntactically correct programs,
as dynamically-typed programming languages do,
are not desirable as well.
Ill-typed programs are easy to write by mistake, like \verb|"3" / 3|;
or even well-typed programs with ambiguous/unexpected meaning like \verb|"3" + 4 + 5|
(for someone who is not familiar with the conventions).



\subsection{System F and Functional Programming}


\subsection{Type Inference for Parametric Polymorphism}


Most statically typed functional languages support a form of
\emph{(implicit) parametric polymorphism}~\cite{reynolds1983types}.
Traditionally, functional languages have employed variants of the
Hindley-Milner~\cite{hindley1969principal,milner1978theory,damas1982principal}
type system, which supports full type-inference without any type annotations.
However the Hindley-Milner type system only supports \emph{first-order
polymorphism}, where all universal quantifiers only occur at the top-level
of a type.  Modern functional programming languages such as Haskell go beyond
Hindley-Milner and support \emph{higher-order polymorphism}. With higher-order
polymorphism there is no restriction on where universal quantifiers can occur.
This enables more code reuse and more expressions to type-check, and has
numerous applications~\cite{jones1995functional,gill1993short,launchbury1995state,lammel2003scrap}.


Modern functional programming languages, such as Haskell or OCaml,
use sophisticated forms of type inference. The type systems of these languages are
descendants of
Hindley-Milner~\cite{hindley1969principal,milner1978theory,damas1982principal}, 
which was revolutionary at the
time in allowing type-inference to proceed without any type
annotation. The traditional Hindley-Milner type system supports
top-level \emph{implicit (parametric) polymorphism}~\cite{reynolds1983types}. With implicit
polymorphism, type arguments of polymorphic functions are
automatically instantiated. Thus implicit polymorphism and the absence of
type annotations mean that the Hindley-Milner type system 
strikes a great balance between expressive power and usability. 



\section{Challenges and Motivations}

\subsection{Higher-Ranked Type Inference}

``predicative implicit higher-rank polymorphism''

Unfortunately, with higher-order polymorphism full type-inference becomes
undecidable~\cite{wells1999typability}. To recover decidability some type annotations 
on polymorphic arguments
are necessary. 
A canonical example that requires higher-order polymorphism in Haskell is:
\begin{verbatim}
hpoly = (\f :: forall a. a -> a) -> (f 1, f 'c')
\end{verbatim}
The function \verb|hpoly| cannot be
type-checked in Hindley-Milner.  The type of \verb|hpoly| is \\
\verb|(forall a. a -> a) -> (Int, Char)|. The single universal quantifier
does not appear at the top-level. Instead it is used to quantify a
type variable \verb|a| used in the first argument of the
function.
Notably \verb|hpoly| requires a type annotation for the first
argument (\verb|forall a. a -> a|). 
Despite these additional annotations,
the type-inference algorithm employed by GHC Haskell~\cite{jones2007practical} preserves 
many of the desirable properties of Hindley-Milner. 
Like in Hindley-Milner type instantiation is \emph{implicit}. That is,
calling a polymorphic function never requires the programmer to 
provide the instantiations of the type parameters.

Central to type-inference with \emph{higher-order polymorphism} is an
algorithm for polymorphic subtyping. 
This algorithm
allows us to check whether one type is more general than another,
which is essential to detect valid instantiations of a polymorphic
type. For example, the type \verb|forall a. a -> a| is more
general than \verb|Int -> Int|. 
% Similarly, \verb|(Int -> Int) -> (Int, Char)| is more general than \verb|(forall a . a -> a) -> (Int, Char)|
% (due to contravariance of argument types). 
A simple declarative polymorphic subtyping relation
was proposed by Odersky and L\"aufer~\cite{odersky1996putting}. Since then several
algorithms have been proposed that implement it. Most
notably, the algorithm proposed by Peyton Jones et al.~\cite{jones2007practical} forms the basis
for the implementation of type inference in the GHC compiler. 
Dunfield and Krishnaswami~\cite{dunfield2013complete} provided a very elegant
formalization of another sound and complete algorithm, which has 
also inspired implementations of type-inference in some polymorphic 
programming languages (such as PureScript~\cite{PureScript} or DDC~\cite{Disciple}).

As functional languages evolved the need for more expressive
power has motivated language designers to look beyond Hindley-Milner.
In particular one popular direction is to allow 
\emph{higher-ranked polymorphism} where polymorphic types can
occur anywhere in a type signature.  
An important challenge is that full type inference for higher-ranked
polymorphism is known to be undecidable~\cite{wells1999typability}. Therefore some type
annotations are necessary to guide type inference. In response to this 
challenge several decidable type systems requiring some annotations 
have been proposed~\cite{,dunfield2013complete,jones2007practical,Serrano2018,le2003ml,leijen2008hmf,vytiniotis2008fph}.
Two closely related type systems that 
support \emph{predicative} higher-ranked type inference were proposed 
by Peyton Jones et al.~\cite{jones2007practical} and Dunfield and
Krishnaswami~\cite{dunfield2013complete} (henceforth denoted as DK). 
These type systems are
popular among language designers and their ideas have been adopted by
several modern functional languages, including Haskell, PureScript~\cite{PureScript} and
Unison~\cite{Unison} among others.
In those type systems
type annotations are required for polymorphic arguments of functions,
but other type annotations can be omitted. A canonical example (here written in Haskell) is:
\begin{verbatim}
    hpoly = \(f :: forall a. a -> a) -> (f 1, f 'c')
\end{verbatim}
The function \verb|hpoly| cannot be
type-checked in the Hindley-Milner type system. The type of \verb|hpoly| is the rank-2 type:
\verb|(forall a. a -> a) -> (Int, Char)|. Notably (and unlike
Hindley-Milner) the lambda argument \verb|f| requires a
\emph{polymorphic} type annotation.
This annotation is needed because the single universal quantifier
does not appear at the top-level. Instead it is used to quantify a
type variable \verb|a| used in the first argument of the
function. 
Despite these additional annotations, Peyton Jones et al. and DK's
type inference algorithms preserve many of the desirable properties 
of Hindley-Milner. For example the applications of \verb|f| implicitly 
instantiate the polymorphic type arguments of \verb|f|.


\subsection{Subtyping}


\section{Machanical Formalizations and Theorem Provers}


Although type inference is important in practice and receives a lot of
attention in
academic research, there is little work on mechanically formalizing
such advanced forms of type inference in theorem provers.
The remarkable exception is work done on the formalization of 
certain parts of Hindley-Milner type inference~\cite{naraschewski1999type,
dubois2000proving,dubois1999certification,urban2008nominal,
garrigue2015certified}. However
there is still no formalization of the higher-ranked type systems
that are employed by modern languages like Haskell.
This is at
odds with the current trend of mechanical formalizations in
programming language research. In particular both the \textsc{POPLMark}
challenge~\cite{aydemir2005mechanized} and
CompCert ~\cite{leroy2012compcert} have significantly promoted
the use of theorem provers to model various aspects of programming
languages. Today papers in various programming language venues routinely
use theorem provers to mechanically formalize: \emph{dynamic and
  static semantics} and their correctness properties~\cite{aydemir2008engineering},
\emph{compiler correctness}~\cite{leroy2012compcert}, \emph{correctness of
  optimizations}~\cite{Bertot04}, \emph{program analysis}~\cite{Chang2006}
or proofs involving \emph{logical relations}~\cite{abel2018}. The
main argument for mechanical formalizations is a simple one. Proofs
for programming languages tend to be \emph{long}, \emph{tedious} and
\emph{error-prone}. In such proofs it is very easy to make mistakes
that may invalidate the whole development. Furthermore, readers and
reviewers often do not have time to look at the proofs carefully to
check their correctness. Therefore errors can go unnoticed for a
long time.  Mechanical formalizations provide, in principle, a natural
solution for these problems. Theorem provers can automatically check and
validate the proofs, which removes the burden of checking from both
the person doing the proofs as well as readers or reviewers.



Unfortunately, while many aspects of programming languages and type systems
have been mechanically formalized in theorem provers, there is little work on
formalizing algorithms related to type-inference. The main exceptions to the rule
are mechanical formalizations of algorithm $\mathcal{W}$ and other aspects
of traditional Hindler-Milner 
type-inference~\cite{naraschewski1999type,dubois2000proving,dubois1999certification,urban2008nominal,garrigue2015certified}.
However, as far as we know, there is no mechanisation of algorithms 
used by modern functional languages like Haskell, and
polymorphic subtyping included is no exception.  
This is a shame because recently there has been a lot of effort
in promoting the use of theorem provers to check the meta-theory 
of programming languages, e.g., through well-known examples like the \textsc{POPLMark} challenge~\cite{aydemir2005mechanized} and the CompCert project~\cite{leroy2012compcert}.
Mechanical formalizations are especially valuable for proving the
correctness of the semantics and type systems 
of programming languages. Type-inference algorithms are arguably among
the most non-trivial aspects of the implementations of programming
languages. In particular the information discovery process required by 
many algorithms (through unification-like or constraint-based
approaches), is quite subtle and tricky to get right. Moreover,
extending type-inference algorithms with new programming language features is often quite 
delicate. Studying the meta-theory for such extensions would be
greatly aided by the existence of a mechanical
formalization of the base language, which could then be extended by
the language designer.

Handling variable binding is particularly challenging in type inference,
because the algorithms typically do not rely simply on local environments, but
instead propagate information across judgements. Yet, there is little work on
how to deal with these complex forms of binding in theorem provers. We believe
that this is the primary reason why theorem provers have still not been widely
adopted for formalizing type-inference algorithms.



\section{Contributions}

In summary the contributions of this thesis are:


\begin{description}
\item[\Cref{chap:ITP}]
  \begin{itemize}
  \item The first mechanical formalization of a polymorphic subtyping algorithm.
  \item Information propagation across worklist judgments with eager substitution.
  \end{itemize}
\item[\Cref{chap:ICFP}]
  \begin{itemize}
  \item A fully mechanical formalization of type inference with higher-ranked types.
  \item A new algorithm for DK's type system.
  \item Worklists with inference judgments.
  \item Unification of worklists and contexts.
  \end{itemize}
\item[\Cref{chap:Top}]
  \begin{itemize}
  \item A mechanical formalization of type inference with higher-ranked types and subtyping.
  \item A new backtracking-based sound type inference algorithm.
  \item Formalization of rank-1 restriction.
  \end{itemize}
\end{description}


\section{Organization}

This thesis is partially based on the publications by the author~\citep{},
as indicated below.
\begin{description}
\item[\cref{chap:ITP}:]~
\item[\cref{chap:ICFP}:]~
\end{description}


\noindent\makebox[\linewidth]{\rule{0.7\textwidth}{0.4pt}}

\vspace{1.5\baselineskip}



\section{Introduction}



%
% Secondly, many older type-inference algorithms typically use
% sets to collect constraints or substitutions for unification; these are hard to
% handle in the purely inductive style favored by theorem provers.


In summary the contributions of this paper are:

\begin{itemize}
\item {\bf A mechanical formalization of a polymorphic subtyping
    algorithm.} We show that the algorithm is \emph{sound},
  \emph{complete} and \emph{decidable} in the Abella theorem prover,
  and make the Abella formalization available online\footnote{\url{https://github.com/JimmyZJX/Abella-subtyping-algorithm}}.

\item {\bf Information propagation using worklist judgments:} we
  employ worklists judgments in our algorithmic specification of polymorphic subtyping
  to propagate information across judgments.

%\item {\bf Abella formalization and discussion.} We have a complete
%  Abella formalization of all the results, as well as discussions on
%  the advantages/disadvantages of using Abella for such formalization.
%  \tom{Where do we discuss the disadvantages?}\jimmy{Texts in this subsection~\ref{subsection:discussion} discuss the disadvantages: no user-defined tactics, lack of packages}
\end{itemize}


\section{Introduction}



This paper presents the first fully mechanized formalization of the
metatheory for higher-ranked polymorphic type inference.
The system
that we formalize is the bidirectional type system by \citet{dunfield2013complete}.
We chose DK's type system because it is
quite elegant, well-documented and it comes with detailed manually
written proofs. Furthermore the system is adopted in practice by a few
real implementations of functional languages, including PureScript and
Unison. The DK type system has two variants: a declarative
and an algorithmic one. The two variants have been
\emph{manually} proved to be \emph{sound}, \emph{complete} and
\emph{decidable}.
We present a mechanical formalization in the Abella theorem prover~\cite{AbellaDesc} for
DK's declarative type system using a different algorithm. While our
initial goal was to formalize both DK's declarative and algorithmic
versions, we faced technical challenges with the latter, prompting us to find
an alternative formulation.

The first challenge that we faced were missing details as well as
a few incorrect proofs and lemmas in DK's formalization. While DK's
original formalization comes with very well written manual proofs,
there are still several details missing. These complicate the task of
writing a mechanically verified proof. Moreover some proofs and
lemmas are wrong and, in some cases, it is not clear to us how to fix them.


Despite the problems in DK's manual formalization,
we believe that these problems do not
invalidate their work and that their results are still true. In fact we have nothing but praise for their detailed
and clearly written metatheory and proofs, which provided invaluable
help to our own work.
We expect that for most non-trivial manual
proofs similar problems exist, so this should not be understood as a sign of sloppiness
on their part. Instead it should be an indicator that reinforces the arguments
for mechanical formalizations: manual formalizations are error-prone due to the multiple
tedious details involved in them.
There are several other examples of manual formalizations that were found to have
similar problems. For example, Klein et al.~\cite{KleinRunYourResearch}
mechanized formalizations
in Redex for nine ICFP 2009 papers and all were found to have mistakes.

Another challenge was variable binding. Type inference algorithms
typically do not rely simply on local environments but instead
propagate information across judgments. While local environments are
well-studied in mechanical formalizations, there is little work on how
to deal with the complex forms of binding employed by type inference algorithms
in theorem provers. To
keep track of variable scoping, DK's algorithmic version employs input
and output contexts to track information that is discovered through
type inference. However modeling output contexts in a theorem prover
is non-trivial.

Due to those two challenges, our work takes a different approach by refining and
extending the idea of \emph{worklist judgments}~\cite{itp2018},
proposed recently to mechanically formalize an algorithm for
\emph{polymorphic subtyping}~\cite{odersky1996putting}. A key innovation in our work is how
to adapt the idea of worklist judgments to
\emph{inference judgments}, which are not needed for polymorphic
subtyping, but are necessary for type-inference.  The idea is to use a \emph{continuation
passing style} to enable the transfer of inferred information across
judgments. A further refinement to the idea of worklist judgments is
the \emph{unification between ordered
  contexts~\cite{gundry2010type,dunfield2013complete} and worklists}.  This
enables precise scope tracking of free variables in
judgments. Furthermore it avoids the duplication of context
information across judgments in worklists that occurs in other
techniques~\cite{Reed2009,Abel2011higher}.
Despite the use of a different algorithm we prove the
same results as DK, although with significantly different proofs and
proof techniques. The calculus and its metatheory
have been fully formalized in the Abella theorem prover~\cite{AbellaDesc}.

In summary, the contributions of this paper are:

\begin{itemize}

\item {\bf A fully mechanized formalization of type inference with
  higher-ranked types:} Our work presents the first fully mechanized formalization
  for type inference of higher ranked types. The formalization is done in the
  Abella theorem prover~\cite{AbellaDesc} and it is available
  online at \url{https://github.com/JimmyZJX/TypingFormalization}.

\item {\bf A new algorithm for DK's type system:} Our work proposes a novel algorithm that implements
  DK's declarative bidirectional type system. We prove
  \emph{soundness}, \emph{completeness} and
  \emph{decidability}. 

\item {\bf Worklists with inference judgments:} One technical contribution is the
  support for inference judgments using worklists. The idea is to
  use a continuation passing style to enable the transfer of inferred information across
  judgments. 

\item {\bf Unification of worklists and contexts:} Another technical contribution is the unification
  between ordered contexts and worklists. This enables precise scope tracking
  of variables in judgments, and avoids the duplication of context information across
  judgments in worklists.

\begin{comment}
\jimmy{Notes @20190211 4 points of novalty:\\
1) Dealing with inference judgments and CPS-style chains\\
2) The form of the judgment itself with a single shared context\\
3) The way we deal with scope (which may follow from 2)\\
4) Immediate substitution (judgment list)
}
\end{comment}

\end{itemize}



