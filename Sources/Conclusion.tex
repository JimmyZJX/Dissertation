%%%%%%%%%%%%%%%%%%%%%%%%%%%%%%%%%%%%%%%%%%%%%%%%%%%%%%%%%%%%%%%%%%%%%%%%
\chapter{Conclusion and Future Work}
\label{chap:conclusion}
%%%%%%%%%%%%%%%%%%%%%%%%%%%%%%%%%%%%%%%%%%%%%%%%%%%%%%%%%%%%%%%%%%%%%%%%


\section{Conclusion}

In this thesis, we proposed new bidirectional type inference algorithms for
predicative higher-ranked implicit parametric polymorphic systems.
We showed how worklists ease the design of algorithms
and also mechanical formalizations.
By collecting all judgments in a single worklist,
the algorithm performs unification with a bird's-eye view of all the judgments,
therefore propagation between judgments is as simple as a global substitution.
Compared with classical HM unification procedure,
Our algorithm does not need seperated relations (or fixpoints) for unification.
In the formalization point of view,
eager substitutions are also easier to state and reason in a proof assistant.
Therefore, we obtained fully formalized properties for all our developments
relatively easily.
Overall, we developed the following systems and/or type inference algorithms:

\begin{itemize}
    \item We developed a worklist algorithm for OL's higher-ranked subtyping system.
        The algorithm operates on a worklist of subtyping judgments and
        a single context where variable declaration is shared across the worklist.
        Compared with DK's algorithm, our approach avoids the use of output contexts,
        which complicates the scoping of variables and is hard to formalize
        in a proof assistant.
        Eager substitutions that solves existential variables are directly applied to
        the worklist, therefore passing the partial information to the rest judgments.
        We proved \emph{soundness}, \emph{completeness} and \emph{decidability} of
        the algorithm in the Abella theorem prover.

    \item We developed a worklist algorithm for DK's higher-ranked bidirectional type system.
        In order to properly encode judgments that
        output types, such as the type inference judgment,
        continuation-passing-style \emph{judgment chain} is developed.
        Compared with the previous work that uses a single context,
        we further unifies the worklist with variable declarations.
        Such unification results in a much more accurate track of variable scopings,
        and the algorithm will garbage-collect variables
        as soon as they are not referred to any more.
        Unlike using output contexts as DK's algorithm,
        designing rules for worklist context is less likely to contain bugs
        in terms of variable scoping.
        Once again, based on eager substitutions, the algorithm is easy to formalize.
        We showed \emph{soundness}, \emph{completeness} and \emph{decidability}
        in the Abella theorem prover.

    \item We developed a backtracking-based algorithm for a higher-ranked
        bidirectional type system with object-oriented subtyping.
        With the introduction of the top and bottom types and relavent subtyping relations,
        meta-variable instantiations are no longer deterministic like the HM system.
        The backtracking-based algorithm preserves most characteristics of the previous
        worklist context, and it ``tries'' obvious solutions in parallel with detailed analysis
        as previous work.
        We proved that the algorithm is always \emph{sound},
        and subtyping is complete under the rank-1 restriction.
\end{itemize}


\section{Future Work}

In this section, we discuss several interesting possibilities to explore in the future.

\paragraph{Practical}

\paragraph{Elaboration}

\paragraph{Optimization}

\paragraph{Class Hierarchy of Nominal Subtyping}

\paragraph{Type Inference with Recursive Types}

\paragraph{Bounded Quantification / F-Bounded Quantification}

