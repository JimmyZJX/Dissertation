%%%%%%%%%%%%%%%%%%%%%%%%%%%%%%%%%%%%%%%%%%%%%%%%%%%%%%%%%%%%%%%%%%%%%%%%
\chapter{Conclusion and Future Work}
\label{chap:conclusion}
%%%%%%%%%%%%%%%%%%%%%%%%%%%%%%%%%%%%%%%%%%%%%%%%%%%%%%%%%%%%%%%%%%%%%%%%


\section{Conclusion}

In this paper we have shown how to mechanise an algorithmic subtyping relation
for higher-order polymorphism, together with its proofs of soundness,
completeness and decidability, in the Abella proof assistant. 
In ongoing work we are extending our mechanisation with a bidirectional type
inference algorithm. The main difficulty there is communicating the
instantiations of existential variables from the subtyping algorithm to the
type inference. To make this possible we are exploring a continuation passing
style formulation, which generalises the worklist approach.
Another possible extension is to have the algorithm return an explicit witness
for the subtyping as part of type-directed elaboration into System F.


In this paper we have shown how to mechanise an algorithmic subtyping relation
for higher-order polymorphism, together with its proofs of soundness,
completeness and decidability, in the Abella proof assistant. 
In ongoing work we are extending our mechanisation with a bidirectional type
inference algorithm. The main difficulty there is communicating the
instantiations of existential variables from the subtyping algorithm to the
type inference. To make this possible we are exploring a continuation passing
style formulation, which generalises the worklist approach.
Another possible extension is to have the algorithm return an explicit witness
for the subtyping as part of type-directed elaboration into System F.




\section{Future Work}


\paragraph{Elaboration}

\paragraph{Optimization}

\paragraph{OO class hierarchy}

\paragraph{Bounded Quantification / F-Bounded Quantification}

