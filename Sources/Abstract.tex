%
\noindent

Modern functional programming languages, such as Haskell or OCaml,
use sophisticated forms of type inference.
In the meantime, more and more object-oriented programming languages
implement advanced type inference algorithms,
which greatly reduces the number of redundant and uninteresting types
written by programmers,
including C++11, Java 10, and C\# 3.0.
While being an important topic in the Programming Languages research,
there is little work on the mechanization of
the metatheory of type inference in theorem provers.

In particular, we are unaware of any complete formalization of
the type inference algorithms that are the backbone of modern programming languages.
This thesis presents the \emph{first} full mechanical formalizations
of the metatheory for three higher-ranked polymorphic type inference algorithms.
Higher-ranked polymorphism is an advanced feature that enables more code reuse
and has numerous applications, which is already implemented in languages like Haskell.
All three systems are based on the bidirectional type system
by Dunfield and Krishnaswami (DK).
The DK type system has two variants, a declarative and an algorithmic one,
that have been manually proven sound, complete, and decidable.
While DK's original formalization comes with very well-written manual proofs,
there are several missing details and some incorrect proofs,
which motivates us to fully formalize the metatheory in proof assistants.
% which complicate the task of writing a mechanized proof.

Our first system focuses on the core problem in higher-ranked type inference algorithms
---the subtyping relation.
Our algorithm differs from those currently in the literature
by using a novel approach based on \emph{worklist judgments}.
Worklist judgments simplify the propagation of information
required by the unification process during subtyping.
Furthermore, they enable a simple formulation of the meta-theoretical
properties, which can be easily encoded in theorem provers.
We formally prove soundness, completeness, and decidability of the subtyping algorithm
w.r.t DK's declarative specification.

The second system extends the first one with a type system.
We present a mechanical formalization of
DK's declarative type system with a novel algorithmic system.
This system further unifies contexts with judgments,
which precisely captures the scope of variables and
simplifies the formalization of scoping in a theorem prover.
Besides, the use of continuation-passing-style judgments allows
simple formalization for inference judgments.
We formally prove soundness, completeness, and decidability of the type inference algorithm.
Despite the use of a different algorithm, we prove the same results as DK,
although with significantly different proofs and proof techniques.

The third system is based on the second one and extended with object-oriented subtyping.
In presence of object-oriented subtyping,
meta-variables usually have multiple different solutions.
Therefore we present a backtracking-based algorithm that
non-deterministically checks against each possibility.
We prove soundness w.r.t our specification,
and also completeness under the rank-1 restriction.

Since such type inference algorithms are quite subtle and have a complex metatheory,
mechanical formalizations are an important advance in type-inference research.
With machine-checked proofs, there is little chance that any logical derivation goes wrong.
In this thesis, all the properties we declare are fully formalized in the Abella theorem prover.



\vspace{1.5\baselineskip}

\noindent\makebox[\linewidth]{\rule{0.7\textwidth}{0.4pt}}

\begin{center}
    \emph{\textbf{An abstract of exactly 483 words}}
\end{center}

\newpage

\begin{flushright}
  \null\vspace{\stretch{1}}
  \textit{To my beloved parents and grandparents}
  \vspace{\stretch{2}}\null
\end{flushright}

