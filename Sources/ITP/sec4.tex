\section{Metatheory}\label{metatheory}

This section presents the three main meta-theoretical results that we have proved
in Abella. The first two are soundness and 
completeness of our algorithm with respect to Odersky and L\"aufer's declarative
subtyping. The third result is our algorithm's decidability. 

% The formalization of our higher-order polymorphism is developed in 
% Abella. Abella offers great flexibility on variable bindings and
% variable substitutions, which significantly decreases the difficulty
% of handling freshness and substitutions on the judgment list.

%-------------------------------------------------------------------------------
\subsection{Transfer to the Declarative System}

To state the correctness of the algorithmic subtyping rules,
Figure~\ref{fig:transfer} 
introduces two \textit{transfer} judgements to
relate the declarative and the algorithmic system. 
The first judgement, transfer of contexts $\Gamma \to \Psi$, removes
existential variables from the algorithmic context $\Gamma$ to obtain a
declarative context $\Psi$.
The second judgement, transfer of the judgement list $\Gamma \mid \exps \rightsquigarrow \exps'$,
replaces all occurrences of existential variables in $\exps$  by well-scoped mono-types.
Notice that this judgment is not decidable, i.e. a pair of
$\Gamma$ and $\exps$ may be related with multiple $\exps'$. However, if there
exists some substitution that transforms $\exps$ to $\exps'$, and each
subtyping judgment in $\exps'$ holds, we know that $\exps$ is potentially
satisfiable.

\begin{figure}[t]
\centering \framebox{$\Gamma \to \Psi$}
$$
\inferrule*[right=$\mathtt{{\to}\cdot}$]
	{~}{\cdot\to \cdot}
\qquad
\inferrule*[right=$\mathtt{{\to}var}$]
	{\Gamma\to \Psi}
	{\Gamma, a\to \Psi, a}
\qquad
\inferrule*[right=$\mathtt{{\to}exvar}$]
	{\Gamma\to \Psi}
	{\Gamma, \al\to \Psi}
\qquad
$$
\framebox{$\Gamma \mid \exps \rightsquigarrow \exps'$}
$$
\inferrule*[right=$\mathtt{{\rightsquigarrow}\cdot}$]
	{~}{\cdot \mid \exps\rightsquigarrow \exps}
\qquad
\inferrule*[right=$\mathtt{{\rightsquigarrow}var}$]
	{\Gamma \mid \exps \rightsquigarrow \exps'}
	{\Gamma, a \mid \exps \rightsquigarrow \exps'}
\qquad
\inferrule*[right=$\mathtt{{\rightsquigarrow}exvar}$]
	{\Gamma\to \Psi \quad \Psi\vdash \tau\quad
		\Gamma \mid [\tau/\al]\exps \rightsquigarrow \exps'}
	{\Gamma, \al \mid \exps \rightsquigarrow \exps'}
$$
\caption{Transfer Rules}
\label{fig:transfer}
\end{figure}

The following two lemmas generalize Rule~$\mathtt{{\rightsquigarrow}exvar}$ from substituting the first
existential variable to substituting any existential variable.

\begin{lemma}[Insert]
If $\Gamma\to\Psi$ and $\Psi\vdash \tau$ and $\Gamma, \Gamma_1 \mid [\tau/\al]\exps \rightsquigarrow \exps'$
, then $\Gamma, \al, \Gamma_1\mid \exps \rightsquigarrow \exps'$.
% \[
% \inferrule
%   {\Gamma\to\Psi \\ \Psi\vdash \tau \\ \Gamma, \Gamma_1 \mid [\tau/\al]\exps \rightsquigarrow \exps'}
%   {\Gamma, \al, \Gamma_1\mid \exps \rightsquigarrow \exps'}
% \]
\end{lemma}
\begin{lemma}[Extract]
	If $\Gamma, \al, \Gamma_1\mid \exps \rightsquigarrow \exps'$, then $\exists\tau$ s.t. $\Gamma\to\Psi, \Psi\vdash \tau$ and $\Gamma, \Gamma_1 \mid [\tau/\al]\exps \rightsquigarrow \exps'$.
\end{lemma}

In order to match the shape of algorithmic subtyping relation for the following proofs, we define a relation $\Psi\vdash \exps$ for the declarative system, meaning that all the declarative judgments hold under context $\Psi$.
\begin{definition}[Declarative Subtyping Worklist]
	$$\Psi\vdash\exps := \forall (A\le B)\in \exps, \Psi\vdash A\le B$$
\end{definition}

%The transfer rules play a role of an axiom which specifies the relation between our algorithmic system and declarative system.

%-------------------------------------------------------------------------------
\subsection{Soundness}
Our algorithm is sound with respect to the
declarative specification. For any derivation of a list of algorithmic
judgments $\Gamma\vdash\exps$, we can find a valid transfer
$\Gamma\mid\exps\rightsquigarrow\exps'$ such that all judgments in $\exps'$ hold
in $\Psi$, with $\Gamma \to \Psi$.

\begin{theorem}[Soundness]
	If $\Gamma\vdash \exps$ and $\Gamma \to \Psi$, then there exists $\exps'$, s.t. $\Gamma \mid \exps\rightsquigarrow \exps'$ and $\Psi\vdash \exps'$.
\end{theorem}
The proof proceeds by induction on the derivation
of $\Gamma\vdash\exps$, finished off by appropriate applications of the insertion and
extraction lemmas.

%\tom{Say something about the corollary, e.g., why it's important.}
%\begin{corollary}[Soundness\_decl]
%	If $\Gamma = \Psi, A = A', B = B'$, $\Gamma\vdash A\le B : \cdot$, then $\Psi\vdash A'\le B'$.
%\end{corollary}

%-------------------------------------------------------------------------------
\subsection{Completeness}

Completeness of the algorithm means that any declarative derivation has an
algorithmic counterpart. 
% More specifically, for any derivation of declarative
% judgments $\Psi\vdash\exps'$ and any extended environment $\Gamma$, any
% algorithmic judgments that satisfies the transfer relation
% $\Gamma\mid\exps\rightsquigarrow\exps'$, we can successfully derive
% $\Gamma\vdash\exps$.
\begin{theorem}[Completeness]
	If $\Psi\vdash \exps'$ and $\Gamma \to \Psi$ and $\Gamma \mid \exps\rightsquigarrow \exps'$, then $\Gamma\vdash \exps$.
\end{theorem}

The proof proceeds by induction on the derivation of $\Psi\vdash\exps'$. As the
declarative system does not involve information propagation across judgments,
the induction can focus on the subtyping derivation of the first judgment
without affecting other judgments.
The difficult cases correspond to the $\mathtt{{\le_a}instL}$ and
$\mathtt{{\le_a}instR}$ rules.  When the proof by induction on $\Psi\vdash
\exps'$ reaches the $\mathtt{{\le}{\to}}$ case, the first declarative
judgment has a shape like $A_1\to A_2 \le B_1\to B_2$. 
One of the possibile cases for the first corresponding algorithmic judgement
is $\al\le A\to B$. However, the case analysis does not indicate
that $\al$ is fresh in $A$ and $B$. Thus we cannot apply
Rule~$\mathtt{{\le_a}instL}$ and make use of the induction hypothesis.
The following lemma helps us out in those cases:
it rules out subtypings with infinite types as solutions (e.g. $\al \le 1
\to \al$) and guarantees that $\al$ is free in $A$ and $B$.
\begin{lemma}[Prune Transfer for Instantiation]
	If $\Psi \vdash \jcons{A_1\to A_2 \le B_1\to B_2}{\exps'}$ and $\Gamma\to \Psi$ and $\Gamma\mid (\jcons{\al \le A\to B}{\exps}) \rightsquigarrow
	(\jcons{A_1\to A_2 \le B_1\to B_2}{\exps'})$
	% or $\Gamma\mid A\to B \le \al : \exps \rightsquigarrow A_1\to A_2 \le B_1\to B_2 : \exps'$
	, then $\al \notin FV(A)\cup FV(B)$. 
\end{lemma}
A similar lemma holds for the symmetric case $(\jcons{A \to B \le \al}{\Omega})$.
%\tom{Say something about the corollary, e.g., why it's important.}
%\begin{corollary}[Completeness\_decl]
%	If $\Gamma = \Psi, A = A', B = B'$, $\Psi\vdash A'\le B'$, then $\Gamma\vdash A\le B$.
%\end{corollary}

%-------------------------------------------------------------------------------
\subsection{Decidability}

The third key result for our algorithm is decidability.
\begin{theorem}[Decidability]
	Given any well-formed judgment list $\exps$ under $\Gamma$, it is decidable whether $\Gamma\vdash \exps$ or not.
\end{theorem}

\noindent We have proven this theorem by means of a lexicographic group of induction
measurements $\left\langle|\Omega|_\forall,|\Gamma|_{\al}, |\Omega|_\to\right\rangle$ on the worklist $\Omega$
and algorithmic context $\Gamma$. 
The worklist measures $|\cdot|_\forall$ and $|\cdot|_\to$ count the
number of universal quantifiers and function types respectively.
\begin{definition}[Worklist Measures]
\[
\begin{array}{rcl@{\hspace{-10mm}}rclrcl} 
|1|_\forall = |a|_\forall = |\al|_\forall &=& 0      &|1|_\to = |a|_\to = |\al|_\to &=& 0        \\
|A\to B|_\forall &=& |A|_\forall + |B|_\forall  &|A\to B|_\to &=& |A|_\to + |B|_\to + 1\\
|\forall x. A|_\forall &=& |A|_\forall + 1              &|\forall x. A|_\to &=& |A|_\to                        \\
|\exps|_\forall &=& \sum_{A \le B\in \exps} |A|_\forall + |B|_\forall       &|\exps|_\to &=& \sum_{A \le B \in \exps} |A|_\to + |B|_\to
\end{array}
\]
\end{definition}

\noindent The context measure $|\cdot|_{\al}$ counts the number of unsolved existential variables.
\begin{definition}[Context Measure]
$$|\cdot|_{\al} = 0\qquad
|\Gamma,a|_{\al} = |\Gamma|_{\al}\qquad
|\Gamma,\al|_{\al} = |\Gamma|_{\al} + 1$$
\end{definition}

It is not difficult to see that all but two of the algorithm's rules decrease
one of the three measures. The two exceptions are the Rules $\mathtt{{\le_a}instL}$ and $\mathtt{{\le_a}instR}$; both increment the number of existential
variables and the number of function types without affecting the number of
universal quantifiers.
To handle these rules, we handle a special class of judgements, which
we call \emph{instantiation judgements} $\exps_i$, separately. They 
take the form:
\begin{definition}[$\exps_i$]
$$\exps_i := \cdot \mid \jcons{\al\le A}{\exps_i'} \mid \jcons{A\le \al}{\exps_i'}
\quad\text{where } \al\notin FV(A)\cup FV(\exps_i')$$
\end{definition}
These instantiation judgements are these ones consumed and
produced by the Rules $\mathtt{{\le_a}instL}$ and $\mathtt{{\le_a}instR}$.
The following lemma handles their decidability.
\begin{lemma}[Instantiation Decidability]
	For any context $\Gamma$ and judgment list $\exps_i, \exps$, it is decidable whether $\Gamma\vdash \exps_i,\exps$ if both of the conditions hold
\begin{enumerate}[1)]
	\item $\forall \Gamma', \exps'$ s.t. $|\exps'|_\forall < |\exps_i,\exps|_\forall$, it is decidable whether $\Gamma'\vdash \exps'$.
	\item $\forall \Gamma', \exps'$ s.t. $|\exps'|_\forall = |\exps_i,\exps|_\forall$ and $|\Gamma'|_{\al} = |\Gamma|_{\al} - |\exps_i|$, it is decidable whether $\Gamma'\vdash \exps'$.
\end{enumerate}
\label{lemma:inst_decidable}
\end{lemma}
In other words, for any instantiation judgment prefix $\exps_i$, the algorithm
either reduces the number of $\forall$'s or solves one existential variable per
instantiation judgment. The proof of this lemma is by induction on the measure
$2*|\exps_i|_{\to} + |\exps_i|$ of the instantiation judgment list.

In summary, the decidability theorem can be shown through a lexicographic group
of induction measurements $\left\langle|\Omega|_\forall, |\Omega|_{\al},
|\Omega|_\to\right\rangle$. The critical case is that, whenever we
encounter an instantiation judgment at the front of the worklist, we refer to
Lemma~\ref{lemma:inst_decidable}, which reduces the number of unsolved
variables by consuming that instantiation judgment, or reduces the number of
$\forall$-quantifiers. Other cases are relatively straightforward.
