\section{The Choice of Abella}

We have chosen the Abella (v2.0.5) proof assistant~\citep{AbellaDesc} to
develop our formalization.
Our development is only based on the reasoning logic of Abella, and does not make use of its specification logic.
%Although Abella takes a two-level logic approach,
%where the specification logic can be expressed separately from the
%reasoning logic, we only make use of its reasoning logic, due to the
%difficulty of expressing our algorithmic rules with only the
%specification.
Abella is particularly helpful due to its built-in support for variable bindings, and
its $\lambda$-tree syntax~\citep{miller2000abstract} is a form of HOAS,
which helps with the encoding and reasoning about substitutions.  For
instance, the type $\forall x. x \to a$ is encoded as \abellae{all (x\  arrow x a)}, where \abellae{x\ arrow x a} is a lambda abstraction in
Abella. An opening $[b/x](x\to a)$ is encoded as an application
\abellae{(x\ arrow x a) b}, which can be simplified(evaluated) to
\abellae{arrow b a}.
Name supply and freshness conditions are controlled by the
$\nabla$-quantifier.  The expression \abellae{nabla x, F} means that
\abellae{x} is a unique variable in \abellae{F}, i.e. it is different
from any other names occurring elsewhere.  Such variables are called
nominal constants.  They can be of any type, in other words, every
type may contain an unlimited number of such atomic nominal constants.

%\subsection{Encoding of Declarative System}

\paragraph{Encoding of the Declarative System}
As a concrete example, our declarative context and well-formedness rules are encoded as follows.
\begin{abella}
	Kind ty     type.
	Type i      ty.                % the unit type
	Type all    (ty -> ty) -> ty.    % forall-quantifier
	Type arrow  ty -> ty -> ty.      % function type
	Type bound  ty -> o.            % variable collection in contexts
	
	Define env : olist -> prop by
		env nil;
		nabla x, env (bound x :: E) := env E.
	
	Define wft : olist -> ty -> prop by
		wft E i;
		nabla x, wft (E x) x := nabla x, member (bound x) (E x);
		wft E (arrow A B) := wft E A /\ wft E B;
		wft E (all A) := nabla x, wft (bound x :: E) (A x).
\end{abella}

We use the type \abellae{olist} just as normal list of \abellae{o} with two constructors, namely \abellae{nil : olist} and \abellae{(::) : o -> olist -> olist}, where \abellae{o} purely means ``the element type of \abellae{olist}''. The \abellae{member : o -> olist -> prop} relation is also pre-defined.
The second case of the relation \abellae{wft} states Rule $\mathtt{wf_d var}$.
The encoding \abellae{(E x)} basically means that the context \emph{may} contain \abellae{x}.
If we write \abellae{(E x)} as \abellae{E}, then the context should not contain \abellae{x}, and both \abellae{wft E x} and \abellae{member (bound x) E} make no sense.
Instead, we treat \abellae{E : ty -> olist} as an \emph{abstract structure} of a context, such as \abellae{x\ bound x :: bound a :: nil}.
For the fourth case of the relation \abellae{wft}, the type $\forall x. A$ in our target language is expressed as \abellae{(all A)}, and its opening $A$, \abellae{(A x)}.

\paragraph{Encoding of the Algorithmic System}
In terms of the algorithmic system, notably, Abella handles the
$\mathtt{{\le_a}instL}$ and $\mathtt{{\le_a}instR}$ rules in a nice way:
\begin{abella}
	% sub_alg_list : enva -> [subty_judgment] -> prop
	Define subal : olist -> olist -> prop by
		subal E nil;
		subal E (subt i i :: Exp) := subal E Exp;
		% some cases omitted ...
		% <: instL
		nabla x, subal (E x) (subt x (arrow A B) :: Exp x) :=
			exists E1 E2 F, nabla x y z, append E1 (exvar x :: E2) (E x) /\
				append E1 (exvar y :: exvar z :: E2) (F y z) /\
				subal (F y z) (subt (arrow y z) (arrow A B) :: Exp (arrow y z));
		% <: instR is symmetric to <: instL, omitted here
		% other cases omitted ...
\end{abella}
Thanks to the way Abella deals with nominal constants, the pattern 
\abellae{subt x (arrow A B)} implicitly states that $x\notin FV(A) \land x\notin FV(B)$.
If the condition were not required, we would have encoded the pattern as 
\abellae{subt x (arrow (A x) (B x))} instead.

\subsection{Statistics and Discussion}\label{subsection:discussion}
\begin{figure}[t]
	\centering\begin{tabular}{|c|c|c|l|}\hline
		File(s) & SLOC & \# of Theorems & Description\\\hline
		olist.thm, nat.thm      &  303 & 55  & Basic data structures\\\hline
		higher.thm, order.thm   &  164 & 15  & Declarative system\\\hline
		higher\_alg.thm         &  618 & 44  & Algorithmic system\\\hline
		trans.thm               &  411 & 46  & Transfer\\\hline
		sound.thm               &  166 & 2   & Soundness theorem\\\hline
		depth.thm               &  143 & 12  & Definition of depth\\\hline
		complete.thm            &  626 & 28  & Lemmas and Completeness theorem\\\hline
		decidable.thm           & 1077 & 53  & Lemmas and Decidability theorem\\\hline
		Total                   & 3627 & 267 & (33 definitions in total)\\\hline
	\end{tabular}
	\caption{Statistics for the proof scripts}\label{fig:SLOC}
\end{figure}

Some basic statistics on our proof script are shown in Figure~\ref{fig:SLOC}.
The proof consists of 3627 lines of code with a total of 33 definitions and 267 theorems.
We have to mention that Abella provides few built-in tactics and does not support user-defined ones, and we would reduce significant lines of code if Abella provided more handy tactics.
Moreover, the definition of natural numbers, the plus operation and less-than relation are defined within our proof due to Abella's lack of packages.
However, the way Abella deals with name bindings
is very helpful for type system formalizations and substitution-intensive formalizations, such as this one.
