\section{Introduction}



%
% Secondly, many older type-inference algorithms typically use
% sets to collect constraints or substitutions for unification; these are hard to
% handle in the purely inductive style favored by theorem provers.


In summary the contributions of this paper are:

\begin{itemize}
\item {\bf A mechanical formalization of a polymorphic subtyping
    algorithm.} We show that the algorithm is \emph{sound},
  \emph{complete} and \emph{decidable} in the Abella theorem prover,
  and make the Abella formalization available online\footnote{\url{https://github.com/JimmyZJX/Abella-subtyping-algorithm}}.

\item {\bf Information propagation using worklist judgments:} we
  employ worklists judgments in our algorithmic specification of polymorphic subtyping
  to propagate information across judgments.

%\item {\bf Abella formalization and discussion.} We have a complete
%  Abella formalization of all the results, as well as discussions on
%  the advantages/disadvantages of using Abella for such formalization.
%  \tom{Where do we discuss the disadvantages?}\jimmy{Texts in this subsection~\ref{subsection:discussion} discuss the disadvantages: no user-defined tactics, lack of packages}
\end{itemize}
