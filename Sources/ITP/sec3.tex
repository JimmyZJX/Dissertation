\section{A Worklist Algorithm for Polymorphic Subtyping}\label{algorithmic_subtyping}

This section presents our algorithm for polymorphic
subtyping. A novel aspect of our algorithm is the use of worklist
judgments: a form of judgement that facilitates the propagation 
of information. 


%-------------------------------------------------------------------------------
\subsection{Syntax and Well-Formedness of the Algorithmic System}
Figure~\ref{fig:alg:syntax} shows the
syntax and the well-formedness judgement.  

% - - - - - - - - - - - - - - - - - - - - - - - - - - - - - - - - - - - - - - - - 
\paragraph{Existential Variables}
In order to solve the unknown types $\tau$, the algorithmic system extends the
declarative syntax of types with \emph{existential variables} $\al$.  They
behave like unification variables, but are not globally defined. Instead, the
ordered \emph{algorithmic context}, inspired by \citet{dunfield2013complete},
defines their scope. Thus 
the type $\tau$ represented by the corresponding existential variable is
always bound in the corresponding declarative context $\Psi$.

%- - - - - - - - - - - - - - - - - - - - - - - - - - - - - - - - - - - - - - - - 
\paragraph{Worklist Judgements} The form of our algorithmic judgements is
non-standard. 
%tracks the (partial) solutions of existential variables
%in the algorithmic context; they denote a delayed substitution that is
%incrementally applied to outstanding work as it is encoutered.  
%Instead of reifying the substitution, 
Our algorithm keeps track of an explicit list of
outstanding work: the list $\Omega$ of (reified) \emph{algorithmic judgements} 
of the form $A \leq B$,
to which a substitution can be applied once and for all to propagate the solution
of an existential variable. 

\begin{figure}[t]
\[
\begin{array}{l@{\qquad}lcl}
\text{Type variables} & a, b\\
\text{Existential variables} & \al, \bt\\[3mm]
\text{Algorithmic types} &A, B, C &::=&\quad 1 \mid a \mid \al \mid \forall a. A \mid A\to B\\
\text{Algorithmic context}&\Gamma &::=&\quad \cdot \mid \Gamma, a \mid \Gamma, \al\\
\text{Algorithmic judgments}&\exps &::=&\quad \cdot \mid \jcons{A \le B}{\exps}
\end{array}
\]
\centering \framebox{$\Gamma \vdash A$}
\begin{gather*}
\inferrule*[right=$\mathtt{{wf_a}unit}$]
  {~}
  {\Gamma \vdash 1}
\qquad
\inferrule*[right=$\mathtt{{wf_a}var}$]
  {a\in\Gamma}
  {\Gamma\vdash a}
\qquad
\inferrule*[right=$\mathtt{{wf_a}exvar}$]
  {\al\in\Gamma}
  {\Gamma\vdash \al} \\
\inferrule*[right=$\mathtt{{wf_a}{\to}}$]
  {\Gamma\vdash A \\ \Gamma\vdash B}
  {\Gamma\vdash A\to B}
\qquad
\inferrule*[right=$\mathtt{{wf_a}\forall}$]
  {\Gamma, a\vdash A}
  {\Gamma\vdash \forall a. A}
\end{gather*}
\caption{Syntax and Well-Formedness Judgement for the Algorithmic System.}\label{fig:alg:syntax}
\end{figure}

\begin{comment}
\begin{figure}[t]
\centering \framebox{$\Gamma \vdash A$}
\begin{gather*}
\inferrule*[right=$\mathtt{{wf_a}unit}$]
  {~}
  {\Gamma \vdash 1}
\qquad
\inferrule*[right=$\mathtt{{wf_a}var}$]
  {a\in\Gamma}
  {\Gamma\vdash a}
\qquad
\inferrule*[right=$\mathtt{{wf_a}exvar}$]
  {\al\in\Gamma}
  {\Gamma\vdash \al} \\
\inferrule*[right=$\mathtt{{wf_a}{\to}}$]
  {\Gamma\vdash A \\ \Gamma\vdash B}
  {\Gamma\vdash A\to B}
\qquad
\inferrule*[right=$\mathtt{{wf_a}\forall}$]
  {\Gamma, a\vdash A}
  {\Gamma\vdash \forall a. A}
\end{gather*}
\caption{Well-Formedness Judgement of the Algorithmic System}\label{fig:alg:wf}
\end{figure}
\end{comment}

\paragraph{Hole Notation}
To facilitate context manipulation, we use the syntax $\Gamma[\Gamma_M]$ to
denote a context of the form $\Gamma_L, \Gamma_M, \Gamma_R$ where $\Gamma$ is
the context $\Gamma_L, \bullet, \Gamma_R$ with a hole ($\bullet$).
Hole notations with the same name implicitly share the same $\Gamma_L$ and $\Gamma_R$. A multi-hole notation like $\Gamma[\al][\bt]$ means $\Gamma_1,\al,\Gamma_2,\bt,\Gamma_3$.

%-------------------------------------------------------------------------------
\subsection{Algorithmic Subtyping}

The algorithmic subtyping judgement, defined in Figure~\ref{fig:alg}, has the form $\Gamma\vdash\exps$, where
$\exps$ collects multiple subtyping judgments $A\le B$. 
% \bruno{Text comparing to Dunfield. Maybe mention in RW instead?:
% In contrast to the
% original formulation---which features 3 interdepenent judgements---our
% algorithmic rules are all part of the same judgement. This is better for
% formalization in proof assistants and avoids mutual dependencies. 
% }
The algorithm treats $\exps$ as a worklist. In every step
it takes one task from the worklist for processing, possibly
pushes some new tasks on the worklist, and repeats this
process until the list is empty. This last and single base case
is handled by Rule~$\mathtt{a\_nil}$.
The remaining rules all deal with the first task in the worklist.
Logically we can discern 3 groups of rules.

\begin{figure}[t]
\centering \framebox{$\Gamma \vdash \exps$}
\begin{gather*}
\inferrule*[right=$\mathtt{{\le_a}nil}$]
  {~}
  {\Gamma \vdash \cdot} 
\\ \\
\inferrule*[right=$\mathtt{{\le_a}unit}$]
  {\Gamma \vdash \exps}
  {\Gamma \vdash \jcons{1\le 1}{\exps}}
\qquad
\inferrule*[right=$\mathtt{{\le_a}var}$]
  {a\in\Gamma \\ \Gamma \vdash \exps}
  {\Gamma \vdash \jcons{a\le a}{\exps}}
\qquad
\inferrule*[right=$\mathtt{{\le_a}exvar}$]
  {\al\in\Gamma \\ \Gamma \vdash \exps}
  {\Gamma \vdash \jcons{\al\le \al}{\exps}}
\\
\inferrule*[right=$\mathtt{{\le_a}{\to}}$]
  {\Gamma \vdash \jcons{B_1\le A_1}{\jcons{A_2\le B_2}{\exps}}}
  {\Gamma \vdash \jcons{A_1\to A_2\le B_1\to B_2}{\exps}}
\\
\inferrule*[right=$\mathtt{{\le_a}\forall L}$]
  {\al \text{ fresh} \\ \Gamma,\al \vdash \jcons{[\al/a]A\le B}{\exps}}
  {\Gamma\vdash \jcons{\forall a. A\le B}{\exps}}
\qquad
\inferrule*[right=$\mathtt{{\le_a}\forall R}$]
  {b \text{ fresh} \\ \Gamma,b \vdash \jcons{A\le B}{\exps}}
  {\Gamma\vdash \jcons{A\le \forall b. B}{\exps}}
\\
\\
\inferrule*[right=$\mathtt{{\le_a}instL}$]
  {\al\notin \mathit{FV}(A)\cup FV(B)\quad
  	\Gamma[\al[1], \al[2]]\vdash \jcons{\al[1]\to \al[2]\le A\to B}{ [\al[1]\to \al[2]/\al]\exps}}
  {\Gamma[\al] \vdash \jcons{\al\le A\to B}{\exps}}
\\
\inferrule*[right=$\mathtt{{\le_a}instR}$]
  {\al\notin FV(A)\cup FV(B)\quad
	\Gamma[\al[1], \al[2]]\vdash \jcons{A\to B\le \al[1]\to \al[2]}{ [\al[1]\to \al[2]/\al]\exps}}
  {\Gamma[\al] \vdash \jcons{A\to B\le \al}{\exps}}
\\
\\
\inferrule*[right=$\mathtt{{\le_a}solve\_ex}$]
  {\Gamma[\al][]\vdash [\al/\bt]\exps}
  {\Gamma[\al][\bt]\vdash \jcons{\al\le \bt}{\exps}}
\qquad
\inferrule*[right=$\mathtt{{\le_a}solve\_ex'}$]
  {\Gamma[\al][]\vdash [\al/\bt]\exps}
  {\Gamma[\al][\bt]\vdash \jcons{\bt\le \al}{\exps}}
\\
\inferrule*[right=$\mathtt{{\le_a}solve\_var}$]
  {\Gamma[a][]\vdash [a/\bt]\exps}
  {\Gamma[a][\bt]\vdash \jcons{a\le \bt}{\exps}}
\qquad
\inferrule*[right=$\mathtt{{\le_a}solve\_var'}$]
  {\Gamma[a][]\vdash [a/\bt]\exps}
  {\Gamma[a][\bt]\vdash \jcons{\bt\le a}{\exps}}
\\
\inferrule*[right=$\mathtt{{\le_a}solve\_unit}$]
  {\Gamma[]\vdash [1/\al]\exps}
  {\Gamma[\al]\vdash \jcons{\al\le 1}{\exps}}
\qquad
\inferrule*[right=$\mathtt{{\le_a}solve\_unit'}$]
  {\Gamma[]\vdash [1/\al]\exps}
  {\Gamma[\al]\vdash \jcons{1\le \al}{\exps}}
\end{gather*}
\caption{Algorithmic Subtyping}
\label{fig:alg}
\end{figure}

Firstly, we have five rules that are similar to those in the declarative
system, mostly just adapted to the worklist style. For instance, Rule
$\mathtt{{\le_a}{\to}}$ consumes one judgment and pushes two to the
worklist.  A notable difference with the declarative Rule $\mathtt{{\le}\forall
L}$ is that Rule $\mathtt{{\le_a}\forall L}$ requires no guessing of a type $\tau$ to instantiate
the polymorphic type $\forall a. A$, but instead
introduces an existential variable $\al$ to the context and to $A$. In
accordance with the declarative system, where 
the monotype $\tau$ should be bound in the context $\Psi$, here $\al$ should only
be solved to a monotype bound in $\Gamma$. More generally, for any algorithmic context $\Gamma[\al]$, the algorithmic variable $\al$ 
can only be solved to a monotype that is well-formed with respect to $\Gamma_L$.

Secondly, Rules $\mathtt{{\le_a}instL}$ and $\mathtt{{\le_a}instR}$ partially
instantiate existential types $\al$, to function types. The domain and range
of the new function type are undetermined: they are set to two
fresh existential variables $\al[1]$ and $\al[2]$. To make sure that
$\al[1] \to \al[2]$ has the same scope as $\al$, the new variables
$\al[1]$ and $\al[2]$ are inserted in the same position in the context
where the old variable $\al$ was. To propagate the instantiation to the remainder
of the worklist, $\al$ is substituted for $\al[1] \to \al[2]$ in $\Omega$.
The \emph{occurs-check} side-condition is necessary to prevent a diverging
infinite instantiation. For example
$1 \to \al \le \al$ would diverge with no such check.

Thirdly, in the remaining six rules an existential variable can be immediately
solved. Each of the six similar rules removes an existential variable from the
context,  performs a substitution on the remainder of the worklist and
continues.

The algorithm on judgment list is designed to share the context across all judgments.
However, the declarative system does not share a single context in its derivation.
This gap is filled by strengthening and weakening lemmas of both systems,
where most of them are straightforward to prove,
except for the strengthening lemma of the declarative system, which is a little trickier.

\begin{figure}[t]
\begin{tabular}{cc}
$\begin{aligned}
	\inferrule*[Right=$\mathtt{{\le_a}\forall L}$]
	{\inferrule*[Right=$\mathtt{{\le_a}{\to}}$]
		{\inferrule*[Right=$\mathtt{{\le_a}\forall L}$]
			{\inferrule*[Right=$\mathtt{{\le_a}instR}$]
				{\inferrule*[Right=$\mathtt{{\le_a}{\to}}$]
					{\inferrule*[Right=$\mathtt{{\le_a}solve\_ex}$]
						{\inferrule*[Right=$\mathtt{{\le_a}solve\_ex}$]
							{\inferrule*[Right=$\mathtt{{\le_a}unit}$]
								{\inferrule*[Right=$\mathtt{a\_nil}$]
									{~}
									{\al[1] \vdash \cdot}
								}
								{\al[1] \vdash \jcons{1 \le 1}{\cdot}}
							}
							{\al[1], \al[2] \vdash \jcons{\al[1] \le \al[2]}{\jcons{1 \le 1}{\cdot}}}
						}
						{\al[1], \al[2], \bt \vdash \jcons{\al[1] \le \bt}{\jcons{\bt \le \al[2]}{\jcons{1 \le 1}{\cdot}}}}
					}
					{\al[1], \al[2], \bt \vdash \jcons{\bt \to \bt \le \al[1] \to \al[2]}{\jcons{1 \le 1}{\cdot}}}
				}
				{\al, \bt \vdash \jcons{\bt \to \bt \le \al}{\jcons{1 \le 1}{\cdot}}}
			}
			{\al \vdash \jcons{\forall a.\ a \to a \le \al}{\jcons{1 \le 1}{\cdot}}}
		}
		{\al \vdash \jcons{\al \to 1 \le (\forall a.\ a \to a) \to 1}{\cdot}}
	}
	{\cdot \vdash \jcons{\forall a.\ a\to 1\le (\forall a.\ a\to a)\to 1}{\cdot }}
	\end{aligned}\qquad\quad$
	%\label{fig:alg_sample}
&
	$\begin{aligned}
	\inferrule*[Right=$\mathtt{{\le_a}\forall L}$]
	{\inferrule*[Right=$\mathtt{{\le_a}{\to}}$]
		{\inferrule*[Right=$\mathtt{{\le_a}unit}$]
			{\inferrule*[Right=$\mathtt{{\le_a}\forall R}$]
				{\inferrule*[Right=$\mathtt{?}$]
					{stuck
					}
					{\al, b \vdash \jcons{\al \le b}{\cdot}}
				}
				{\al \vdash {\jcons{\al \le \forall b.\ b}{\cdot}}}
			}
			{\al \vdash \jcons{1\le 1}{\jcons{\al \le \forall b.\ b}{\cdot}}}
		}
		{\al \vdash \jcons{1 \to \al \le 1\to \forall b.\ b}{\cdot}}
	}
	{\cdot\vdash \jcons{\forall a.\ 1\to a \le 1\to \forall b.\ b}{\cdot }}
	\end{aligned}$
	%\label{fig:alg_sample2}
\end{tabular}
\caption{Successful and Failing Derivations for the Algorithmic Subtyping Relation}
\label{fig:alg_samples}
\end{figure}

\paragraph{Example}
We illustrate the subtyping rules through a sample derivation in the left of
Figure~\ref{fig:alg_samples}, which shows that that $\forall a.\ a\to 1\le (\forall a.\ a\to
a)\to 1$. Thus the derivation starts with an empty context and a
judgment list with only one element.


% \begin{center}
% 	\begin{tabular}{|c|c|c|c|}\hline
% 		\# & Context & Worklist & Rule\\\hline
% 		1&$\cdot$ & $\forall x.\ x\to 1\le (\forall x.\ x\to x)\to 1$ & $\mathtt{{\le_a}\forall L}$\\\hline
% 		2&$\al$ & $\al\to 1\le (\forall x.\ x\to x)\to 1$ & $\mathtt{{\le_a}{\to}}$\\\hline
% 		3&$\al$ & $\forall x.\ x\to x\le \al : 1\le 1$ & $\mathtt{{\le_a}\forall L}$\\\hline
% 		4&$\al,\bt$ & $\bt\to \bt\le \al : 1\le 1$ & $\mathtt{instR}$\\\hline
% 		5&$\al[1],\al[2], \bt$ & $\bt\to \bt\le \al[1]\to \al[2] : 1\le 1$ & $\mathtt{{\le_a}{\to}}$\\\hline
% 		6&$\al[1],\al[2], \bt$ & $\al[1]\le \bt : \bt\le \al[2] : 1\le 1$ & $\mathtt{{\le_a}solve\_ex (\bt\leftarrow \al[1])}$\\\hline
% 		7&$\al[1],\al[2]$ & $\al[1]\le \al[2] : 1\le 1$ & $\mathtt{{\le_a}solve\_ex (\al[2]\leftarrow \al[1])}$\\\hline
% 		8&$\al[1]$ & $1\le 1$ & $\mathtt{{\le_a}unit}$\\\hline
% 	\end{tabular}
% \end{center}

In step 1, we have only one judgment, and that one has a top-level $\forall$ on
the left hand side. So the only choice is rule $\mathtt{{\le_a}\forall L}$, which
opens the universally quantified type with an unknown existential variable
$\al$. Variable $\al$ will be solved later to some monotype that is well-formed
within the context before $\al$. That is, the empty context $\cdot$ in this
case.
In step 2, rule $\mathtt{{\le_a}{\to}}$ is applied to the worklist,
splitting the first judgment into two.
Step 3 is similar to step 1, where the left-hand-side $\forall$ of the first
judgment is opened according to rule $\mathtt{{\le_a}\forall L}$ with a fresh
existential variable.
In step 4, the first judgment has an arrow on the left hand side, but the
right-hand-side type is an existential variable. It is obvious
that $\al$ should be solved to a monotype of the form
$\sigma \to \tau$. Rule $\mathtt{instR}$ implements this, but avoids
guessing $\sigma$ and $\tau$ by ``splitting'' $\al$ into two existential
variables, $\al[1]$ and $\al[2]$, which will be solved to some $\sigma$ and
$\tau$ later.
Step 5 applies Rule $\mathtt{{\le_a}{\to}}$ again. Notice that after the
split, $\bt$ appears in two judgments. When the first $\bt$ is solved
during any step of derivation, the next $\bt$ will be substituted by that
solution.  This propagation mechanism ensures the consistent solution of the
variables, while keeping the context as simple as possible.
Steps 6 and 7 solve existential variables. The existential
variable that is right-most in the context is always solved in terms of the other. Therefore in step 6,
$\bt$ is solved in terms of $\al[1]$, and in step 7, $\al[2]$ is solved in terms of $\al[1]$.
Additionally, in step 6, when $\bt$ is solved, the substitution $[\al[1] /
\bt]$ is propagated to the rest of the judgment list, and thus the second
judgment becomes $\al[1]\le\al[2]$.
Steps 8 and 9 trivially finish the derivation. Notice that $\al[1]$ is not
instantiated at the end. This means that any well-scoped instantiation is fine.

\paragraph{A Failing Derivation} We illustrate the role of ordered contexts through another example: $\forall a.\ 1\to a \le 1\to \forall b.\ b$. From the declarative perspective, $a$ should be instantiated to some $\tau$ first, then $b$ is introduced to the context, so that $b\notin FV(\tau)$. As a result, we cannot find $\tau$ such that $\tau \le b$. The right of Figure~\ref{fig:alg_samples} shows the algorithmic derivation, which also fails due to the scoping---$\al$ is introduced earlier than $b$, thus it cannot be solved to $b$.





