
\newcommand\mynote[3]{\textcolor{#2}{#1: #3}}
\newcommand\bruno[1]{\mynote{Bruno}{red}{#1}}
\newcommand\jimmy[1]{\mynote{Jimmy}{blue}{#1}}
\newcommand\tom[1]{\mynote{Tom}{magenta}{#1}}

\newcommand\wh{\widehat}

\newcommand\jg{\omega}                                 % judgment
\newcommand\toto{\rightrightarrows}
\newcommand\To{\Rightarrow}                            % =>
\newcommand\Lto{\Leftarrow}                            % <=
\newcommand\TTo{\mathrel{\mathrlap{\To}\phantom{~}\To}}  % =>>
\newcommand\sto{\rightsquigarrow}
\newcommand\tto{\rightarrowtail}

\newcommand\Gm{\Gamma}
\newcommand\Om{\Omega}

\newcommand\nil\cdot

% just to fool TexStudio
\providecommand\inferrule{}

\newcommand\rto{\longrightarrow}                % "reduce to" arrow ---`
\newcommand\redto{\longrightarrow^*}            % ---`*
\newcommand\rrule[1]{~\longrightarrow_{\makebox[0pt][l]{$\scriptstyle#1$}\hphantom{00}}}

\newcommand\jExt{\rightharpoonup}

\newcommand{\tRed}[1]{\textcolor{red}{#1}}

\newcommand{\lam}{} %[2]{\lambda {#1}.~{#2}}
\RenewDocumentCommand \lam {O{x} m} {\lambda {#1}.~{#2}}  % \lambda x.~e --- x is optional. \lam[x]{e}

\newcommand{\all}{} %[2]{\forall {#1}.~{#2}}
\RenewDocumentCommand \all {O{a} m} {\forall {#1}.~{#2}}  % \forall a.~A --- a is optional. \all[a]{A}

\newcommand{\appInf}[3]{{#1}\bullet{#2}\TTo{#3}}
\newcommand{\appInfAlg}{} %[4]{{#1}\bullet{#2}\TTo_{#3} {#4}}
\RenewDocumentCommand \appInfAlg {m m O{a} O{\jg}} {{#1}\bullet{#2}\TTo_{#3} {#4}}

\newcommand\al{}
\newcommand\bt{}

\RenewDocumentCommand \al {O{}} {\wh\alpha_{#1}}   % \widehat{\alpha} --- subscript is optional \al[1]
\RenewDocumentCommand \bt {O{}} {\wh\beta_{#1}}    % \widehat{\beta}

\newcommand{\blue}[1]{\textcolor{blue}{#1}}



% https://tex.stackexchange.com/questions/346870/bf-is-an-undefined-command
\DeclareOldFontCommand{\bf}{\normalfont\bfseries}{\mathbf}

\newcommand{\Description}[1]{}

% ITP used
\newcommand\exps{\Omega}
\newcommand\jcons[2]{#1 ; #2}


\mathchardef\dash="2D
% HM Let ... in
\newcommand{\letin}[2]{\textbf{let } x={#1} \textbf{ in } {#2}}
